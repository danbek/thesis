\chapter{Subarray Characterization}\label{c:det-array}

The first 251-detector subarray has not yet been characterized as thoroughly as the prototype detectors.
In particular, only good detectors with working heaters have been characterized.
The reason for this is that the saturation power of the detectors is xxx time higher than expected, which means that only some detectors be driven normal, and only at bath temperatures very close to Tc.
In particular, none of the detector can be driven normal at the operating bath temperature for the array of 1100~mK.
This means that we do not have IV curves for any detectors at operating conditions, which limits the characterization than can be performed.

Fourteen detectors used in taking images have working heaters.
These heaters allow us to take full IV curves at all bath temperatures, and to directly measure detector responsivity and time constants, and thus to measure detector noise referred to optical power incident on the detector.
This chapter describes these measurements on these fourteen detectors.
% xxx list the specific measurements

\section{Shunt Resistance Measurements}

Our shunt resistors are located on interface chips that contain both shunt resistors and Nyquist inductors.
The specific chips used were extra chips leftover from the ABS\cite{EssingerHileman:2010hh} project.
The design resistance of the shunts was 180~\uOhm, and the design inductance was 609~nH\footnote{John Appel, personal communication}.
Each chips contains 32 shunt resistors and 32 inductors.

To measure $R_{sh}$ and $L$ for these chips I took noise measurements using zero detector bias current at two different bath temperature: 980~mK and 1160~mK.
At these bath temperatures and at zero detector bias the detectors are superconducting, so that measured noise is due to the shunt resistor, and parasitic resistance, and SQUID noise in the multiplexed readout system itself.
Data was collected at 3030.3~Hz, and 20 data acquisitions lasting 33 seconds were taken at each bath temperature.

A power spectrum was estimated for each detector for each data acquisition using \MATLAB's \texttt{pwelch} function, using a \FFT\ size of $2^{12}$.
Each resulting power spectrum was fit to a function of the form
\begin{eqnarray}\label{eq:scnoise-fit}
	\frac{4 k_B T_b}{R_{sh}} \frac{1}{1 + (2 \pi f (L/R_{sh}))^2} + SQ,
\end{eqnarray}
where $k_B$ is Boltzmann's constant, $T_b$ is the bath temperature for the measurement, and $f$ is the frequency.
The shunt resistance $R_{sh}$, inductance $L$, and readout chain white noise level $SQ$ are the values extracted from the fit.

\figref{fig:rsh-l-plots} shows histogram plots of the resulting \Rsh\ and $L$ values.
\Rsh\ has a mean of 149~\uOhm\ with standard deviation 6~\uOhm.
The meassured value for \Rsh\ also includes any parasitic resistance in the circuit, but no evidence for significant parasitic resistance has ever been seen, so this parasitic resistance is assumed to be zero.

The measured value for $L$ includes the Nyquist inductance on the interface chip, the input inductance of the first-stage \SQUID\ of the multiplexed readout system, as well as any parasitic inductance in the circuit.
Using this measurement it is not possible to extract the inductance of the Nyqust inductor itself, but this is not a problem because the total inductance is the relevant  quantity for understanding the behavior of the detector and it's circuit.

$L$ has a mean of 574~nH with a standard deviation of 87~nH.
However, this mean includes two sets of clear outliers: all values for multiplexing row 4 are clustered around 200~nH, and all values for multiplexing row 25 are clustered around 440~nH.
The reason for these low inductances is not understood.
Excluding rows 4 and 25,  $L$ has a mean of 593~nH with a standard deviation of 44~nH.

The outlier $L$ values are more clearly visible in the lower left plot in \figref{fig:rsh-l-plots}, which also shows a small correlation between \Rsh\ and $L$.
It is not known whether this correlation is a real physical effect or an artifact of the measurement process.

\begin{figure*}
\documentclass{standalone}
\usepackage{tikz} 
\usepackage{pgfplots} % drawing plots right here in this file!
\pgfplotsset{compat=1.8} % latest stable release

\newcommand*{\RCm}[2]{
	\newcount\tmpR
	\newcount\tmpC
	\tmpR=#1
	\tmpC=#2
	\advance\tmpR by -1
	\advance\tmpC by -1
	R{\number\tmpR}C{\number\tmpC}
} % this allows me specify row/col indices MATLAB-style

\providecommand*{\here}{.}

\begin{document}

%\pgfplotsset{small}

\begin{tabular}{rl}

	\begin{tikzpicture}[baseline,trim axis left]
	\begin{axis}[ymin=0, 
						title={Measured Shunt Resistance},
						xlabel={$R_{sh}$ ($\mu\Omega$)}, ylabel={Quantity} ]
		\addplot [hist={data min=125,data max=165,bins=20},
			   fill=blue!75]
			table[y=Rsh]{\here/rsh-L.dat};
	\end{axis}
	\end{tikzpicture}

&

	\begin{tikzpicture}[baseline]
	\begin{axis}[ymin=0, %yticklabel pos=right,
						title={Measured Nyquist Inductance},
						xlabel={$L_{ny}$ (nH)}, ylabel={Quantity} ]
		\addplot [hist={data min=180,data max=640,bins=23},
			          fill=blue!75]
			table[y=L]{\here/rsh-L.dat};
	\end{axis}
	\end{tikzpicture}

\\
	\begin{tikzpicture}[baseline,trim axis left]
	\begin{axis}[ 
						title={Measured $R_{sh}$ vs $L_{ny}$},
						xlabel={$R_{sh}$ ($\mu\Omega$)}, ylabel={$L_{ny}$ (nH)} ]
		\addplot+[only marks]
			table[x=Rsh,y=L]{\here/rsh-L.dat};
	\end{axis}
	\end{tikzpicture}

&
	\begin{tikzpicture}[baseline]
	\begin{loglogaxis}[ %yticklabel pos=right,
								clip mode=individual,xmin=4e-1,xmax=2e3,
								title={\RCm{20}{6} Superconducting Current Noise},
								xlabel={Frequency (Hz)}, ylabel={Current Noise (A/$\sqrt{\mbox{Hz}}$)} ]

		\addplot+[only marks,mark size=1.0pt]
			table[x=f,y expr=sqrt(\thisrow{Pxx})]{\here/rsh-pxx.dat};
		\addlegendentry{Data}

		\addplot[mark=none,domain=4e-1:2e3,line width=1pt]
			{sqrt(1.38e-23 * 4 * 0.980  / 1.5290e-04 / (1 + (2*3.14159*x*5.0658e-3)^2) + 1.4308e-20)};
		\addlegendentry{Fit}
	\end{loglogaxis}
	\end{tikzpicture}
\\
\end{tabular}

\end{document}

\caption{Plots summarizing results of measurements of shunts and Nyqust inductors.
\textbf{Upper Left} Histogram of shunt resistance \Rsh.
\textbf{Upper Right} Histogram of total inductance in circuit, which includes the interface chip Nyquist inductor, the inductance of the SQ1 input coil, and any parasitic inductance.
\textbf{Lower Left} Scatterplot showing all measured Rsh and L values. A correlation is clearly visible, the explanation for which is not understood.
\textbf{Lower Right} Plot showing current noise power spectrum extracted from a single data acquisition for \RCm{20}{6}, along with predicted power spectrum based on best fit to \eqref{eq:scnoise-fit} across all data acquisitions. The best fit values are \Rsh\ = 153 \uOhm, $L$ = 622 nH, and \SQUID\ white noise level of 1.2e-10~ A/$\sqrt{\mbox{Hz}}$.}
\label{fig:rsh-l-plots}
\end{figure*}

\section{Measurement of Heater Resistors}

Only 23 of the 251 detectors have heaters wired up.
Of these 23, there are nine detectors which show no response to applied heater power.
Five of these nine are on the cut list (xxx see ch xxx), so this is not surprising.
But four detectors can be biased into the transition and work well, but show no response to applied heater power (\RC{4}{7}, \RC{5}{7}, \RC{6}{6}, \RC{7}{6}).
The reasons for these detectors not showing a response to heater power are not understood.
This leaves 14 working detectors that also show a response to applied heater power.
The heaters on these 14 detectors can be used to directly measure the detector responsivity, noise referred to input optical power, time constants and thermal conductance $G$.
But all of these measurements require knowing the resistance of the heaters.
This section describes my measurement of the heater resistances.

To measure the heater resitances, I took a set of \IV\ curves at constant bath temperature.
For each \IV\ curve I applied a different heater current, so that a different amount of heater power was applied to the \TES\ for each curve.
The total amount of power flowing through the \TES\ thermal conductance $G$ is then given by
\begin{equation}\ref{eq:tes-ptot}
P_{tot} = K(T^n - T_b^n) = P_{opt} + P_{htr} + I^2 R(T,I),
\end{equation}
where $T$ is the temperature of the \TES, $R$ is the resistance of the \TES, and $I$ is the current flowing through the \TES.
I make the assumption that at the start of the superconducting transition $\beta_I = 0$, i.e.\ the resistance of the \TES\ depends only on the \TES\ temperature, and not on the current through the \TES.
This implies that each time the \TES\ reaches, e.g.\ 99 \% of $R_n$, the temperature of the \TES\ is the same, and therefore $P_{tot}$ will be the same.
A relationship of the following form must therefore hold:
\begin{equation}\label{eq:rhtr-fit}
P_{J} = (P_{tot} - P_{opt}) - I_{htr}^2 R_{htr}.
\end{equation}
A fit can be made to this function, with the quantities $R_{htr}$ and $P_{x} \equiv P_{tot} - P_{opt}$ to be solved for.

\figref{fig:heater-r-plots} explains this idea further by a series of plots.
The upper left plot shows the \TES\ \IV curves.
The upper right plot shows the same data, but transformed into \TES\ Joule power and \TES\ resistance.
As applied heater current decreases, the Joule power at the start of the transition decreases.
In the lower left, the Joule power at 99 \% $R_{n}$ is plotted vs applied heater current.
A fit to \eqref{eq:rhtr-fit} is also plotted.
Finally, the lower right plot show the $R$ vs $P_J$ plots after the heater power has been added to each curve.
This plots shows that the powers are equalized very high in the transition, where the assumption of Joule power dependent only on \TES\ resistance holds.
It also shows that this assumption breaks down deeper in the transition.

\begin{figure*}
\documentclass{standalone}
\usepackage{tikz} 
\usepackage{pgfplots} % drawing plots right here in this file!
\pgfplotsset{compat=1.8} % latest stable release

\newcommand*{\RCm}[2]{
	\newcount\tmpR
	\newcount\tmpC
	\tmpR=#1
	\tmpC=#2
	\advance\tmpR by -1
	\advance\tmpC by -1
	R{\number\tmpR}C{\number\tmpC}
} % this allows me specify row/col indices MATLAB-style
\newcommand*{\TES}{{\small TES}}
\newcommand*{\uA}{\ensuremath{\mu}A}
\newcommand*{\uV}{\ensuremath{\mu}V}
\newcommand*{\uW}{\ensuremath{\mu}W}
\newcommand*{\Ohm}{\ensuremath{\Omega}}
\newcommand*{\mOhm}{m\ensuremath{\Omega}}
\newcommand*{\uOhm}{\ensuremath{\mu\Omega}}

\providecommand*{\here}{.}

\begin{document}

\begin{tabular}{rl}

	\begin{tikzpicture}[baseline,trim axis left]
	\begin{axis}[xmin=-0.05,ymin=0, 
						title={IV Curves for \RCm{29}{1}},
						xlabel={Voltage Across \TES\ (\uV)}, ylabel={Current Through \TES\ (\uA)} ]
		\addplot+[mark=none] table[x=Vtes01,y=Ites01]{\here/heater-r.dat};
		\addplot+[mark=none] table[x=Vtes02,y=Ites02]{\here/heater-r.dat};
		\addplot+[mark=none] table[x=Vtes03,y=Ites03]{\here/heater-r.dat};
		\addplot+[mark=none] table[x=Vtes04,y=Ites04]{\here/heater-r.dat};
		\addplot+[mark=none] table[x=Vtes05,y=Ites05]{\here/heater-r.dat};
		\addplot+[mark=none] table[x=Vtes06,y=Ites06]{\here/heater-r.dat};
		\addplot+[mark=none] table[x=Vtes07,y=Ites07]{\here/heater-r.dat};
		\addplot+[mark=none] table[x=Vtes08,y=Ites08]{\here/heater-r.dat};
		\addplot+[mark=none] table[x=Vtes09,y=Ites09]{\here/heater-r.dat};
		\addplot+[mark=none] table[x=Vtes10,y=Ites10]{\here/heater-r.dat};
		\addplot+[mark=none] table[x=Vtes11,y=Ites11]{\here/heater-r.dat};
		\addplot+[mark=none] table[x=Vtes12,y=Ites12]{\here/heater-r.dat};
		\addplot+[mark=none] table[x=Vtes13,y=Ites13]{\here/heater-r.dat};
		\addplot+[mark=none] table[x=Vtes14,y=Ites14]{\here/heater-r.dat};
		\addplot+[mark=none] table[x=Vtes15,y=Ites15]{\here/heater-r.dat};
	\end{axis}
	\end{tikzpicture}

&

	\begin{tikzpicture}[baseline,trim axis right]
	\begin{axis}[ymin=0, 
						title={$R$ vs $P_{J}$ for \RCm{29}{1}},
						xlabel={Joule Power Dissipated in \TES\ (pW)}, ylabel={\TES\ Resistance (\mOhm)} ]
		\addplot+[mark=none] table[x=Ptes01,y=Rtes01]{\here/heater-r.dat};
		\addplot+[mark=none] table[x=Ptes02,y=Rtes02]{\here/heater-r.dat};
		\addplot+[mark=none] table[x=Ptes03,y=Rtes03]{\here/heater-r.dat};
		\addplot+[mark=none] table[x=Ptes04,y=Rtes04]{\here/heater-r.dat};
		\addplot+[mark=none] table[x=Ptes05,y=Rtes05]{\here/heater-r.dat};
		\addplot+[mark=none] table[x=Ptes06,y=Rtes06]{\here/heater-r.dat};
		\addplot+[mark=none] table[x=Ptes07,y=Rtes07]{\here/heater-r.dat};
		\addplot+[mark=none] table[x=Ptes08,y=Rtes08]{\here/heater-r.dat};
		\addplot+[mark=none] table[x=Ptes09,y=Rtes09]{\here/heater-r.dat};
		\addplot+[mark=none] table[x=Ptes10,y=Rtes10]{\here/heater-r.dat};
		\addplot+[mark=none] table[x=Ptes11,y=Rtes11]{\here/heater-r.dat};
		\addplot+[mark=none] table[x=Ptes12,y=Rtes12]{\here/heater-r.dat};
		\addplot+[mark=none] table[x=Ptes13,y=Rtes13]{\here/heater-r.dat};
		\addplot+[mark=none] table[x=Ptes14,y=Rtes14]{\here/heater-r.dat};

	\end{axis}
	\end{tikzpicture}

\\

	\begin{tikzpicture}[baseline,trim axis left]
	\begin{axis}[xmin=2.8,xmax=5,
						title={\TES\ $P_{J}$ vs Heater Current},
						xlabel={$I_{htr}$ (\uA)}, ylabel={$P_{tes}$ (pW)} ]
		\addplot+[only marks]
			table[x=Ihtr,y=Psat]{\here/heater-r-pows.dat};
		\addlegendentry {Data}

		\addplot+[mark=none,domain=2.8:5] {597.7 - 23.5927*x^2)}
			node[right] at (axis cs:3,150) {$R_{htr}$ = 23.6 \Ohm}
			node[right] at (axis cs:3,100) {$P_{sat}$ = 598 pW}
			;
		\addlegendentry{Fit}

	\end{axis}
	\end{tikzpicture}

&

	\begin{tikzpicture}[baseline,trim axis right]
	\begin{axis}[ymin=0, 
						title={$R$ vs $P_{J} + P_{htr}$ for \RCm{29}{1}},
						xlabel={Total Power Dissipated in \TES\ (pW)}, ylabel={\TES\ Resistance (\mOhm)} ]
		\addplot+[mark=none] table[x=Ptot01,y=Rtes01]{\here/heater-r.dat};
		\addplot+[mark=none] table[x=Ptot02,y=Rtes02]{\here/heater-r.dat};
		\addplot+[mark=none] table[x=Ptot03,y=Rtes03]{\here/heater-r.dat};
		\addplot+[mark=none] table[x=Ptot04,y=Rtes04]{\here/heater-r.dat};
		\addplot+[mark=none] table[x=Ptot05,y=Rtes05]{\here/heater-r.dat};
		\addplot+[mark=none] table[x=Ptot06,y=Rtes06]{\here/heater-r.dat};
		\addplot+[mark=none] table[x=Ptot07,y=Rtes07]{\here/heater-r.dat};
		\addplot+[mark=none] table[x=Ptot08,y=Rtes08]{\here/heater-r.dat};
		\addplot+[mark=none] table[x=Ptot09,y=Rtes09]{\here/heater-r.dat};
		\addplot+[mark=none] table[x=Ptot10,y=Rtes10]{\here/heater-r.dat};
		\addplot+[mark=none] table[x=Ptot11,y=Rtes11]{\here/heater-r.dat};
		\addplot+[mark=none] table[x=Ptot12,y=Rtes12]{\here/heater-r.dat};
		\addplot+[mark=none] table[x=Ptot13,y=Rtes13]{\here/heater-r.dat};
		\addplot+[mark=none] table[x=Ptot14,y=Rtes14]{\here/heater-r.dat};

	\end{axis}
	\end{tikzpicture}
\\
\end{tabular}

\end{document}

\caption{Plots heater measurements, for the case of \RCm{29}{1}.
\textbf{Upper Left} \IV\ curves. The \IV\ curves should turn completely vertial when the detector becomes fully superconducting at zero voltage, but these curves shown a non-infinite slope. The reason for this is that the readout system as configured for these \IV\ curves was unable keep up with the rapid change of current in the superconducting branch.
\textbf{Upper Right} Same data as in upper left plot, but represented in terms of \TES\ Joule power and resistance. As the bias current for the heaters is increased, the curves shift to the left.
\textbf{Lower Left} Measured $P_{J}$ vs heater current at 99\% $R_n$, as well as fit to \eqref{eq:rhtr-fit}.
\textbf{Lower Right} Same plot as upper right, but the heater power based on $R_{htr} = 23.6 \Ohm$ has been added to each curve.}
\label{fig:heater-r-plots}
\end{figure*}

\tableref{tab:all-heater-r} lists all measured heater resistors.
The seven heaters for columns 0 and 1 have a mean of 23.8~\Ohm\ with a standard deviation of 0.26~\Ohm.
The seven heaters for column 6 and 7 exhibit much more scatter, with a mean of 28.6~\Ohm\ with a standard deviation of 4.16~\Ohm.
The reason for the difference between the columns pairs is unknown.

\begin{table*}[t]
\centering
\caption{Measured Heater Resistances}
\label{tab:all-heater-r}
\begin{tabular}{l r}
\toprule
Detector &  $R_{htr}$ (\Ohm) \\
\midrule
\RCm{1}{7}   & 27.7 \\
\RCm{2}{7}   & 31.6 \\
\RCm{3}{7}   & 31.2 \\
\RCm{3}{8}   & 19.8 \\
\RCm{4}{7}   & 28.4 \\
\RCm{4}{8}   & 31.0 \\
\RCm{5}{7}   & 30.5 \\
\RCm{29}{1}   & 23.6 \\
\RCm{29}{2}   & 23.8 \\
\RCm{30}{1}   & 23.5 \\
\RCm{31}{1}   & 24.3 \\
\RCm{31}{2}   & 23.6 \\
\RCm{32}{1}   & 23.9 \\
\RCm{32}{2}   & 23.8 \\
\bottomrule
\end{tabular}
\end{table*}

\section{Measurement of \TES\ $G$}

With knowledge of the heater resistances, \IV\ curves can be taken over a wide range of bath temperatures, which enables a measurement of the \TES\ thermal conductance $G$.
Similarly to \sectionref{xxx}, I took \IV\ curves at bath temperatures ranging from xxx mK -- xxx mK, while adjusting the applied heater power so that each \IV\ curve had a clear normal branch.
I again used the asumption that high in the transition the \TES\ resistance depends on only the \TES\ temperature, so that the Joule power $P_J$ disippated in the \TES\ at 99 \% $R_n$ depends only on the bath temperature $T_b$ and the applied heater power $P_{htr} = I_{htr}^2 R_{htr}$. A fit can then be made to \eqnref{eq:tes-ptot} in the form
\begin{equation}\ref{eq:g-fit}
P_{htr} + P_J + P_{opt}= \frac{G T_c}{n}(1 - \left(\frac{T_b}{T_c}\right)^n).
\end{equation}
The parameters to be fit to are $G$, $T_c$, and $n$.

A problem arrises because the data described in this section were taken when the cryostat was open, so that $P_{opt}$ was non-zero, with an unknown value.
Because $P_{opt}$ is a simple additive constant, it is not possible to fit for this value unless another constraint (such as the value of $T_c$) is known.
However, the value of $P_{opt}$ can be estimated in two different ways. 
First, the predicted optical load of 165~pW from \sectionref{xxx} can be used.
Second, the measured optical load of 


