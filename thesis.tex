% !TEX TS-program = pdflatex
% !TEX encoding = UTF-8 Unicode

% for proofreading the text
%\documentclass[10pt,twocolumn,article,draft]{memoir}
\documentclass[10pt,twocolumn,article]{memoir}

%% closer to real thesis format
%\documentclass[12pt,letterpaper]{memoir}
%\OnehalfSpacing

% 1in margins everywhere. I do this for proofreading too, so that I can see exactly how the figures will look on their pages in the final document
\settypeblocksize{9in}{6.5in}{*}
\setlrmargins{1.0in}{*}{*}
\setulmargins{1.0in}{*}{*}
\checkandfixthelayout

\usepackage[utf8]{inputenc} % set input encoding to utf8

\usepackage[sc]{mathpazo} % Palatino with math support

\usepackage{url} % typesetting urls
\usepackage[kerning]{microtype} % better typography
\usepackage{booktabs} % better tables
\usepackage{graphicx} % including graphical figures
\usepackage{xspace} % fix trailing spaces after abbreviation macros
\usepackage{tikz} % drawing figures right here in this file!

% Bibliography
\usepackage[backend=biber,natbib=true,style=numeric-comp]{biblatex}
\addbibresource{thesis.bib}

% My macros
\newcommand*{\figref}[1]{Figure~\ref{#1}}
\newcommand*{\tableref}[1]{Table~\ref{#1}}
\newcommand*{\sectionref}[1]{Section~\ref{#1}}
\newcommand*{\TES}{{\small TES}\xspace}
\newcommand*{\NETD}{{\small NETD}\xspace}
\newcommand*{\He}[1]{$^{#1}$He\xspace}
\newcommand*{\uW}{\ensuremath{\mu}W\xspace}
\newcommand*{\Ohm}{\ensuremath{\Omega}\xspace}
\newcommand*{\Imager}{350~GHz Video Imager\xspace}

\title{A 350 GHz Video Imaging System}
\author{Dan Becker}
%\date{} % Delete this line to display the current date

%%% BEGIN DOCUMENT
\begin{document}

\maketitle

%%%\begin{abstract}
%%%Passive millimeter-wavelength video imaging systems hold promise for detection of security threats at a distance, such as including suicide bomb belts and maritime threats in fog.
%%%Achieving optimal noise and optical performance for these system requires large numbers of cryogenic millimeter-wavelength radiation detectors. Large-format arrays of superconducting Transition Edge Sensor (TES) bolometers have been proven to meet requirement for both noise and number of detectors.
%%%We are developing a video- rate millimeter-wavelength imaging system using 1004 TES bolometers as detectors.
%%%This demonstration system detects is intended to have photon-noise-limited performance, and will be used to investigate phenomenology of passive millimeter-wavelength video images, with the goal of identifying what performance tradeoffs can be made when building a deployable system.
%%%It observes light in a 10\% band centered at 350 GHz, and is designed to take video images at distances ranging from 16 m to 28 m.
%%%When operating at 16 m, the resolution is 1 cm over a 1 m by 1 m field of view.
%%%The system is predicted to take video images with a noise equivalent temperature difference (NETD) of 100 mK at 20 frames per second.
%%%This thesis describes the design and implementation of this system, as well as imaging results from the first 251-detector subarray to be installed.
%%%\end{abstract}

%\tableofcontents* % the asterisk means that the contents itself isn't put into the ToC

%%%(211 words, 350 allowed)

\chapter{Introduction}\label{c:intro}

\chapter{System Specifications, Challenges and Solutions}\label{c:specs}

\chapter{TES Bolometer Theory}\label{c:tes}

\chapter{System Design Overview}\label{c:sys-design}

\section{Cryostat Design}\label{s-cryo-design}

The cryostat for the \Imager was designed with the goals of simplicity, reliability and turn-key automated operation.
Highly reliable and easy-to-use cryogen-free mechanical cryocoolers are available from many vendors, but these cryocoolers are seldom capable of reaching temperatures below 2.5~K.
Reaching sub-Kelvin temperatures requires a second refrigeration stage, which in our case is a He-4 sorption refrigerator.
The He-4 sorption refrigerator is based on a proven design and its use can easily be automated.
Three temperature stages within the cryostat are provided in order to provide intercepts for heatsinking wiring and other objects that are thermal connected to room temperature. 
The result is a reliable cryogen-free cryogenic system that can be controlled remotely.

The cryostat itself was built by Precision Cryogenics\footnote{Precision Cryogenics Systems, Inc. Indianapolis, IN. \url{http://www.precisioncryo.com}} to designs provided by the \Imager team.
\figref{fig:cryo-cutaway} shows a cutaway view of the cryostat, and \tableref{tab:temp-optical-load} lists the temperatures typically reached by different parts of the cryostat during operation when the cryostat is open optically.
The cryostat has two main parts: a cylinder containing both the PTC and the \He4-sorption refrigerator, and a box located at the bottom of the cylinder which contains temperature intercept plates and the focal plane.
There are three temperature stages, the ``90~K'' Cold Plate, the ``4~K'' Cold Plate, and the Focal Plane.
The PTC 1st stage is connected to the ``90~K'' Cold Plate by a tube of Al 1100 and a set of CDA-101 Cu braids.
The combination of this long thermal path with the high heat load on the optical filters sunk to the ``90~K'' stage explains the 45~K temperature differential between the ``90~K`` cold plate and the PTC 1st stage.
The PTC 2nd stage is connected to the ``4 K'' Cold Plate by a large (3.0 in diameter by 2.78 in long) cylinder of CDA-110 Cu\footnote{This cryostat was originally designed to work with a different cryocooler. The PTC currently installed had a shorter distance between the 1st and 2nd stages, necessitating the Cu cylinder to take up this extra space}, followed by tube of alloy CDA-101 Cu followed by a set of  CDA-101 copper braids.
The Cu tube is broken into two halves, and the condensation plate (see below) of the sorption fridge is clamped between these two halves. The ``90 K'' Cold Plate is stood off from the cryostat vacuum jacket by four ``roll wrapped'' carbon fiber tube standoffs. The ``4~K'' Cold Plate stands off from the ``90 K'' Cold Plate by eight supports made of G-10.

The first two temperature intercept stages are provided by a Cryomech PT407 Pulse Tube Cryorefrigerator\footnote{Cryomech, Inc. Syracure, NY. \url{http://www.cryomech.com}}
The PT407 has two cooling stages.
The first stage has 25~W of cooling power at 55~K while the second stage has 0.7~W at 4.2 K.
Our PT407 uses a remote motor, so that the cold head attached to the cryostat has no moving parts, minimizing vibration of the cryostat.
Vibration of the cryostat can lead to microphonic pickup either directly in the detectors themselves or in the readout circuitry, leading to much higher detector noise.

\begin{figure*}[t]
\centering
\begin{tikzpicture}
    \node[anchor=south west,inner sep=0] (image) at (0,0) {\includegraphics[width=2.6in]{images/cryostat-cutaway.png}};
    \begin{scope}[x={(image.south east)},y={(image.north west)}]
	    %\draw[help lines,xstep=.1,ystep=.1] (0,0) grid (1.5,1);
		%\foreach \x in {0,1,...,15} { \node [anchor=north] at (\x/10,0) {0.\x}; }
		%\foreach \y in {0,1,...,9} { \node [anchor=east] at (0,\y/10) {0.\y}; }
		
        \draw[red,ultra thick,rounded corners] (0.50,0.7) rectangle (0.85,0.75) node[below left] {\textbf{A}}; % PTC407 1st stage CH
        \draw[red,ultra thick,rounded corners] (0.56,0.595) rectangle (0.71,0.64) node[below left] {\textbf{B}}; % PTC407 2st stage CH
        \draw[red,ultra thick,rounded corners] (0.53,0.55) rectangle (0.78,0.595) node[below left] {\textbf{C}}; % Cu Cylinder
        \draw[red,ultra thick,rounded corners] (0.505,0.32) rectangle (0.75,0.55) node at +(-0.1,-0.03) {\textbf{D}}; % sorp fridge
        \draw[red,ultra thick,rounded corners] (0.4,0.06) rectangle (0.70,0.26) node[red,below left] {\textbf{E}}; % Focal Plane
        
        \draw[thick,<->] (1.05,0.04) -- +(0,0.31) node[midway,right] {18 in}; % scale bar
    \end{scope}
\end{tikzpicture}
\caption{Cutaway view of the 350 GHz Imager. \textbf{A} PT407 1st stage cold head \textbf{B} PT407 2nd stage cold head \textbf{c} Cu cylinder connected PT407 2ns stage cold head to Cu tube, which then connects to \He4-sorption refrigerator condensation plate. \textbf{D} \He4-sorption refrigerator \textbf{E} Focal Plane. Copper ropes connecting focal plane to 1~K cold plate are not visible in this view.}
\label{fig:cryo-cutaway}
\end{figure*}

\begin{table*}[t]
\centering
\caption{Temperatures Reached Under Optical Load xxx should I add temps when closed optically?}
\label{tab:temp-optical-load}
\begin{tabular}{l r}
\toprule
Temperature Stage &  Temperature (K)\\
\midrule
PTC 1st Stage Cold Head 			& 52 \\
PTC 2st Stage Cold Head 			& 3.6 \\
Cryostat ``50 K'' Cold Plate 		& 96 \\
Cryostat ``4 K'' Cold Plate 			& 6.4 \\
Sorption Fridge Condensation Plate 	& 3.8 \\
Focal Plane 						& 0.970 \\
\bottomrule
\end{tabular}
\end{table*}

Options for reaching temperatures below the $\sim$1.2~K transition temperature of our \TES detectors include: dilution refrigerators, adiabatic magnetization refrigerators, pumped \He4 baths, and \He3- and/or \He4-sorption refrigerators.
We chose a \He4-sorption fridge because of both it's low and cost ease of operation compared to other solutions, and the fact that the typical base temperatures under no load of $\sim$700~mK is well-matched to our application.
A  \He4-sorption fridge works by using a charcoal adsorber to pump on a bath of liquid \He4, reducing the \He4 boiling point and thus the temperature of the bath.
The \He4 is contained withing a sealed reservoir so that the refrigerator acts as a closed system requiring no \He4 replenishment. 
While \He4-sorption fridges are commercially available, our team choose to design and build a custom fridge based on a design that has been proven in astronomical applications\cite{devlin2004high}.

\figref{fig:he4sorp} shows a schematic depiction of the 350~GHz Imager's \He4-sorption refrigerator.
The entire refrigerator is filled with 2.07~moles of \He4 gas, giving a pressure of 900~psi at room temperature.
In normal operation the heat switch between the charcoal pumping chamber (``pump'') and the \He4 condensation plate is closed, keeping the charcoal as cold as possible in order to adsorb as much \He4 as possible, keeping the temperature of the cold plate as low as possible.

\begin{figure*}[th]
\centering
\begin{tikzpicture}
    \node[anchor=south west,inner sep=0] (image) at (0,0) {\includegraphics[width=3.0in]{images/he4-sorp-fridge-cutaway.png}};
    \begin{scope}[x={(image.south east)},y={(image.north west)}]
	    %\draw[help lines,xstep=.1,ystep=.1] (0,0) grid (1.5,1);
		%\foreach \x in {0,1,...,15} { \node [anchor=north] at (\x/10,0) {0.\x}; }
		%\foreach \y in {0,1,...,9} { \node [anchor=east] at (0,\y/10) {0.\y}; }
        \draw[blue,ultra thick,rounded corners] (0.10,0.56) rectangle (0.86,1.01) node[below left] {\textbf{A}}; % pump
        \draw[blue,ultra thick,rounded corners] (0.02,0.56) rectangle (1.03,0.44) node[above left] {\textbf{B}}; % cond plate
        \draw[blue,ultra thick,rounded corners] (0.70,0.56) rectangle (0.88,0.36) node[above left] {\textbf{C}}; % heat switch
        \draw[blue,ultra thick,rounded corners] (0.35,0.25) rectangle (0.72,-0.02) node[above left] {\textbf{D}}; % pot
        
		\draw[thick,<->] (0.9,0.02) -- ++(0,0.21875) node[midway,right] {3 in}; % scale bar
    \end{scope}
\end{tikzpicture}
\caption{Cutaway view of the \He4-sorption refrigerator. \textbf{A} Charcoal pumping chamber (``pump''). The charcoal is attached to the concentric copper cylinders. Cylinders are used to maximize the surface area covered by the charcoal. \textbf{B} Condensation plate. This copper plate is kept below the boiling point of \He4 in order to provide a point in the refrigerator for \He4 to condense and drip into the condensation pot. \textbf{C} \He4 gas gap heat switch. This heat switch is used to cool the charcoal in order to pump on the \He4 bath in the condensation pot. \textbf{B} \He4 condensation pot (``pot''). The condensed \He4 accumulates here. The concentric cylinders provide additional surface area for thermal contact to the liquid \He4. Not shown is a small (xxx check size from Bob) constriction in the stainless steel tube connecting the pump and pot, intended to restrict the flow of superfluid \He4 away from the pot.}
\label{fig:he4sorp}
\end{figure*}

Cycling the refrigerator requires four steps.
First the heat switch is opened.
The \He4-sorption refrigerator uses a \He4 gas-gap heat switch manufactured by Chase Cryogenics\footnote{Chase Research Cryogenics, Ltd. Sheffield, UK}, which requires 5 minutes of waiting time in order for the switch to fully open.
Second, the pump is heated by applying 5 W via a 500 \Ohm power resistor.
This power is applied until the temperature of the pump reaches 40 K, which is high enough to drive nearly all of the adsorbed \He4 off of the charcoal.
Third, the power to the pump is turned off.
Once the temperature of the condensation plate falls below the boiling point of \He4, \He4 will begin to condense on it's walls, dripping into the \He4 condensation pot (``pot'').
Fourth, once the temperature of the pot has fallen to 4.0~K, the heat switch is turned back on.
This cools the pump, allowing \He4 to again adsorb onto the charcoal, which has the effect of pumping strongly on the pot, and cooling the \He4 contained there to the base temperature of $\sim$ 970~mK under optical load.
This process is easy to automate, and a LabView program cycles the fridge automatically every night while the system is operating.

Under optical load, a full cycle of the \He4-sorption refrigerator takes approximately 4 hours, reaching a base temperature of $\sim$~970~mK.
When no additional load is applied, such as to warm the focal plane to a typical operating temperature of 1100~mK, the hold time is 9 hours.
Shorter hold times are observed when the operating temperature is raised. % xxx should give an exact time here

\tableref{tab:fp-thermal-load} shows the predicted heat load on the \He4-sorption refrigerator.
It excludes parasitic load inherent to the sorption refrigerator itself. xxx mention parasitic load here?

xxx - include data on temp vs load for the fridge?

\begin{table*}[ht]
\centering
\caption{Predicted Thermal Load on 1~K Stage. These calculations assume that the readout wiring and Ti-6Al-4V spiders are running from 6.4~K to 1.0~K.}
\label{tab:fp-thermal-load}
\begin{tabular}{@{}lrr@{}}
\toprule
 & \multicolumn{2}{c}{Predicted Thermal Load} \\
\cmidrule(r){2-3}
Heat Load Source & 1004 Detectors (\uW) &  251 Detectors (\uW) \\
\midrule
Readout wiring 								& 50 & 125 \\
Series array SQUID modules 		& xxx & xxx \\
SQUID multiplexing chips 				& xxx & xxx \\
Detectors shunt resistors 			& 5 & 18 \\
Detectors (Optical + Electrical)		& 0.25 & 1 \\
Ti-6Al-4V spiders 							& 130 & 130 \\
Out-of-band optical power 			& xxx & xxx \\
\midrule
Total												& 185 + xxx & 274 + xxx \\
\bottomrule
\end{tabular}
\end{table*}

\section{Optical Design}\label{s:optical-design}

\section{Feedhorn Design}\label{s:feedhorn-design}

Millimeter and submillimeter astronomical instruments using \TES detectors use a number of different strategies for coupling light onto detectors: filled array \cite{act,scuba}, lenslet antennas \cite{polarbear?}, phased antenna arrays \cite{jpl stuff}, corrugated feedhorns \cite{sptpol,actpol}, and smooth-walled conical feedhorns \cite{apex-sz,spt}.
The \Imager uses smooth-walled feedhorns because they are easy to design and and easy to machine.
Smooth-walled feedhorns do not have the nice polarization properties of corrugated feedhorns \cite{claricoats & olver?}, but this is not a concern here because the \Imager is polarization-insensitive.

For an individual feedhorn, there is an size that will maximize coupling to the rest of the optics.
\figref{f:xxx} shows a schematic of a smooth-walled conical feedhorn and it's beam, along with the parameters of the feedhorn and beam.
The critical relationship is that as a feedhorn's opening diameter increases, the corresponding beam angle become smaller.
If the beam angle is very small, the feedhorn will see only a small fraction of the primary mirror, worsening spatial resolution.
If the beam angle is very large, most of the beam will spill off the primary mirror, reducing the 

The goal of the \Imager is to minimize \NETD.

As discussed in section \ref{xxx}, the more detectors a system has, the lower the \NETD will be.
However, the diffraction-limited focal plane area is limited, and set by the design of the optics (see \sectionref{s:optical-design}).
Under the constraint of fixed focal plane area, adding more feedhorns means making the opening diameter of each feedhorn smaller.
But as the feedhorn opening diameter becomes smaller, the feedhorns beam.
Mandana Amiri and Matthew Hasselfield provided extensive, timely and invaluable support for the MCE hardware and software.
 
\section{Acknowledgments}

Bob Schwall and William Duncan of NIST designed the cryostat/mirror mount and the \He4 sorption refrigerator.
The refrigerator was filled with \He4 by Simon Dicker at the University of Pennsylvania.
Bob Schwall of NIST designed the cryostat, and provided useful advice and help in the lab during commissioning of the cryostat.
William Duncan designed, procured, and assembled the optical system.

\chapter{Detector Design}\label{c:det-design}

\chapter{Focal Plane Design}\label{c:fp-design}

\chapter{Detector Characterization}\label{c:det-char}

\chapter{Imaging}\label{c:imaging}

\chapter{Vocab}

Placeholder for terms

\begin{description}
\item[``90~K'' Cold Plate]
\item[``4~K'' Cold Plate]
\item[Sorption Fridge]
\item[350~GHz Imager]

\end{description}

\printbibliography

\end{document}
