% !TEX TS-program = pdflatex
% !TEX encoding = UTF-8 Unicode

\documentclass[12pt,letterpaper,openany]{memoir}

% This is how you get rid of a (possible) blank page between
% frontmatter and main matter. See [1].
% [1]: http://tex.stackexchange.com/questions/153650/memoir-class-forces-recto-page-after-mainmatter
\usepackage{etoolbox}
\makeatletter
\patchcmd{\@smemmain}{\cleardoublepage}{\clearpage}{}{}
\patchcmd{\@smemmain}{\cleardoublepage}{\clearpage}{}{}
\makeatother

% Not double space, not single, but inbetween
\OnehalfSpacing

% 1in margins everywhere. The tricky part is that the "upper margin"
% refers to where the *text* starts, not the header, so you need to
% give some extra space for the header. Probably some way to
% explicitly set the distance to the header, but this is simple and
% close enough.

\setlrmarginsandblock{1.0in}{1.0in}{*}
\setulmarginsandblock{1.45in}{1.0in}{*}
\checkandfixthelayout

% See memoir manaul for this stye. It put page numbers in the footer.
% The header has chapter/section name and a ruled line.
\copypagestyle{my-ruled}{ruled}
  \makeevenfoot{my-ruled}{}{\thepage}{}
  \makeoddfoot{my-ruled}{}{\thepage}{}

\pagestyle{my-ruled}


\usepackage{thesis}

% show labels in printed document. Remove for ``real'' typesetting
%\usepackage{showkeys}

% number equations, tables, and figures by chapter
\numberwithin{equation}{chapter}
\numberwithin{figure}{chapter}
\numberwithin{table}{chapter}

% Bibliography
% For more bibliography format stuff, see http://www.khirevich.com/latex/biblatex/
\usepackage[
    backend=biber,
    style=numeric-comp,
    sorting=none,
    url=false,
    maxbibnames=100
    ]{biblatex}
\addbibresource{thesis.bib}

%%% BEGIN DOCUMENT
\begin{document}

\frontmatter

% Title page. All this code (and the signature page code) was copied
% from the "official" CU latex template [1].
% [1]: http://www.colorado.edu/oit/sites/default/files/thesis.cls

\newcommand{\thesistitle}{Submillimeter Video Imaging with a Superconducting Bolometer Array}

\thispagestyle{empty}
\vspace*{\fill}
\begin{center}
  \parbox{6.0in}{
    \begin{center}
      \DoubleSpacing
      \MakeUppercase{\thesistitle} \\
      by \\
      DANIEL THOMAS BECKER \\
      B.S., Rice University, 1994 \\
      M.S., University of Colorado, 2009
    \end{center}
  }
\end{center}
\vspace*{1in}
\begin{center}
  \DoubleSpacing
  A dissertation submitted to the\\
  Faculty of the Graduate School of the\\
  University of Colorado in partial fulfillment\\
  of the requirements for the degree of\\
  Doctor of Philosophy \\
  Department of Physics \\
  2014
\end{center}
\vspace*{\fill}
\newpage

% Signature page

\thispagestyle{empty}
\vspace*{\fill}
\begin{center}
  \SingleSpacing
  This dissertation entitled:\\
  \thesistitle{} \\
  written by Daniel Thomas Becker \\
  has been approved for the Department of Physics \\
\end{center}

\vspace*{2mm}

\begin{center}
  \normalsize
  \vspace*{16mm}
  \vrule width 80mm height 0.2mm\\
  Nils Halverson
\end{center}

\begin{center}
  \normalsize
  \vspace*{16mm}
  \vrule width 80mm height 0.2mm\\
  Kent Irwin
\end{center}

\vspace*{9mm}

\begin{flushright}
  Date~{\vrule width 55mm height 0.2mm}
\end{flushright}

\vspace*{\fill}
\begin{center}
  \SingleSpacing
  \parbox{4.50in}{
  The final copy of this dissertation has been examined by
  the signatories, and we find that both the content
  and the form meet acceptable presentation standards
  of scholarly work in the above mentioned discipline.}
\end{center}
\vspace*{\fill}
\newpage

% Now the abstract, in CU's format


\begin{flushleft}
Becker, Daniel Thomas (Ph.D., Physics)

\thesistitle

Dissertation directed by Associate Professor Nils Halverson
\end{flushleft}

\vspace*{0.7cm}

% use siunitx
Millimeter wavelength radiation holds promise for detection of security threats at a distance, including suicide bombers and maritime threats in poor weather.
The high sensitivity of superconducting Transition Edge Sensor (TES) bolometers makes them ideal for passive imaging of thermal signals at millimeter and submillimeter wavelengths.

I have built a 350 GHz video-rate imaging system using an array of feedhorn-coupled TES bolometers.
The system operates at standoff distances of 16 m to 28 m with a measured spatial resolution of 1.4 cm (at 17 m).
It currently contains one 251-detector sub-array, and can be expanded to contain four sub-arrays for a total of 1004 detectors.
The system has been used to take video images that reveal the presence of weapons concealed beneath a shirt in an indoor setting.

This dissertation describes the design, implementation and characterization of this system.
It presents an overview of the challenges associated with standoff passive imaging and how these problems can be overcome through the use of large-format TES bolometer arrays.
I describe the design of the system and cover the results of detector and optical characterization.
I explain the procedure used to generate video images using the system, and present a noise analysis of those images.
This analysis indicates that the Noise Equivalent Temperature Difference (NETD) of the video images is currently limited by artifacts of the scanning process.
More sophisticated image processing algorithms can eliminate these artifacts and reduce the NETD to 100 mK, which is the target value for the most demanding passive imaging scenarios.
I finish with an overview of future directions for this system.
%%%(264 words, 350 allowed)

\newpage
\begin{flushleft}\textbf{\Huge Acknowledgements}\end{flushleft}
\vspace{0.25in}

It is a truth universally acknowledged that a graduate student in possession of a draft dissertation must be in want of an editor.
I thank Nils Halverson, Bill Becker, and Linda Becker for carefully reading this dissertation and providing much useful feedback and advice.
Any remaining grammatical, spelling, or other errors are entirely my own.

I had help from many different people while performing the work described in this dissertation.
At some point in time, everyone associated with NIST's Quantum Sensors Project provided either useful advice as I worked through the design process, or gave concrete help in the lab.
The sections at the ends of Chapters 4 and 5 give the names of people who were particularly helpful with specific areas.
I would also like to thank Mike Niemack, who explained to me how to use the \MCE\ as well as giving useful advice regarding the design and testing of the detectors.
Conversations with Doug Bennett have been helpful in understanding \TES\ detectors in general.
Hannes Hubmayr was a useful resource for \TES\ detectors and submillimeter detection.
Randy Doriese helped me to understand \SQUID\ readout.
Carl Reintsema designed circuit boards for reading out the prototype detectors, and came up with the idea of reading out the mirror position using the \SQUID\ readout system.
Colin Fitzgerald, Jon Gard, Cale Gentry, and Bob Schwall all helped mount and unmount the cryostat from the primary mirror, a time-consuming and unpleasant task.

Of course none of this work would be possible without good advisors, and I thank Nils Halverson for giving me an initial opportunity to work in the field of submillimeter detection, and Kent Irwin for the opportunity to continue working in the field.
Both Nils and Kent have been very supportive of me, particularly in light of the challenges I have faced as a graduate student attempting to support a family with young children, and I deeply appreciate their support and help.

I am blessed with a supportive family and extended family. I particularly thank my parents for, well, everything, and my sister Kate for serving as an inspiration in many different ways.
During the months of writing this dissertation my children have had to put up with a father spending too much time writing in the basement, and I thank them for their patience with me.
Finally, and most importantly, I thank my wife, Terzah, who helped in many small as well as large ways, and without whose love and support none of this would be possible.

\newpage

\tableofcontents* % the asterisk means that the contents itself isn't put into the ToC

\newpage
\listoftables

\newpage
\listoffigures

\mainmatter

\chapter{Introduction} \label{c:intro}

% http://ieeexplore.ieee.org/stamp/stamp.jsp?tp=&arnumber=6005328
% Instead, the image is dominated by speckle that is characteristic for coherent imaging. Speckle comes from the large variations in the intensity of backscattered radiation coming from the diversity in angles of the beam-target incidence, and it dominates any difference in the intrinsic reflectivity of, say, PVC pipe material, clothing, and skin.

%% RFI HSHQDC-14-00015
%% commercially available standoff - Brijot Gen 2, SET CounterBomber
%% TSA 700 AIT (advanced imaging technology) systems at 160 airports

For over thirty years the millimeter and submillimeter-wavelength spectrum has been the subject of instense interest for military and security maging applications.
The reason for this interest is that the spectral region from \SIrange{100}{1000}{\GHz} offers a good compromise between transmission through obscurant materials (favoring lower frequencies) and spatial resolution (favoring higher frequencies) \cite{kruse_why_1981}.
This interest has driven a long line of technological advancement in sources, detectors, and other technologies at these wavelengths \cite{popovic_thz_2011}.
This advancement has taken place in both ``active'' imaging, in which an observation target is illuminated by light and the reflections from that target are recorder, and `passive'' imaging, in which ambient thermal emissions are detected.

Over a similar timespan the same time the millimeter and sub-millimeter astronomical community has also been very interested in these wavelengths.
In particular, the desire to make more and more sensitive maps of the Cosmic Microwave Background (\CMB) radiation have driven this communities need for instruments capable of higher and higher sensitivies.
During the 1980's and early 1990's cooled bolometric detectors capable of achieving photon-noise-limited --- or near photon-noise-limited --- performance were developed.
By 1994 it was clear that for many astromonical applications the only way to increase sensitivity was to develop arrays of bolometers, but at that time no technology for the production of monolithic arrays was yet available \cite{richards_bolometers_1994}.

The development of voltage-biased superconducting Transition-Edge-Sensor detectors enabled the development of large-scale detector array.
These detectors are based on the use of thin superconducing films as the bolometer's themometry element, and can fabricated at large scale using standard lithographic techniques.
Their operation and advantages for array-scale operation are described in \chapterref{c:tes}.
To readout these detector arrays, several groups have developed \SQUID-based multiplexed readout systems.
This technology is now mature and is routinely deployed on both ground and balloon-borne experiements in arrays containing up to 10,000 bolometric detectors \cite{holland_scuba-2:_2013}.

The development of this technology offers new opportunities for passive imaging for security and other applications
Specifically, it is now possible to develop focal planes capable of video-rate imaging with temperature resolution of \SI{100}{\mK} or below.
This thesis decribes the design and development of the \NIST\ \Imager, a system developed to perform detection of concealed weapons at distances of \SIrange{26}{28}{\m} by producing video-rate images at \SI{350}{\GHz}.
This chapter provides an overview of millimeter-wavelength imageing and the problems that our system is intended to solve.

\section{Security Imaging}

The frequency range \SIrange{100}{1000}{\GHz} is attractive for detection of concealed weapons or contraband because many common clothing materials have hi transmission in this range \cite{bjarnason_millimeter-wave_2004}.
In general, as shown in \figref{fig:ch1-clothes-atmos-trans}, transmission through clothing steadily decreases as frequency inncreases.
This trend tends to push systems toward lower frequencies.
An example likely familiar to readers L3 Provision (correct name?) systems operating at many airports within the USA.
These are radar systems operating at ~\SI{35}{\GHz}, intended for close-range portal screening.
For applications in which is is acceptable to require individuals to pass through and pause at a particular location, these systems have excellent image quality.
Although the ProVision system only takes still images with a xxx s image acquisition time, similar portal screen systems with video-rate capabilities are also under development (xxx ref system from talk in Dresden).

\begin{figure*}
\centering
\includegraphics{drawings/ch1-clothes-bjarnason.pdf}
\includegraphics{drawings/ch1-atmos-trans.pdf}
\caption[Clothing and Atmospheric Transmission vs Frequency]{
  \textbf{Left}
  Plot of clothing transmission vs frequency.
  Taken from \cite{bjarnason_millimeter-wave_2004}.
  As frequency increases, transmission through all kinds of clothing decreases.
  The \SI{-10}{\dB} observation band of the \Imager\ is highlighted (\SIrange{318}{376}{\GHz}).
  \textbf{right}
  Plot showing a model of zenith atmospheric transmission on the summit of Mauna Kea, assuming \SI{0.5}{\mm} of percipital water vapor.
  The plot is illustrative only, demonstrating the presence of atmospheric transmission windows and the general trend of worse transmission at higher frequencies.
  The data was obtained using the Caltech Submillimeter Observatory (CSO) Atmospheric Transmission Interactive Plotter \cite{darek_lis_cso_????}, which is based on a published model \cite{pardo_atmospheric_2001}.
}
\label{fig:ch1-clothes-atmos-trans}
\end{figure*}

But there are other applications and operational screnarios in which portal screening system are not feasiable.
One example is detection of suicide bomb belts, a scenario in which is it desirable to make a detection while the person being observed is some distance away.
These applications are generally referred to as ``standoff'' detection because the imaging system ``stands off'' some distance from the target being imaged.

In addition, also as shown in \figref{clothes-atmos-trans}, atmospheric transmission also tends to decrease with frequency in this range, although there are many atmosheric and other windows present so that the increase is not monotonic.
One example is detection of suicide bomb belts.


\chapter{System Specifications, Challenges and Solutions}\label{c:specs}

\section{Specifications} \label{sec:ch2-specifications}
% need to say something about reqd det time constants here

\section{Noise Requirements} \label{sec:ch2-noise}


\chapter{\textsc{TES} Bolometer Theory}\label{c:tes}

% xxx Nils - replace "plot" with "panel" when referring to a part of a figure

This chapter summarizes the \TES\ theory used in this dissertation.
I start by describing the \TES\ electrical and thermal circuits, defining relevant parameters, and stating the linearized \TES\ equations.
For reference, I then summarize the important consequences of these equations, including expressions for detector responsivity, detector response to step functions in applied power and bias current, and detector noise.
I do not derive most of these results, because excellent references are available \cite{irwin_application_1995,irwin_transition-edge_2005, mather_bolometer_1982}.

I discuss the derivation of two results in more detail.
First, I give an expression for the time-domain response to a step function in applied detector bias current.
Second, I describe a new approach for measuring the natural detector time constant $\tau$ by extrapolating several measurements of the effective detector time constant $\tau_{eff}$ high in the transition. 

\section{\textsc{TES} Electrical And Thermal Circuits} \label{sec:ch3-circuits}

\figref{fig:elec-thermal-circuit} shows the electrical and thermal circuits for a \TES\ bolometer.
The bolometer is voltage-biased by passing a bias current $I_{bias}$ through a shunt resistor \Rsh\ which has a much lower resistance than the normal-state resistance $R_n$ of the \TES.
The current through the \TES\ is inductively coupled into a \SQUID\ for readout.
The inductance $L$ in the diagram represents the sum of the input inductance of the \SQUID, a Nyquist inductor used to limit the noise bandwidth of the detector circuit, and any parasitic inductance present in the circuit.

The \TES\ itself is represented by a variable resistance $R$, which depends on both the current through the \TES\ and the temperature of the \TES.
The \TES\ is thermally sunk to a heat capacity $C$ which is weakly linked to a temperature bath $T_b$ through a thermal conductance $G$.
Optical power is absorbed by the heat capacity, causing the temperature $T$ of the heat capacity and the \TES\ to rise above $T_b$.
Power dissipated in any heater resistor\footnote{As described in \sectionref{sec:heater-r}, 31~detectors have heater resistors} present on the \TES\ also contributes to this temperature rise.

Because the resistance of the \TES\ depends on the temperature of the \TES, and the temperature of the \TES\ depends on the resistance of the \TES\ through Joule heating, the electrical and thermal behavior of the \TES\ are coupled.
This coupling acts as feedback, termed ``negative electrothermal feedback'', first described in the context of \TES\ detectors by Irwin\cite{irwin_application_1995}.
As the optical power absorbed by the \TES\ increases, the temperature of the \TES\ increases, which causes the resistance of the \TES\ to increase as well.
Because the \TES\ is voltage-biased, the Joule heating is inversely proportional to the resistance, so the Joule heating decreases, which causes the temperature of the \TES\ to decrease, opposing the effect of the increased optical power.
The negative electrothermal feedback speeds up the response time of the detector and allows the detector to self-bias into the superconducting transition.

\begin{figure*}
\centering
\includegraphics{drawings/ch3-elec-thermal-circuit.pdf}
\caption[Electrical and thermal \TES\ circuits]{
Electrical and thermal \TES\ circuits.
\textbf{Left} Schematic of real electrical \TES\ circuit.
The \TES\ is biased by a stiff current $I_{bias}$ shunted across a resistor $R_{sh}$ that is much smaller than the normal-state resistance of the \TES.
The \TES\ is represented by a variable resistance $R$, and $R_{par}$ represents any parasitic resistance in the circuit.
The current through the \TES\ is inductively coupled into a \SQUID\ for readout.
The inductance $L$ represents the sum of the input inductance of the \SQUID, a Nyquist inductor, and any parasitic inductance present in the circuit.
\textbf{Middle} Thevenin-equivalent \TES\ circuit used in derivation of the linearized electrical and thermal equations for the \TES.
\textbf{Right} Thermal \TES\ circuit.
The \TES\ is thermally sunk to a heat capacity $C$ which absorbs optical power.
The heat capacity $C$ is connected to a heat bath $T_b$ by a weak thermal link $G$, so that its temperature is elevated to a temperature $T$ above $T_b$ by applied optical power $P_{opt}$, power dissipated in a heater via $I_{htr}$ (if present), and Joule heating of the \TES\ itself.}
\label{fig:elec-thermal-circuit}
\end{figure*}

\section{Linearized Electrical and Thermal Circuits}\label{sec:lin-tes-eqn}

In the limit of small changes in \TES\ current and temperature, the resistance of the \TES\ can be expressed as
\begin{equation}
R(T_0+\delta T,I_0+\delta I) = R_0 + \alpha \frac{R_0}{T_0} \delta T + %
									 \beta_I \frac{R_0}{I_0} \delta I,
\end{equation}
where $R_0$, $I_0$ and $T_0$ are the resistance of the \TES, the current flowing through the \TES, and the temperature of the \TES\ at the operating bias point, $\alpha$ is the \TES\ temperature sensitivity, $\beta_I$ is the \TES\ current sensitivity, and $\tau \equiv C / G$ is the ``natural'' detector time constant..
Note that all terms used in these equations and the rest of this chapter are defined in \tableref{tab:tes-theory-summary}.

The power $P_b$ flowing through the thermal link $G$ is assumed to follow a power law of the form
\begin{eqnarray} \label{eqn:ch3-p-bath}
P_{b} = K(T^n - T^n_{b}),
\end{eqnarray}
which can also be written in the form
\begin{eqnarray}
P_{b} = \frac{GT}{n}\left(1 - \left(\frac{T_{b}}{T}\right)^n\right),
\end{eqnarray}
where
\begin{equation}
G \equiv \frac{d P_{b}}{dT} = K n T^{n-1}.
\end{equation}

With these definitions it can be shown \cite{irwin_transition-edge_2005} that the behavior of the \TES\ is described by a pair of coupled first-order differential equations:
\begin{equation}
\frac{d}{\mathop{dt}} \begin{pmatrix} \delta I \\ \delta T \end{pmatrix}
	= - \mathcal{M} \begin{pmatrix}	\delta I \\	\delta T \end{pmatrix}
      + \begin{pmatrix} \delta V / L \\ \delta P /C \end{pmatrix},
\end{equation}
where the matrix $\mathcal{M}$ is
\begin{equation}
% For spacing see http://tex.stackexchange.com/questions/14071/how-can-i-increase-the-line-spacing-in-a-matrix
\mathcal{M} = \begin{pmatrix}
		\dfrac{1}{\tau_{el}} & \dfrac{\Loop G}{I_0 L} \\[0.75em] 
		-\dfrac{I_0 R_0(2 + \beta_I)}{C} & \dfrac{1}{\tau_I}
    \end{pmatrix}.
\end{equation}
Here $\tau_{el} \equiv L / (R_0(1+\beta_I) + R_L)$ is the electrical time constant of the detector, $\Loop \equiv I_0^2 R_0 \alpha / G T_0$ is the detector loop gain, and $\tau_I \equiv \tau / (1 - \Loop)$.

These coupled equations can be solved under different initial conditions and applied forces $\delta V$ and $\delta P$.
Discussion of three cases follows.

\textbf{\TES\ Power-to-Current Responsivity}
Driving the \TES\ with a sinusoidal $\delta P$ term and holding detector bias constant leads to the following expression for the detector power-to-current responsivity:
\begin{equation}\label{eqn:si-full}
s_I(\omega) = 
- \dfrac{ \dfrac{1}{V_0} \dfrac{1}{\gamma} \dfrac{\Loop}{\Loop + 1} }
       { 1 + j \omega \left( \tau_{eff} - \dfrac{1}{\gamma}\dfrac{\Loop}{\Loop + 1} \dfrac{L}{R_0}\right) - \omega^2 \dfrac{L}{R_0}\dfrac{\tau_{eff}}{1 + \beta_I + R_L / R_0}},
\end{equation}
\begin{equation}
\gamma \equiv 1 + \dfrac{\beta_I}{1+\Loop} - \dfrac{\Loop - 1}{\Loop + 1}\dfrac{R_L}{R_0}.
\end{equation}
Here $\tau_{eff}$ is the ``effective'' detector time constant and is given by
\begin{equation} \label{eqn:teff}
  \tau_{eff} \equiv \dfrac{\tau}{1 + \dfrac{1 - R_L / R_0}{1 + \beta_I + R_L / R_0} \Loop}.
\end{equation}

While imposing, these expressions are much simpler in the limit which generally hold for operating \TES\ detectors: strong voltage bias ($R_L \ll R_0$), and $\tau \ll L/R$. 
In this limit the power-to-current responsivity becomes
\begin{equation}
s_I(\omega) = -\dfrac{1}{V_0} \dfrac{\Loop}{1+\beta_I + \Loop} \dfrac{1}{1 + j \omega \frac{\tau}{1 +\Loop/(1+\beta_I)}}
\end{equation}
The detector response time is given by the natural detector time constant $\tau$, sped up by a factor of $1 + \Loop(1+\beta_I)$; for $\beta_I \ll 1$, this factor is typical of negative feedback, and justifies calling \Loop\ the ``loop gain'' of the detector.
In the further limit of strong electrothermal feedback ($\Loop \gg 1$, $\Loop \gg \beta_I$), the \DC\ responsivity $s_I(0)$ is simply the inverse of the voltage bias.
This means that because of the strong electrothermal coupling, any increase in applied optical (or heater) power is exactly canceled by a decrease in detector Joule heating, so that the \TES\ temperature remains unchanged.

\textbf{\TES\ Response to Step Function in Power}
As demonstrated in \sectionref{sec:bias-step}, our detectors are always operated in a regime where $\tau_{eff} \gg \tau_{el}$.
Under these conditions, \eqnref{eqn:si-full} simplifies to
\begin{equation} \label{eqn:htr-step-resp-high}
s_I(\omega) = - \dfrac{1}{V_0 \gamma}\dfrac{\Loop}{\Loop + 1}
                       \left(1 + j \omega \tau_{eff}\right)^{-1}.
\end{equation}
This implies that the time-domain response to step in applied power, for example from a heater, is
\begin{align} \label{eqn:htr-step-resp-high-time}
    \delta I(t) & = - \delta P s_I(0) (1 - e^{-t/\tau_{eff}}) \\
                & = - \dfrac{\delta P}{V_0 \gamma}\dfrac{\Loop}{\Loop + 1}
                      (1 - e^{-t/\tau_{eff}}).
\end{align}
This can be used to measure $\tau_{eff}$ directly as well as the \DC\ responsivity once the heater power has been calibrated (\sectionref{sec:teff-resp}).
As described in \sectionref{sec:tau-nat-theory}, it can also be used to measure the detector natural time constant $\tau$.
These measurements are described further in \sectionref{sec:tau-nat}.

\textbf{\TES\ Response to Step Function in Bias Current}
To derive the behavior of the \TES\ after a step function in applied bias, we solve the equations under the conditions
\begin{equation}
\begin{pmatrix} \delta I(0) \\ \delta T(0) \end{pmatrix} = \begin{pmatrix} 0 \\ 0 \end{pmatrix}
\end{equation}
with constant driving force starting at time zero of
\begin{equation}
\begin{pmatrix} \delta I_{bias} R_{sh} / L \\ 0 \end{pmatrix}
\end{equation}
Solving this system leads to the following expression for the \TES\ current as a function of time:\footnote{A Mathematica notebook which verifies this solution, as well as other solutions to the linearized \TES\ equations, can be found at https://gist.github.com/danbek/8591076}
\begin{equation}\label{eqn:bias-step-resp}
\delta I (t)
   = - \dfrac{\delta I_{bias} R_{sh}}{R_0} 
       \dfrac{(\Loop - 1)
             \left(1 - \dfrac{\tau_{eff} - \tau_I}{\tau_{eff} - \tau_{el}} e^{-t/\tau_{eff}}
                 	       + \dfrac{\tau_{el} - \tau_I}{\tau_{eff} - \tau_{el}} e^{-t/\tau_{el}} \right)}
            {1 + \beta_I + R_L/R_0 + \Loop(1 - R_L/R_0)}
       .
\end{equation}
This expression is complex, but the behavior can be understood as follows.
Immediately after an increase in bias current the voltage across the \TES\ begins to increase, with a time constant of $\tau_{el}$.
As the voltage increases, the Joule power in the \TES\ increases, which warms the \TES.
This warming increases the resistance of the \TES.
Because the \TES\ is voltage-biased, this reduces Joule power in the \TES, which tends to cool the detector as well as reduce current through the detector.
This negative electrothermal feedback effect occurs with a time constant of $\tau_{eff}$.
Whether the final current through the \TES\ is higher or lower than the original current depends on the loop gain.
For $\Loop < 1$ the current increases, for $\Loop > 1$ it decreases and for $\Loop = 1$ the current through the \TES\ remains unchanged.

\eqnref{eqn:bias-step-resp} depends on \Loop\ and $\beta_I$ in a complex way through $\tau_{eff}$, $\tau_I$, $\tau_{el}$, and the prefactor.
Nevertheless, if the response of a \TES\ to a bias step can be measured with sufficient bandwidth to track the initial fast electrical response, bias steps can be used to measure \Loop\ and $\beta_I$ by performing non-linear parameter fitting to \eqnref{eqn:bias-step-resp}.
Measurements of \Loop\ and $\beta_I$ using this technique are described in \sectionref{sec:bias-step}.

When the \TES\ is superconducting, \eqnref{eqn:bias-step-resp} takes on a much simpler form.
Setting $R_0 = \Loop = \beta_I = 0$, the result is
\begin{equation}\label{eqn:bias-step-resp-sc}
\delta I(t)
   = - \dfrac{\delta I_{bias} R_{sh}}{R_{L}} 
       \left(1 - e^{-t/(L/R_L)} \right).
\end{equation}
Similarly, when the detector is fully normal, so that $\Loop = \beta_I = 0$, \eqnref{eqn:bias-step-resp} becomes
\begin{equation}\label{eqn:bias-step-resp-normal}
\delta I(t)
   = - \dfrac{\delta I_{bias} R_{sh}}{R_n + R_{L}} 
       \left(1 - e^{-t/(L/(R_n+R_L))} \right).
\end{equation}
The \TES\ response to bias steps in the superconducting and normal states can thus be used as measurements of $L$ and $R_n$.

\begin{table*}[t]
\centering
\caption{Symbols and parameters used in describing behavior of \TES\ detectors.}
\label{tab:tes-theory-summary}
{\renewcommand{\arraystretch}{1.5}%
\begin{tabular}{l l}
\toprule
Symbol &  Explanation \\
\midrule
\addlinespace
$I_{bias}$ & Current applied across shunt to bias \TES. \\
$R$ & \TES\ resistance (depends on temperature and current) \\
$R_n$ & \TES\ normal-state resistance \\
$R_{sh}$ & Shunt resistance \\
$R_{par}$ & Represents any parasitic resistance in \TES\ circuit \\
$R_L \equiv R_{sh} + R_{par}$ & Load resistance used in analysis of \TES\ circuit \\
$T$ & \TES\ temperature \\
$T_b$ & Thermal bath temperature \\
$I_0$, $R_0$, $V_0$, $T_0$ & \TES\ current, resistance, voltage, and temperature at bias point \\
$P_{bath} = K(T^n - T_{b}^n)$ & Total heat flow from \TES\ island to heat bath \\
$n$ & Power-flow index \\
$P_{opt}$ & Optical power falling onto \TES\ heat capacity \\
$P_{htr}$ & Power applied to \TES\ by heater resistor \\
$P_{J}$ & Joule power dissipated by \TES\ \\
$C$ & Heat capacity of \TES\ island \\
$G \equiv \frac{d P_{bath}}{d T} = K n T^{n-1}$ & Weak-link differential thermal conductance \\
$\tau \equiv \frac{C}{G}$ & \TES\ natural time constant \\
$\tau_{el} \equiv \dfrac{L}{R_L + R_0(1 + \beta_I)}$ & \TES\ electrical time constant \\
$\tau_I \equiv \dfrac{\tau}{1-\Loop}$ & \TES\ constant-current time constant \\
$\tau_{eff} \equiv \dfrac{\tau}{1 + \frac{1 - R_L / R_0}{1 + \beta_I + R_L / R_0}\Loop}$ & \TES\ effective time constant \\
$\alpha \equiv \frac{T_0}{R_0} \frac{\partial R}{\partial T}$ & Logarithmic \TES\ temperature sensitivity \\
$\beta_I \equiv \frac{I_0}{R_0} \frac{\partial R}{\partial I}$ & Logarithmic \TES\ current sensitivity \\
$\Loop \equiv \frac{I_0^2 R_0 \alpha}{G T_0}$ & Loop gain \\
$\delta V = \delta I_{bias} R_{sh}$ & Change in bias voltage applied to \TES\ \\
$\delta P$ & Change in power (optical or heater) falling on \TES\ \\
\bottomrule
\end{tabular}
}
\end{table*}

\section{Measurement of Natural Time Constant}\label{sec:tau-nat-theory}

Near the top of the superconducting transition, $\Loop < 1$, so that $\tau_{eff} > 0.5 \tau$.
The \TES\ detectors used for the \Imager\ have been designed so that $\tau \gg L/R_n$ (see \sectionref{sec:det-parm-choice}), so that the response to a step in applied heater power is given by \eqnref{eqn:htr-step-resp-high-time}.
As the fully normal state is approached, $\tau_{eff}$ approaches $\tau$, so that measuring the $\tau_{eff}$ very high in the transition will give a measurement of $\tau$.
However, the power-to-current responsivity decreases high in the transition, reducing the signal-to-noise of the measurement.

To avoid this problem, an expression can be obtained linking $\tau$ and $\tau_{eff}$ that holds independent of location in the transition, as long as the assumption $\tau_{eff} \gg L/R_0$ holds.
The \DC\ response to a step in applied power $\delta P$ is given by
\begin{equation}
\delta I = \frac{\delta P}{I_0 R_0}\frac{\Loop}{1 + \beta_I + R_L/R_0 + (1 - R_L / R_0)\Loop}.
\end{equation}
This equation can be solved for \Loop, and then substituted into the expression for $\tau_{eff}$.
This leads to
\begin{equation}\label{eqn:teff-from-tau}
\tau_{eff} = \tau - \tau \mathcal{K} I_{bias} \delta I,
\end{equation}
\begin{equation}
\mathcal{K} \equiv \frac{R_{sh}}{\delta P} \frac{R_0 - R_L}{R_0 + R_L}.
\end{equation}
Here the relation
\begin{equation}
I = I_{bias}\frac{R_{sh}}{R + R_L}
\end{equation}
has also been used.

\eqnref{eqn:teff-from-tau} holds independent of \Loop\ and $\beta_I$.
The factor $\mathcal{K}$ depends on the bias point, but high in the transition this dependence is weak, so that $\mathcal{K}$ can be treated as a constant.

To use \eqnref{eqn:teff-from-tau} to measure $\tau$, steps in heater power are applied to the \TES\ at a set of bias points close to the normal state.
At each bias point the \DC\ change in \TES\ current $\delta I$ and $\tau_{eff}$ are measured by fitting the \TES\ response to \eqnref{eqn:htr-step-resp-high-time}, and the bias current $I_{bias}$ is recorded.
A non-linear curve fit can then be applied to \eqnref{eqn:teff-from-tau} to solve for $\tau$ and $\mathcal{K}$.
Alternately, $\mathcal{K}$ can be calculated if all factors feeding into it are known, and then \eqnref{eqn:teff-from-tau} can be solved directly for $\tau$.

\sectionref{sec:tau-nat} presents measurements of $\tau$ for four detectors using this technique.

\section{\textsc{IV} Curve Analysis} \label{sec:ch3-iv-curve}

\TES\ detector current-vs-voltage (\IV) curves contain important information about the behavior of \TES\ detectors.
They directly yield the resistance of the \TES\ in both the normal state and throughout the superconducting transition.
But they also allow other properties of the \TES\ to be measured by comparing \IV\ curves taken under different operating conditions, such as different bath temperatures and applied heater and/or optical power loads.

The total amount of power flowing through the \TES\ thermal conductance $G$ is given by
\begin{equation}\label{eqn:ch3-tes-ptot}
P_{tot} = K(T^n - T_b^n) = P_{opt} + P_{htr} + I^2 R(T,I).
\end{equation}
We can make the assumption that at the start of the superconducting transition, where $R \approx R_n$, $\beta_I = 0$, i.e.\ the resistance of the \TES\ depends only on the \TES\ temperature, and not on the current through the \TES.
This assumption has been observed to hold empirically for many different types of \TES\ detectors, and there are also theoretical reasons to expect it to be true \cite{bennett_resistance_2013}.
Under this assumption, near the top of the transition the total power $P_{tot}$ is current-independent, so the following relationship must hold:
\begin{equation}\label{eqn:ch3-ptot-99Rn}
P_{J} \equiv I^2 R = P_{tot} - P_{opt} - P_{htr} = P_{tot} - P_{opt} - I_{htr}^2 R_{htr}.
\end{equation}
In practice, I treat $R = 0.99 R_n$ as sufficiently high in the transition for this relationship to hold.

\eqnref{eqn:ch3-ptot-99Rn} is used in two different ways in this dissertation to extract information about the \TES\ detectors, as described in the following subsections.

\subsection{Calibration of Heater Resistors}

If a set of \IV\ curves are taken at the same bath temperature but different heater biases, \eqnref{eqn:ch3-ptot-99Rn} allows measurement of the resistance of the \TES\ heater by fitting for $R_{htr}$ and $(P_{tot} - P_{opt})$.
\figref{fig:ch6-heater-r-plots} (reproduced in this chapter for convenience as \figref{fig:ch3-heater-r-plots}) shows how this is done using a series of \IV\ curves, all of which were taken for a particular detector labeled \RCm{29}{1}.
The upper left plot shows a set of \TES\ \IV\ curves taken at $T_b = 1100$~mK, with only the applied heater bias varying.
The upper right plot shows the same data, but in terms of \TES\ Joule power and \TES\ resistance.
As applied heater current decreases, the Joule power at the top of the transition decreases.
In the lower left, the Joule power at $0.99R_{n}$ is plotted vs applied heater current.
A fit to \eqnref{eqn:ch3-ptot-99Rn} is also plotted.
Finally, the lower right plot shows $R$ vs $P_J$ after the heater power has been added to each curve.
This plot shows that the powers are equal very high in the transition, where the assumption that Joule power only depends on \TES\ resistance holds.
It also shows that this assumption breaks down deeper in the transition.

% xxx this would probably be better if I explained that we set heater
% current through a DAC value and convert to current via measure bias resistors.
Note that in order to determine the value of $R_{htr}$ one much know the heater current.
Any error in the assumed heater current will lead to a corresponding error in the derived $R_{htr}$ value.
But because $R_{htr}$ is derived from the power dissipated in the resistor, the product $I_{htr}^2 R_{htr}$ will remain unchanged.
This means that whenever the value $R_{htr}$ is used to calculated a power, the power value will be correct even with an incorrect value for the heater current.

\sectionref{sec:heater-r} uses this approach to calculate $R_{htr}$ for the seven working heaters on columns 0 and 1.

\begin{figure*}
\includegraphics{drawings/ch6-heater-r-plots.pdf}
\caption[Plots describing heater measurements]{
Plots describing heater measurements, for the case of the detector labeled \RCm{29}{1}.
\textbf{Upper Left} \IV\ curves. The \IV\ curves should become vertical when the detector becomes fully superconducting at zero voltage, but these curves shows a non-infinite slope. The reason for this is that the readout system as configured for these \IV\ curves was unable keep up with the rapid change of current in the superconducting branch.
\textbf{Upper Right} Same data as in upper left plot, but represented in terms of \TES\ Joule power and resistance. As the bias current for the heaters is increased, the curves shift to the left.
\textbf{Lower Left} Measured $P_{J}$ vs heater current at $0.99R_n$, as well as fit to \eqnref{eqn:ch3-ptot-99Rn}.
\textbf{Lower Right} Same plot as upper right, but the heater power based on $R_{htr} = \SI{23.6}{\ohm}$ has been added to each curve.
This demonstrates that $\beta_I = 0$ does not hold below the very top of the transition.}
\label{fig:ch3-heater-r-plots}
\end{figure*}

\subsection{Measurement of \textsc{TES} Differential Thermal Conductance $G$}

With knowledge of the heater resistances, \IV\ curves can be taken over a wide range of bath temperatures, which enables a measurement of the \TES\ thermal conductance $G$ and transition temperature $T_c$.
In this case $P_{tot}$ will be different for each \IV\ curve, so that \eqnref{eqn:ch3-tes-ptot} can be used in the form
\begin{equation}\label{eqn:ch3-g-fit}
P_{htr} + P_J + P_{opt}= \frac{G T_c}{n}\left(1 - \left(\frac{T_b}{T_c}\right)^n\right).
\end{equation}
A non-linear curve fit can then be used to find $G$, $T_c$, and $n$.
The upper plots in \figref{fig:heater-g-plots} (reproduced in this chapter for convenience as \figref{fig:ch3-g-plots}) show an example of this fit for the detector labeled \RCm{31}{2}.
The fit procedure leads to correlation between the fit values of $G$ and $n$ which indicated degeneracy in the fit between $G$ and $n$.

\sectionref{sec:g-psat} uses this approach to calculate $G$, $T_c$ and $n$ for the seven detectors with working heaters on columns 0 and 1.

\begin{figure*}
\includegraphics{drawings/ch3-g-plots.pdf}
\caption[Plots showing fit to \eqnref{eqn:ch3-g-fit}]{
Plots showing fit to \eqnref{eqn:ch3-g-fit} for the detector labeled \RCm{31}{2}.
\textbf{Left} Plot showing $P_{sat}$ vs $T_b$ assuming $P_{opt} = 150$~pW (see \sectionref{sec:g-psat}).
The red line shows the best fit to \eqnref{eqn:ch3-g-fit}.
The data cover 36 data points including 25 temperatures from \SIrange{995}{1160}{\mK} and 11 different heater biases.
\textbf{Right} Scatter plot showing covariance between the fitted values of $G$ and $n$, in terms of 95 \% confidence ellipses.} 
\label{fig:ch3-g-plots}
\end{figure*}

\section{\textsc{TES} Saturation Power} \label{sec:ch3-psat}

Consider \eqnref{eqn:ch3-tes-ptot}:
\begin{equation}\label{eqn:ch3-tes-ptot-2}
P_{tot} = K(T^n - T_b^n) = P_{opt} + P_{htr} + I^2 R(T,I).
\end{equation}
The value of $P_{tot}$ when $T = T_c$, is called the ``saturation power'' ($P_{sat}$) of the detector:
\begin{align}
  P_{sat} & \equiv K(T_c^n - T_b^n) \\
         & = \frac{G T_c}{n} \left( 1 - \left(\frac{T_b}{T_c}\right)^n \right).
 \label{eqn:ch3-psat}
\end{align}
If the power flowing across $G$ is larger than this value the detector temperature is forced to be higher than $T_c$ so that the detector goes normal and no longer works.
This is an important parameter of a \TES, and $G$ must be chosen so that $P_{opt} < P_{sat}$. 
The ratio $P_{sat} / P_{opt}$ is called the ``safety factor''.
% xxx nils wants to give a sample safety factor here, something about
% chosen based on stability?

\section{Stability of \textsc{TES} Bolometers} \label{sec:ch3-tes-stability}

In any physical system in which negative feedback is applied, the system response can become unstable if the feedback is applied with a phase change that approaches \ang{180}.
This situation can occur in \TES\ detectors if the inductance in the \TES\ bias circuit is so large that the \TES\ electrical time constant $\tau_{el} = L/(R_0(1+\beta_I) + R_L)$ becomes too close to the \TES\ effective time constant $\tau_{eff}$.
For a voltage-biased \TES\ with $R_L \ll R_0$, the criteria for stable operation is \cite{irwin_transition-edge_2005}
\begin{equation}
  \label{eqn:ch3-L-stable}
  L > \left[ \Loop(3 + \beta_I) + (1 + \beta_I) - 2\sqrt{\Loop(2+\beta_I)(1 + \beta + \Loop)} \right] \frac{R_0 \tau}{(\Loop-1)^2}
\end{equation}
% xxx mention that this is used in ch 6?

\section{\textsc{TES} Bolometer Noise} \label{sec:ch3-tes-noise}

There are three sources of detector noise in \TES\ bolometers: Johnson noise in the \TES\ resistance, Johnson noise in the load resistor $R_L$, and thermal fluctuation noise across the weak thermal link $G$.
Additionally, intrinsic fluctuations in the number of arriving photons leads to photon noise, which can be a significant source of noise for low-temperature \TES\ bolometers.
Expressions for these sources of noise are shown in \tableref{tab:tes-noise}.

The function $F$ that enters into the thermal fluctuation noise accounts for the temperature gradient between the \TES\ and the bath.
The form of $F$ depends on whether the mean free path of phonons crossing the thermal link is long or short compared with the length of the link.
In the case of a short mean free path, $F$ depends on $n$ and is given by \cite{mather_bolometer_1982}
\begin{equation}
  F(T_0,T_b) = \frac{n}{2n+1} \frac{1 - (T_b/T_0)^{2n+1}}{1 - (T_b/T_0)^{n}}.
\end{equation}
In the case of a long mean free path, $F$ is independent of $n$ and is given by \cite{boyle_performance_1959}
\begin{equation}
  F(T_0, T_b) = \frac{1}{2} (1 + (T_b/T_0)^5)
\end{equation}
For the detectors described in this dissertation, $n \approx 3.5$, $T_0 \approx 1.2$~K, and $T_b \approx 1.1$~K.
Under these conditions, both expressions for $F$ have approximately the same value, 0.83.

For typical operating conditions of \TES\ bolometers, thermal fluctuation noise dominates Johnson noise at low frequencies.
This can be see by taking the ratio of $S^2_{TES}$ to $S^2_{TFN}$.
After some simplifying algebra the result is (ignoring factors of order unity):
\begin{equation}
\frac{S^2_{TES}}{S^2_{TFN}} \approx \frac{1}{\alpha \Loop} (1 + (\omega \tau)^2)
\end{equation}
At low frequencies the \TES\ resistor current noise is suppressed below thermal fluctuation noise by a factor of $1/\alpha \Loop$.
\TES\ detectors are always biased so that $\Loop > 1$, and values for $\alpha$ in the transition for our detectors are 20--400 (see \figref{fig:ch6-bias-step-results} in \sectionref{sec:bias-step}).
Examination of \tableref{tab:tes-noise} shows that current noise from the load resistor is lower than that from the \TES\ resistor by a factor of $(\Loop-1)^2 (R_0 / R_L) (T_0/T_L)$.
Therefore, intrinsic detector noise in our \TES\ bolometers is dominated by thermal fluctuation noise.

Another source of noise in any bolometer is Photon noise, arising due to quantum fluctuations in the number of photons arriving during a given time interval.
This noise is expressed as \cite{zmuidzinas_thermal_2003}
\begin{equation}\label{eqn:photon-noise}
  S^2_{ph} = 2 h \nu P_{opt} (1 + \eta \bar{n}),
\end{equation}
where $\bar{n}$ is the photon occupation number, given by
\begin{equation}
  \bar{n} = \frac{1}{e^{\frac{h \nu}{k_B T}} + 1}
\end{equation}

\sectionref{sec:ch5-predicted-noise} discusses predicted noise levels for our detectors.
\sectionref{sec:det-noise} discusses measurements of detector noise.

\begin{table*}
\centering
\caption[Noise in \TES\ bolometers]{
  Noise in \TES\ bolometers, referred to power absorbed in bolometer.
  To obtain current noise passing through the bolometer, multiply each power spectral density by $|s_I(\omega)|^2$.}
\label{tab:tes-noise}
{\renewcommand{\arraystretch}{2.0}%
\begin{tabular}{l l}
\toprule
Noise Source &  Noise Power Spectral Density \\
\midrule
\TES\ Resistor & $S^2_{TES} = 4 k_B T_0 I_0^2 R_0 \xi(I_0) \frac{(1 + (\omega \tau)^2) } {\Loop^2}$ \\
Load Resistor & $S^2_{L} = 4 k_B T_L I_0^2 R_L \frac{(1 + (\omega \tau_I)^2) (\Loop -1)^2} {\Loop^2}$ \\
Thermal Fluctuation Noise & $S^2_{TFN} = 4 k_B T_0^2 G F(T_0, T_b)$ \\
\bottomrule
\end{tabular}
}
\end{table*}


\chapter{System Design Overview}\label{c:sys-design}

\section{Cryostat Design}\label{s-cryo-design}

The cryostat for the \Imager\ was designed with the goals of simplicity, reliability and turn-key automated operation.
Reliable and easy-to-use cryogen-free mechanical cryocoolers are commercially available, but these cryocoolers are seldom capable of reaching temperatures below \SI{2.5}{\K}.
Reaching sub-Kelvin temperatures requires a second refrigeration stage, which in our case is a He-4 adsorption refrigerator.
The He-4 adsorption refrigerator is based on a proven design and its use can easily be automated.
Three temperature stages within the cryostat are provided in order to provide intercepts for heat sinking wiring and other objects that are thermal connected to room temperature. 
The result is a reliable cryogen-free system that can be controlled remotely.

The cryostat itself was built by Precision Cryogenics\footnote{Precision Cryogenics Systems, Inc. Indianapolis, IN. \url{http://www.precisioncryo.com}} to designs provided by the \Imager\ team.
\figref{fig:cryo-cutaway} shows a cutaway view of the cryostat, and \tableref{tab:temp-optical-load} lists the temperatures typically reached by different parts of the cryostat during operation when the cryostat is open optically.
The first two temperature intercept stages are provided by a Cryomech PT407 Pulse Tube Cryorefrigerator\footnote{Cryomech, Inc. Syracuse, NY. \url{http://www.cryomech.com}}
The PT407 has two cooling stages.
The first stage has \SI{25}{\W} of cooling power at \SI{55}{\K} while the second stage has \SI{0.7}{\W} at \SI{4.2}{\K}.
Our PT407 uses a remote motor, so that the cold head attached to the cryostat has no moving parts, minimizing vibration of the cryostat.
Vibration of the cryostat can lead to microphonic pickup either directly in the detectors themselves or in the readout circuitry, leading to much higher detector noise.

The cryostat has two main parts: a cylinder containing both the \PTC\ and the \He4-sorption refrigerator, and a box located at the bottom of the cylinder which contains temperature intercept plates and the focal plane.
There are three temperature stages, the ``\SI{90}{\K}'' Cold Plate, the ``\SI{4}{\K}'' Cold Plate, and the Focal Plane.
The \PTC\ 1st stage is connected to the ``\SI{90}{\K}'' Cold Plate by a tube of Al 1100 and a set of CDA-101 Cu braids.
The combination of this long thermal path with the high heat load on the optical filters sunk to the ``\SI{90}{\K}'' stage explains the \SI{45}{\K} temperature differential between the ``\SI{90}{\K}`` cold plate and the \PTC\ 1st stage.
The \PTC\ 2nd stage is connected to the ``4 K'' Cold Plate by a large (3.0 in diameter by 2.78 in long) cylinder of CDA-110 Cu\footnote{This cryostat was originally designed to work with a different cryocooler. The \PTC\ currently installed had a shorter distance between the 1st and 2nd stages, necessitating the Cu cylinder to take up this extra space}, followed by tube of alloy CDA-101 Cu followed by a set of  CDA-101 copper braids.
The Cu tube is broken into two halves, and the condensation plate (see below) of the adsorption fridge is clamped between these two halves. The ``90 K'' Cold Plate is stood off from the cryostat vacuum jacket by four ``roll wrapped'' carbon fiber tube standoffs. The ``\SI{4}{\K}'' Cold Plate stands off from the ``\SI{90}{\K}'' Cold Plate by eight supports made of G-10.

\begin{figure*}
\centering
\begin{tikzpicture}
    \node[anchor=south west,inner sep=0] (image) at (0,0) {\includegraphics[width=2.6in]{./images/cryostat-cutaway.png}};
    \begin{scope}[x={(image.south east)},y={(image.north west)}]
	    %\draw[help lines,xstep=.1,ystep=.1] (0,0) grid (1.5,1);
		%\foreach \x in {0,1,...,15} { \node [anchor=north] at (\x/10,0) {0.\x}; }
		%\foreach \y in {0,1,...,9} { \node [anchor=east] at (0,\y/10) {0.\y}; }
		
        \draw[red,ultra thick,rounded corners] (0.50,0.7) rectangle (0.85,0.75) node[below left] {\textbf{A}}; % \PTC407 1st stage CH
        \draw[red,ultra thick,rounded corners] (0.56,0.595) rectangle (0.71,0.64) node[below left] {\textbf{B}}; % \PTC407 2st stage CH
        \draw[red,ultra thick,rounded corners] (0.53,0.55) rectangle (0.78,0.595) node[below left] {\textbf{C}}; % Cu Cylinder
        \draw[red,ultra thick,rounded corners] (0.505,0.32) rectangle (0.75,0.55) node at +(-0.1,-0.03) {\textbf{D}}; % sorp fridge
        \draw[red,ultra thick,rounded corners] (0.4,0.06) rectangle (0.70,0.26) node[red,below left] {\textbf{E}}; % Focal Plane
        
        \draw[thick,<->] (1.05,0.04) -- +(0,0.31) node[midway,right] {18 in}; % scale bar
    \end{scope}
\end{tikzpicture}
\caption{Cutaway view of the \Imager. \textbf{A} PT407 1st stage cold head \textbf{B} PT407 2nd stage cold head \textbf{c} Cu cylinder connected PT407 2\textsuperscript{nd} stage cold head to Cu tube, which then connects to \He4-sorption refrigerator condensation plate. \textbf{D} \He4-sorption refrigerator \textbf{E} Focal Plane. Copper ropes connecting focal plane to \SI{1}{\K} cold plate are not visible in this view.}
\label{fig:cryo-cutaway}
\end{figure*}

\begin{table*}
\centering
\caption{Temperatures Reached Under Optical Load} % xxx should I add temps when closed optically?
\label{tab:temp-optical-load}
\begin{tabular}{l r}
\toprule
Temperature Stage &  Temperature (K)\\
\midrule
\PTC\ 1st Stage Cold Head 			& 48 \\
\PTC\ 2st Stage Cold Head 			& 3.5 \\
Cryostat ``50 K'' Cold Plate 		& 84 \\
Cryostat ``4 K'' Cold Plate 			& 5.8 \\
Adsorption Fridge Condensation Plate 	& 3.7 \\
Focal Plane 						& 0.962 \\
\bottomrule
\end{tabular}
\end{table*}

Options for reaching temperatures below the \abt{\SI{1.2}{\K}} transition temperature of our \TES\ detectors include: dilution refrigerators, adiabatic magnetization refrigerators, pumped \He4 baths, and \He3- and/or \He4-sorption refrigerators.
We chose a \He4-sorption fridge because the typical base temperature under no load of \abt{\SI{700}{\mK}} is well-matched to our application.
\He-sorption fridges are also relatively inexpensive and easy to operate compared to other solutions.
A \He4-sorption fridge works by using a charcoal adsorber to pump on a bath of liquid \He4, reducing the \He4 boiling point and thus the temperature of the bath.
The \He4 is contained withing a sealed reservoir so that the refrigerator acts as a closed system requiring no \He4 replenishment. 
While \He4-sorption fridges are commercially available, our team choose to design and build a custom fridge based on a design that has been proven in astronomical applications \cite{devlin_high_2004}.

\figref{fig:he4sorp} shows a schematic depiction of the \Imager's \He4-sorption refrigerator.
The entire refrigerator is filled with \SI{2.07}{\mole} of \He4 gas, giving a pressure of \SI{900}{psi} at room temperature.
In normal operation the heat switch between the charcoal pumping chamber (``pump'') and the \He4 condensation plate is closed, keeping the charcoal as cold as possible in order to adsorb as much \He4 as possible, keeping the temperature of the cold plate as low as possible.

\begin{figure*}
\centering
\begin{tikzpicture}
    \node[anchor=south west,inner sep=0] (image) at (0,0) {\includegraphics[width=3.0in]{./images/he4-sorp-fridge-cutaway.png}};
    \begin{scope}[x={(image.south east)},y={(image.north west)}]
	    %\draw[help lines,xstep=.1,ystep=.1] (0,0) grid (1.5,1);
		%\foreach \x in {0,1,...,15} { \node [anchor=north] at (\x/10,0) {0.\x}; }
		%\foreach \y in {0,1,...,9} { \node [anchor=east] at (0,\y/10) {0.\y}; }
        \draw[blue,ultra thick,rounded corners] (0.10,0.56) rectangle (0.86,1.01) node[below left] {\textbf{A}}; % pump
        \draw[blue,ultra thick,rounded corners] (0.02,0.56) rectangle (1.03,0.44) node[above left] {\textbf{B}}; % cond plate
        \draw[blue,ultra thick,rounded corners] (0.70,0.56) rectangle (0.88,0.36) node[above left] {\textbf{C}}; % heat switch
        \draw[blue,ultra thick,rounded corners] (0.35,0.25) rectangle (0.72,-0.02) node[above left] {\textbf{D}}; % pot
        
		\draw[thick,<->] (0.9,0.02) -- ++(0,0.21875) node[midway,right] {3 in}; % scale bar
    \end{scope}
\end{tikzpicture}
\caption{Cutaway view of the \He4-sorption refrigerator. \textbf{A} Charcoal pumping chamber (``pump''). The charcoal is attached to the concentric copper cylinders. Cylinders are used to maximize the surface area covered by the charcoal. \textbf{B} Condensation plate. This copper plate is kept below the boiling point of \He4 in order to provide a point in the refrigerator for \He4 to condense and drip into the condensation pot. \textbf{C} \He4 gas gap heat switch. This heat switch is used to cool the charcoal in order to pump on the \He4 bath in the condensation pot. \textbf{B} \He4 condensation pot (``pot''). The condensed \He4 accumulates here. The concentric cylinders provide additional surface area for thermal contact to the liquid \He4. Not shown is a sapphire \SI{1}{\mm} constriction in the stainless steel tube connecting the pump and pot, intended to restrict the flow of super-fluid \He4 away from the pot (Swiss Jewel Company, part A34.00).}
\label{fig:he4sorp}
\end{figure*}

Cycling the refrigerator requires four steps.
First the heat switch is opened.
The \He4-sorption refrigerator uses a \He4 gas-gap heat switch manufactured by Chase Cryogenics\footnote{Chase Research Cryogenics, Ltd. Sheffield, UK}, which requires 5 minutes of waiting time in order for the switch to fully open.
Second, the pump is heated by applying \SI{5}{\W} via a \SI{500}{\ohm} power resistor.
This power is applied until the temperature of the pump reaches 40 K, which is high enough to drive nearly all of the adsorbed \He4 off of the charcoal.
Third, the power to the pump is turned off.
Once the temperature of the condensation plate falls below the boiling point of \He4, \He4 will begin to condense on its walls, dripping into the \He4 condensation pot (``pot'').
Fourth, once the temperature of the pot has fallen to \SI{4.0}{\K}, the heat switch is turned back on.
This cools the pump, allowing \He4 to again adsorb onto the charcoal, which has the effect of pumping strongly on the pot, and cooling the \He4 contained there to the base temperature of \SI{970}{\mK} under optical load.
This process is easy to automate, and a LabVIEW program cycles the fridge automatically every night while the system is operating.

Under optical load, a full cycle of the \He4-sorption refrigerator takes approximately 4 hours, reaching a base temperature of ~\SI{970}{\mK}.
When no additional load is applied the hold time is 9 hours.
When the temperature of the stage is held at the typical operating temperature of \SI{1100}{\mK} the hold time is only 3:45 hours.

\tableref{tab:fp-thermal-load} shows the predicted heat load on the \He4-sorption refrigerator.
It excludes parasitic load inherent to the sorption refrigerator itself; we have estimated this parasitic load as \abt{\SI{0.5}{\mW}}.
% See Black n'Red notebook, Feb 3 2010.

% xxx - include data on temp vs load for the fridge?

\begin{table*}[ht]
\centering
\caption{
  Predicted Thermal Load on \SI{1}{\K} Stage.
  These calculations assume that the readout wiring and Ti-6Al-4V spiders are running from \SI{6.4}{\K} to \SI{1.0}{\K}.
  All wires are AWG36 Phosphur Bronze with thermal conductivity taken from the Lakeshore Cryogenics reference tables \cite{lake_shore_cryogenics_inc._cryogenic_????}.
  ``\BOSE\ readout wires'' refers to the wires running from room temperature used to readout the position of the secondary mirror; see \sectionref{sec:ch8-mirror-readout}.
  The heat load from these wires is very high because they are not currently intercepted anywhere between room temperature and the \SI{1}{\K} stage.
  Intercepting them at the ``\SI{4}{\K}'' Cold Plate would reduce their load to \SI{1.1}{\uW}.
  xxx source for low-temperature thermal conductivity of Ti-6Al-4V.
  ``Other optical power'' refers to in-band power that reaches the \SI{1}{\K} stage but is not absorbed by the detectors.
  The number quoted assumes that all of this power is absorbed on the \SI{1}{\K} stage; this results in an overestimate because it is likely that most of this power is reflected by the feedhorn array, rathern than being absorbed.
  Out-of-band optical power is assumed to be zero because the only warm object shining directly on the 1K stage --- the W1275 bandpass filter --- is only \SI{16}{\K}, and is not emmissive at the wavelengths at which it would be radiating (see \sectionref{sec:ch4-filters} and \cite{tucker_thermal_2006}).
}
\label{tab:fp-thermal-load}
\begin{tabular}{@{}lrr@{}}
\toprule
 & \multicolumn{2}{c}{Predicted Thermal Load} \\
\cmidrule(r){2-3}
  Heat Load Source & 1004 Detectors (\si{\uW}) &  251 Detectors (\si{\uW}) \\
\midrule
  Readout wiring                   & 52 & 130 \\
  Series array \SQUID\ modules     & 0.2 & 0.6 \\ 
  % These are SA10b "two turn" according to an email from Colin. Per
  % this page: http://doc.bqemd.nist.gov/qsp/sa10b they should be
  % about 20 nW of power per chip
  \SQUID\ multiplexing chips       & 0.4 & 1.6 \\
  % Use 100 uA Ic (consistent with Colin spreadhseet) and R_dyn = 5
  % Ohms, which is guess since I don't know whether these chips are
  % low or hig R_dyn, see http://doc.bqemd.nist.gov/qsp/mux11c
  Shunt resistors                  & 5 & 18 \\
  Ti-6Al-4V spiders                & 130 & 130 \\
  Detectors (Optical + Electrical) & 0.25 & 1 \\
  Other in-band optical power      & 5.6 &   5.6 \\
  Out-of-band optical power        &   0 &   0 \\
  \BOSE readout wires              & 507 & 507 \\
\midrule
  Total                            & 700 & 794 \\
\bottomrule
\end{tabular}
\end{table*}

% xxx - I get 500 uW for the load from the four BOSE wires (300 K -
% 1K). Need to add them to the table. Can this be correct? I'm sure I did this calculation in the past and got a much smaller number. Did I screw up in the past? Is today's calculation in error? Am I intercepting this load at \SI{50}{\K} or \SI{4}{\K}, and I forgot about that? Did I use a much longer wire than \SI{1}{\m}? This could be a big cryogenic problem!

\section{Optical Design}\label{sec:ch4-optical-design}

Because I did not design the optical system, this section merely summarizes information about the optical system required for understanding the remainder of this thesis.
Also described are the optical filters located inside the cryostat.

The optical system is a Cassegrain design.
A Cassegrain design was chosen because circular symmetry makes these systems easy to design for on-axis performance at finite distances%
\footnote{William Duncan, personal communication}.
\figref{fig:ch4-optical-schematic} shows a schematic of the optical system including the focal plane.
Light enters the system from the left in this schematic, and bounces off the primary mirror onto the secondary mirror.
From the secondary mirror the light passes through a hole in the center of the primary mirror and through a window into the cryostat.
The cryostat window is a high-density polyethylene (\HDPE) lens that makes the system telecentric; this means that focal surface of the system inside the cryostat is planar, not curved, and simplifies the design of the detector focal plane.
On the focal plane itself, light is captured by smooth-walled conical feedhorns (see \sectionref{sec:ch4-feedhorn-design}) which funnel the light onto the detectors.

\begin{figure*}
\centering
\includegraphics[width=4in]{images/optics-labeled-fixed.jpg}
\caption{
Schematic showing elements of optical system.
\textbf{A} \SI{1.3}{\m} elliptical primary mirror.
\textbf{B} Platform on which secondary mirror is mounted.
\textbf{C} High-density-polyethylene (\HDPE) window leading into cryostat.
           This window acts as a lens to make the system telecentric.
\textbf{D} Detector focal plane package.
           The lid covering the focal plane holds an optical filter which defines the band of observation; see \sectionref{sec:ch4-filters}.
}
\label{fig:ch4-optical-schematic}
\end{figure*}

\tableref{tab:ch4-optical-specs} lists important properties and parameters for the optical elements of the system.
The most important parameter is the \SI{1.3}{\m} diameter of the primary mirror; this sets the resolution of the overall system.
\figref{fig:ch4-spot-diagrams} shows spot diagrams for the optical system generated by a \ZEMAX\ model for the system focused at \SI{16}{\m}.
They demonstrate the the system's optical performance is diffraction-limited over the entire focal plane for mirror rotations in both directions of up to \ang{1}, the maximum angle the mirror is ever displaced.
\tableref{tab:ch4-zemax-parms} lists important optical properties of the system obtained from the \ZEMAX\ model.


\begin{sidewaystable}[ht]
\centering
\caption{
  Optical system specifications.
  Some of the dimensions and parameters are listed to precisions higher than achievable manufacturing tolerances.
  These dimensions and parameters were chosen by optimization routines in \ZEMAX\, and are kept to full precision here for archival purposes.
  The shape of the elliptical and hyperbolic mirrors follows the equation $z(r) = c r^2 / (1 + \sqrt{1 - (1+k) c^2 r^2})$, where $c$ is the inverse of the mirror radius, $k$ is the conic parameter, and $r$ is the radial distance from the center mirrors.
  The lens surfaces follow the equation given in the table.
}
\label{tab:ch4-optical-specs}
\begin{tabular}{p{1.5in} p{1.5in} p{0.7in} p{4.9in} }
\toprule
Optical Element & Type & Outer \newline Diameter & Details \\
\midrule

Primary Mirror    & Elliptical Mirror    & \SI{1.3}{\m} 
    &  Vertex \SI{16}{\m} from far-field focal plane \\
& & &  \ZEMAX\ radius $c^{-1} = \SI{1801.453127}{\mm}$ \\
& & &  \ZEMAX\ conic $k = \num{-0.878728}$ \\
& & &  Semi-major axis: \SI{14.85}{\m} \\
& & &  Semi-minor axis: \SI{5.17}{\m} \\
& & &  Distances from mirror vertex to foci: \SI{0.93}{\m} and \SI{28.8}{\m} \\

Secondary Mirror  & Hyperbolic Mirror    & \SI{0.44}{\m}
    &  Vertex \SI{626.82}{\mm} from primary mirror vertex \\
& & &  \ZEMAX\ radius $c^{-1} = \SI{937.37187}{\mm}$ \\
& & &  \ZEMAX\ conic $k = \num{-4.466172}$ \\
& & &  Eccentricity \num{2.11} \\

Cryostat Window   & \HDPE\ Aspheric Lens & \SI{0.24}{\m}
    &  Outer vertex location depends on focus distance; see \tableref{tab:ch4-zemax-parms} \\
& & &  \SI{2}{\cm} thick at center \\
& & &  Index of refraction $n = \num{1.525}$ (\SI{7.6}{\percent} band-averaged reflection) \\
& & &  $\tan\delta = \num{4.0e-4}$ (\SI{9.1}{\percent} band-averaged absorption) \\
& & &  Outer Surface: $z(k) = c r^2 / (1 + \sqrt{1 - c^2 r^2}) + \sum_{k=1}^{k=8} \beta_k r^k$, where 
       $c^{-1} = \SI{-1052.933}{\mm}$,
       $\beta_1 = -\SI{4724.966}{\mm}$,
       $\beta_2 =  (\SI{0}{\mm})^{-2}$,
       $\beta_3 =  (\SI{88.649}{\mm})^{-3}$,
       $\beta_4 = -(\SI{67.854}{\mm})^{-4}$,
       $\beta_5 =  (\SI{90.995}{\mm})^{-5}$,
       $\beta_6 =  (\SI{89.966}{\mm})^{-6}$,
       $\beta_7 = -(\SI{97.618}{\mm})^{-7}$,
       $\beta_8 = -(\SI{191.31}{\mm})^{-8}$ \\
& & &  Inner Surface: $z(k) = c r^2 / (1 + \sqrt{1 - c^2 r^2}) + \sum_{k=1}^{k=8} \beta_k r^k$, where 
       $c^{-1} = \SI{-699.782}{\mm}$,
       $\beta_1 = -\SI{62.359}{\mm}$,
       $\beta_2 =  (\SI{0}{\mm})^{-2}$,
       $\beta_3 = -(\SI{45.520}{\mm})^{-3}$,
       $\beta_4 =  (\SI{59.477}{\mm})^{-4}$,
       $\beta_5 = -(\SI{74.141}{\mm})^{-5}$,
       $\beta_6 =  (\SI{86.306}{\mm})^{-6}$,
       $\beta_7 = -(\SI{107.763}{\mm})^{-7}$,
       $\beta_8 = -(\SI{124.984}{\mm})^{-8}$ \\
Detector Focal Plane &                   & N/A           &  \SI{171.541}{\mm} from vertex of back of cryostat window \\
\bottomrule
\end{tabular}
\end{sidewaystable}

% Here is how I calculated the inner/outer marginal rays using
% ZEMAX. It is a bit of a pain.
% 
% Click on "Gen" for the "General" dialog. Go to the "Aperture
% Tab". Under "Aperture Type" choose "Object Cone Angle". This sets
% the angle of the outermost ray. Now you can keep trying different
% angles and looking at the ray layout (Lay or L3d) until you find the
% ray that is touching the very edge of the secondary. Do this for
% both the outer and inner case --- secondary limits *both* cases, just
% in different ways!
%
% To get the offset (x) distance on the primary mirror, zoom in on
% where that ray crosses the primary and find the offset by looking
% at the coordinates in the title bar.
%
% The results are:
% 
% 16 m
% 
% inner - 5.142 deg / 229.10569 mm on primary
% outer - 13.61 deg / 610.40411
% 
% 17 m
% 
% inner - 5.252 deg / 228.59374 mm on primary
% outer - 14.025 deg / 614.50288 on primary
% 
% 28 m
% 
% inner - 5.989 deg / 225.12397 mm on primary
% outer - 16.908 deg / 642.22731 mm on primary

\begin{sidewaystable}[ht]
\centering
\caption{
  Parameters of the optical system extracted \ZEMAX\ simulations.
  ``Focus Distance'' is the distance from the apex of the primary mirror at which the system is focused.
  ``Lens Distance'' is the distance from the outer vertex of the lens to the vertex of primary.
  ``Plate Scale'' gives the ratio of distances on the far-field focal plane to the detector focal plane.
  ``Mirror Rotation'' give the distance that the on-axis point moves when the mirror is rotated by \ang{1}.
  ``F-Number'' Gives the ratio of the focal distance to the aperture size as viewed from the detector focal plane; this the ``Working F/\#'' from \ZEMAX.
  ``Marginal Rays'' give the angle away from the optical axis of the inner/outer marginal rays; these are the most extreme rays emerging from the detector focal plane that reach the far-field focal plane.
  ``Primary Range'' give the innermost and outermost radius of the primary that is actually used, e.g., when focused at \SI{16}{\m}, the outer \SI{4}{\cm} of the primary is not used.
}
\label{tab:ch4-zemax-parms}
\begin{tabular}{ccccccc}
\toprule
  \specialcell{Focus Distance \\ (\si{\m})} &
  \specialcell{Lens Distance \\ (\si{\mm})} &
  \specialcell{Plate Scale \\ (\si{\cm}/\si{\mm})} &
  \specialcell{Mirror Rotation \\ (\si{\cm}/\si{\degree})} &
  \specialcell{F-Number  \\ } & 
  \specialcell{Marginal Rays \\ (\si{\degree})} &
  \specialcell{Primary Range \\ (\si{\mm})} \\
\midrule
16 & 230.475 &  6.111 & 18.354 & 2.01 &  \ang{5.14} / \ang{13.61} & 229.1 / 610.4 \\
17 & 197.18  &  6.638 & 19.330 & 1.96 &  \ang{5.24} / \ang{14.03} & 228.6 / 614.5 \\ 
28 &   8.0   & 12.589 & 30.037 & 1.71 &  \ang{5.99} / \ang{16.91} & 225.1 / 642.2 \\
\bottomrule
\end{tabular}
\end{sidewaystable}

\begin{figure*}
\centering
\begin{tabular}{lr}
\includegraphics[width=3in]{images/ch4-zemax-spot-0-0.JPG}
 &
\includegraphics[width=3in]{images/ch4-zemax-spot-1-0.JPG}
 \\
\includegraphics[width=3in]{images/ch4-zemax-spot-0-1.JPG}
 &
\includegraphics[width=3in]{images/ch4-zemax-spot-1-1.JPG}
\end{tabular}
\caption{
\ZEMAX\ spot diagrams for the \Imager's optical system.
The system was focused to \SI{16}{\m}.
This represents the distribution of points that rays trace to on the far-field focal plane.
Each plot shows spot diagrams for nine points in the focal plane covering the area over which detectors in the subarray are located.
The four plots (moving left to right and top to bottom) are for the secondary mirror with no tilt, \ang{1} tilt about one axis, \ang{1} title about other axis, and \ang{1} tilt about both axes.
The black circle gives \ZEMAX's estimate of the size of the Airy disk for the optical system, which is defined as the location of the first null is the system's diffraction pattern.
In all cases, nearly all rays map to within the Airy disk, indicating that the performance of the optical system is diffraction-limited.
}
\label{fig:ch4-spot-diagrams}
\end{figure*}

The secondary mirror is mounted so that it can be pivoted in order to change where the system is pointing to in the far-field.
Two \BOSE\ linear motors\footnote{Bose ElectroForce, Framingham MA} mounted \ang{90} apart allow the mirror to be moved in arbitrary directions and scanning patterns.
\figref{fig:ch4-optics} shows the secondary mirror mounted on it's supporting platform, as well as the \BOSE\ motors.

\begin{figure*}
\centering
\includegraphics[width=3in]{images/optics.jpg}
\includegraphics[width=3in]{images/optics-bose.jpg}
\caption{
  Photographs of the optical system.
  \textbf{Left} The primary and secondary mirrors.
  The secondary is mounted on the Al platform in the upper left of the photograph.
  The black cables leading away from the platform go the the control system for the \BOSE\ linear motors.
  \textbf{Right} Close-up view of one of the \BOSE\ linear motors.
  The secondary mirror itself is located on the right of the photograph, while the right side of the photograph shows the \BOSE.
  A titanium strut connects the \BOSE\ to the mirror.
}
\label{fig:ch4-optics}
\end{figure*}

The \HDPE\ window does not have an anti-reflection coating.
The index of refraction of \HDPE\ is $n=1.525$ \cite{lamb_miscellaneous_1996}, so that expected reflection from the window should be $( (n-1)/(n+1) = \SI{4.3}{\percent}$.
The absorption coefficient for \HDPE\ at \SI{350}{\GHz} is $\alpha = \SI{0.044}{\cm^{-1}}$ \cite{lamb_miscellaneous_1996}, leading to \SI{8.5}{\percent} absorption in-band, due to the \SI{2}{\cm} thickness of the lens.

\section{Optical Filtering} \label{sec:ch4-filters}

Another important component of the system is the series of optical filters located inside the cryostat.
These filters are shown schematically in \figref{fig:ch4-filter-stack}.
These filters include a band-defining filter located at \SI{1}{\K}, thin thermal blocking filters located at \SI{300}{\K}, \SI{50}{\K}, and \SI{4}{\K}, and multi-layer low-pass filters at \SI{50}{\K} and \SI{4}{\K}.

The transmission of the bandpass filter is plotted in blue in \figref{fig:ch4-coupling}.
The integrated bandwidth of the bandpass filter is \SI{31.1}{\GHz}, with a full-width-half-maximum (\FWHM) band of \SI{35.0}{\GHz}.
There is some ambiguity in defining the bandwidth and efficiency of a bandpass filter that does not have a top-hat shape.
In this thesis I have chosen to described the filter as having a bandwidth of \SI{35.0}{\GHz} with efficiency $35.0 / 31.1 = \SI{88.5}{\percent}$.

The thermal blocking filters are sheets of \SI{3.3}{\um} polypropylene with metal (capacitive) grids printed on each side.
Because they are so thin, these filters have very little emission even at the infrared wavelengths at which polypropylene is highly absorptive \cite{tucker_thermal_2006}.
The cutoff wavelengths of these filters are listed in \tableref{tab:ch4-filter-stack}.

The \Imager\ also contains thicker multi-layer filters that act as low-pass filters at wavelengths closer to our observing band.
These filters are also made of metal meshes and polypropylene, but many meshes and polypropylene layers are sandwiched together to form filters that are \abt{\SI{1}{\mm}} thick.
These filters have excellent transmission in-band band as well as good out-of-band rejection \cite{ade_review_2006}, but the thick polypropylene substrates mean that they are also highly absorptive --- and thus also emissive -- in the near-infrared.
Additionally polypropylene has poor thermal conductivity.
This means that unless they are heat-sunk very well, the centers of the filters will be much warmer than the stage at which they are anchored, and so they will re-radiate infrared power into the cryostat.

\tableref{tab:ch4-filter-stack} lists all filters in the system and gives temperatures for the centers of some of the filters, taken while the cryostat was open optically.
The outermost multi-layer filter is very warm (\SI{190}{\K}), and this is a significant source of loading on the \SI{4}{\K} stage, which also warms the \SI{4}{\K} multi-layer filters, leading to a level of loading on the \SI{1}{\K} stage that is so high that the stage can not be brought below \SI{1.2}{\K} unless the aperture is stopped down to reduce IR loading.
This has been accomplished with a \SI{2.875}{\in} by \SI{2.875}{\in} aperture installed between the thermal blocking and multi-layer filters on the \SI{50}{\K} stage.
With this aperture installed, the bandpass filter center reaches a temperature of \SIrange{4}{8}{\K} (varying based on the size of the aperture stop).

Because there are no filters between the bandpass filters and the detectors (other than the highpass filter of the waveguide), all optical power emitted by the bandpass filters falls on the detectors.
xxx need to comment on both total power and photon noise once I work this out.

\tableref{tab:ch4-filter-stack} also lists the in-band transmission of each filter.
The total transmission of the filter stack excluding the bandpass filter is \SI{88.8}{\percent}.

\begin{figure*}
\centering
\includegraphics{drawings/ch4-filter-stack.pdf}
\caption{
  Schematic showing locations of all filters in the \Imager.
  Everything shown here is internal to the cryostat.
  On the left the nominal temperatures for each stage are listed, and on the right the names and cutoff wavelengths are listed.
}
\label{fig:ch4-filter-stack}
\end{figure*}

\begin{table*}
\centering
\caption{
  Details of all filters installed in the \Imager.
  Going down the table, the filters are listed in order from the outside of the cryostat to the inside.
  ``Stage'' refers to the cryostat temperature stage at which the filter is located.
  ``Cutoff'' refers to the point at which the filter transmission falls to \abt{-10}\,dB.
  ``Transmission'' gives the band-averaged transmission of the filter.
  ``Temperature'' gives the temperature at the center of a filter as measured by embedding a diode in a blob of Apiezon-N thermal grease placed on top of a layer of Kapton tape, to prevent applying grease to the filter itself.
  The temperatures of W128 was measured without the aperture stop in place while the filter was after THERM3 rather than before.
  The temperatures of W1269 and W1275 were measured both with and without a $(\SI{2.25}{\in})^2$ aperture stop, which covers \SI{40}{\percent} less area than the stop used in all other measurements in this thesis.
 Temperature measurements are not available for the other filters. 
}
\label{tab:ch4-filter-stack}
\begin{tabular}{llcccc}
\toprule
  \specialcell{Stage} &
  \specialcell{Filter Label} &
  \specialcell{Cutoff \\ (\si{\um}) } &
  \specialcell{Cutoff \\ (\si{\THz}) } &
  \specialcell{Transmission} & 
  \specialcell{Temperature \\ (\si{\K})} \\
\midrule
\SI{300}{\K} & THERM1 \SI{1.9}{\um}    & 1.9 & 158  & 1.00   &     \\
\SI{50}{\K} & THERM2 \SI{4}{\um}      & 4   & 75   & 1.00   &     \\
            & W1428 \SI{18}{\cm^{-1}} & 556 & 0.54 & 0.96   & 193 \\ % cooldown 12 6/9/10
            & THERM3 \SI{6}{\um}      & 6   & 50   & 1.00   &     \\
\SI{4}{\K}  & THERM4 \SI{0}{\um}      & 0   & 0    & 1.00   &     \\
            & W1266 \SI{14}{\cm^{-1}} & 714 & 0.42 & 0.94   &     \\
            & W1269 \SI{32}{\cm^{-1}} & 313 & 0.96 & 0.98   & 14--37   \\ % cooldowns 30 & 31
\SI{1}{\K}  & W1275                   & N/A & N/A  & 0.885 & 3.7--7.8 \\ % cooldowns 30 & 31
\bottomrule
\end{tabular}
\end{table*}

% is ripple in FTS results at low frew for lowpass filters due to
% reflection from the polypropylene?

\section{Feedhorn Design}\label{sec:ch4-feedhorn-design}

% xxx where do I explain the choice of a square grid? Ans: section 6.3
% xxx where do I explain the missing 20 detectors?

Millimeter and sub-millimeter astronomical instruments using \TES\ detectors use many different strategies for coupling light onto detectors: filled arrays of absorbers \cite{swetz_overview_2011,holland_scuba-2:_2013}, antennas with lenslets \cite{keating_ultra_2011}, phased antenna arrays \cite{obrient_antenna-coupled_2012}, corrugated feedhorns \cite{austermann_sptpol:_2012,niemack_actpol:_2010}, and smooth-walled conical feedhorns \cite{schwan_invited_2011,carlstrom_10_2011}.
The \Imager\ uses smooth-walled feedhorns because they are easy to design and easy to machine.
Smooth-walled feedhorns do not have the low cross-polarization properties of corrugated feedhorns \cite{clarricoats_corrugated_1984}, but this is not a concern here because the \Imager\ is polarization-insensitive.

\figref{fig:feedhorn-parms} depicts a smooth-walled conical feedhorn and its interaction with the optical system.
Although the optical system has a secondary mirror as well, for the purposes of feedhorn design the secondary mirror can be ignored and the system treated as a system of feedhorns illuminating only a primary mirror with a hole in its center \cite{goldsmith_quasioptical_1998}.
If a feedhorn is observing a temperature distribution $T_{target}(\theta,\phi)$, and is pointed in a direction $(\theta_0, \phi_0)$, then the temperature observed by the feedhorn will be
\begin{equation}
    T_{eff}(\theta_0,\phi_0) = \int \, d \Omega \, T_{target}(\theta - \theta_0,\phi - \phi_0) P(\theta,\phi).
\end{equation}
The function $P$ is called the ``beam'' or ``beam pattern'' of the feedhorn, and describes the pattern of radiation to which the feedhorn is sensitive.
Although the \Imager\ is always receiving radiation, never transmitting, this thesis often writes about the beam pattern as if it were transmitting radiation, e.g.\ in describing the portion of the ``beam'' that strikes the primary mirror.
This is acceptable because the transmitting and receiving beam patterns are exactly the same\footnote{This is a consequence of the reciprocity relationships obeyed by the Maxwell equations. See, e.g.\ \cite{balanis_antenna_2005} for a detailed explanation.}.
When using this expression one must keep in mind that the fraction of the beam that spills off the primary mirror (e.g.\ the unshaded region in \figref{fig:feedhorn-parms}) will see not the temperature distribution of the target, but a temperature distribution determined by what is beyond the primary mirror\footnote{Because of the presence of the secondary mirror, some of this temperature distribution will be in front of the system, and some will be behind.}.
The fraction of the beam that strikes the primary mirror and proceeds to the target is called the spillover efficiency $\eta_s$.

\begin{figure*}
\centering
\includegraphics{drawings/ch4-feedhorn-parms.pdf}
\caption{Schematic showing important parameters of a feedhorn and its beam. The beam appears to emerge from the phase center, a distance $l_c$ behind the mouth of the feedhorn in this diagram. The main lobe of the beam is approximated well by a Gaussian, here characterized by a full-width-half-maximum (\FWHM) beam width. The shaded fraction of the Gaussian represents the part of the beam that falls onto the primary mirror and reaches the target.}
\label{fig:feedhorn-parms}
\end{figure*}

The feedhorn opening diameter $D$ is chosen to minimize the total \NETD\ of the system.
The total \NETD\ was given in \sectionref{sec:ch2-noise} as
\begin{equation}
    NETD = \frac{NEP_{tot}}{2 k_B \Delta \nu \eta_{tot} \sqrt{2 \tau}}.
\end{equation}
To make the factors depending on the size of the feedhorns more clear we can break the optical efficiency $\eta_{tot}$ into a product of two factors, $\eta_s$ and $\eta_{other} = \eta_{tot} / \eta_s$.
We then note that the integration time per pixel $\tau$ is proportional to the number of detectors $N$.
This leads to
\begin{equation}
    NETD \propto \frac{NEP_{tot}}{\sqrt{N}\eta_s}.
\end{equation}

The critical relationship is that as a feedhorn's opening diameter increases, the width of the beam becomes smaller.
Small beam angles increase $\eta_s$, which improves \NETD.
However, the diffraction-limited area on the focal plane that can be covered by feedhorns is fixed, so that increases in the horn opening diameter reduce the number of detectors $N$, which worsens \NETD.
Choosing an optimal feedhorn size requires trading these two effect off against each other to minimize \NETD.

There are four additional factors to consider.
First, $NEP_{tot}$ is not necessarily independent of $\eta_s$.
In a detector-noise limited system it will be, but in a system that is photon-noise-limited, $NEP_{tot}$ may worsen, stay the same, or improve, depending on the temperature seen by the spilled over portion of the beam.
Indoors all of the beam will see roughly the same temperature as the target, so that total photon noise will stay the same.
Outdoors the situation is more complicated because the temperature seen by the spilled over beam will depend on the local scenery and weather conditions.
For simplicity, the analysis of optimum feedhorn size in this chapter made the assumption that the noise seen by a detector is independent of the beam size.

Second, for a Cassegrain optical system, $\eta_s$ is not a monotonic function of $D$.
The secondary mirror obstructs the central part of the beam, preventing it from reaching the target.
This means that narrow beams can have poor $\eta_s$ because a large fraction of the main lobe of the beam will be blocked.

Third, the choice of readout system and wiring places a firm upper limit on the number of detectors in the system of 1024.

Fourth, the dependence of the number of detectors $N$ on the feedhorn diameter $D$ is not a smooth function, because it is not possible to have, e.g., 1/3 of a feedhorn.
As explained in \sectionref{sec:ch5-layout}, the \Imager's detectors are laid out on a square grid.
So it is more helpful to think of the feedhorn diameter as taking on a discrete set of values that depends on the number of feedhorns per each side of the grid.

A \MATLAB\ program was used to find the optimum feedhorn size.
The program uses an analytic expression for the beam pattern developed in \cite{green_radiation_2006,narasimhan_modes_1971,}.
The far-field electric field pattern takes the form
\begin{equation}
    \vect{E}(\theta,\phi) = E_{\theta}(\theta) \sin{\phi} \hat{\theta} + E_{\phi}(\theta) \cos{\phi} \hat{\phi}.
\end{equation}
Here $E_{\theta}$ and $E_{\phi}$ are functions depending the horn diameter $D$ and opening half-angle $\alpha_0$, and involving definite integrals of Bessel functions, given in \appendixref{app:feedhorn-eqn}.
This expression is for the waveguide mode polarized along the $\pi/2$ direction.
The \Imager's detectors are unpolarized, so detect both waveguide polarizations equally (see \sectionref{sec:ch4-coupling} for confirmation of this via simulations).
The total power beam map is thus given by the incoherent sum
\begin{equation}
    P(\theta, \phi) = |\vect{E}(\theta, \phi)|^2 + |\vect{E}(\theta, \phi+\pi/2)|^2,
\end{equation}
which simplifies to 
\begin{equation}
    P(\theta) = |E_{\theta}(\theta)|^2 + |E_{\phi}(\theta)|^2,
\end{equation}
which is independent of $\phi$, as expected for an unpolarized detector.

To calculate the spillover efficiency of an individual feedhorn, the \MATLAB\ program integrates $P$ over the angles $\theta$ that illuminate the primary mirror and reach the target: \SIrange{5.3}{13.6}{\degree}\footnote{see \sectionref{sec:ch4-optical-design}}.
\figref{fig:ch4-feed-spill} shows a contour plot of spillover efficiency for an individual conical feedhorn as a function of $D$ and $\alpha_0$.
The blue asterisk matches the parameters for the feedhorns chosen for the \Imager.
This plot shows that $\eta_s$ depends much more strongly on $D$ than on $\alpha_0$.
$\alpha_0 = \SI{9.4}{\degree}$ was chosen as a value that is easy to machine, keeps the thermal mass of the feedhorn array relatively low (small values of $\alpha_0$ lead to long feedhorns and higher thermal mass), and it not too far from the maximum achievable $\eta_s$ for any fixed feedhorn diameter $D$.

\begin{figure*}
\centering
\includegraphics{./drawings/ch4-feed-spill.pdf}
\caption{
  Plot showing feedhorn spillover efficiency $\eta_s$ as a function of horn diameter $D$ and horn flare half-angle $\alpha_0$.
  The blue dotted line is for $\alpha_0 = \ang{9.4}$, the value assumed during optimization.
  The black dot shows the feedhorn parameters used in the \Imager: $D = \SI{2.68}{\mm}$ (cold) and $\alpha_0 = \ang{9.4}$
}
\label{fig:ch4-feed-spill}
\end{figure*}

To find the number of horns per array size that minimizes \NETD\, the program assumes that the the feedhorns must cover a square \SI{43.9}{\mm} per side.
It allows for \SI{1}{\mil} spacing between the edges of the feedhorns, and also accounts for a thermal contraction factor of 4.14 parts per thousand \cite[Appendix~A6.4]{ekin_experimental_2006}.
Four horns from each subarray are assumed to be missing in order to accommodate other features on the detector wafer.
The resulting \NETD\ estimates --- normalized to the \NETD\ for the actual feedhorns chosen --- is shown in \figref{fig:ch4-netd-vs-nfeeds}.
The optimal number of feedhorns per array side is 19, for a total (across all four sub-arrays) of 1428 feedhorns of diameter \SI{2.25}{\mm} cold.
Because of readout limitations, the actual number of detectors per side is 16, for 1004 feedhorns with diameter \SI{2.68}{\mm} (\SI{108}{\mils} at room temperature).
The loss in \NETD\ from this sub-optimal choice is only \SI{1.7}{\percent}.

\begin{figure*}
\centering
\includegraphics{./drawings/ch4-netd-vs-nfeeds.pdf}
\caption{
  Plot showing how \NETD\ depends on number of feedhorns in array.
  As discussed in the text, due to the square array the important parameter determining the total number of feedhorns is the number per side of the grid.
  The \NETD\ is plotted relative to the \NETD\ for an array with 16 feedhorns per side, which is the value chosen for this system.
  \NETD\ is minimized with 19 feedhorns per side, giving a total of 1428 feedhorns of diameter \SI{2.25}{\mm} (cold), but the \Imager\ uses 1004 feedhorns of diameter \SI{2.68}{\mm} (cold) because of readout limitations.
  The cost in \NETD\ is only \SI{1.7}{\percent}.
}
\label{fig:ch4-netd-vs-nfeeds}
\end{figure*}

% xxx - need to re-run all data analysis after verifying the proper
% inner angle!!!!

\figref{fig:ch4-beams} contains plots of both the beam pattern for the conical feedhorn and the far-field beam pattern of the entire optical system (also called the ``point spread function'').
The feedhorn beam pattern follows the model described above.
The far-field beam pattern accounts for the fact that the beam is truncated by the secondary mirror in it's center and by the edge of the primary at its edge; these limits are plotted as vertical blue lines on the left plot.
The truncation of the center of the beam is responsible for the side-lobe at \abt{\SI{}{\percent}} visible in the far-field beam pattern.
The far-field beam has a \FWHM\ of \SI{1.38}{\cm}.
After convolution with a \SI{0.2}{\in} circle the \FWHM\ is \SI{1.39}{\cm}, and after convolving with a \SI{1.791}{\cm} circle (same size as a dime) the \FWHM\ is \SI{1.76}{\cm}.

\begin{figure*}
\centering
\includegraphics{./drawings/ch4-beams.pdf}
\caption{
  Plot showed expected beam pattern for design feedhorns.
  \textbf{Left} The unpolarized power pattern for a conical feedhorn with diameter \SI{2.68}{\mm} and opening half-angle $\alpha_0 = \ang{9.4}$, at the center frequency for our band.
  The region between the vertical blue lines indicates the part of the beam that strikes the primary mirror and reaches the far-field focal plane.
  Also plotted is the best-fit Gaussian within the blue lines.
  The \FWHM beam width is \ang{10.6}.
  \textbf{Right} Far-field unpolarized power pattern.
  This accounts for blockage by the secondary mirror and well as the edge tape of the beam pattern on the primary.
  The side-lobe is caused by the secondary mirror, and is \SI{5.7}{\percent} high.
}
\label{fig:ch4-beams}
\end{figure*}

The feedhorns were machined out of Al-6061 and then gold plated.
An estimate of the insertion loss of the feedhorns requires the conductivity of this gold at \SI{1}{\K}. but this quantity is not known.
I assumed that the gold has the standard value for conductivity at room temperature of \SI{41e6}{\S\per\m}, with a residual resistivity ratio of 3.
Using this value, and assuming a surface roughness of \SI{5}{\um}, \HFSS\ simulations predict an insertion loss of \SI{-10.5}{\dB}, including both the feedhorn and the waveguide.
This corresponds to a feedhorn efficiency of \SI{91}{\percent}.
\HFSS\ simulations indicate that the return loss for the feedhorns is \SI{-28}{\dB}, so I ignore return loss.

\section{Optical Coupling to Detectors} \label{sec:ch4-coupling}

The feedhorns described in \sectionref{sec:ch4-feedhorn-design} funnel light coming into the cryostat into circular waveguide.
The waveguide carries the light to the bolometer, where it is absorbed by a Palladium Gold (PdAu) mesh.
This section describes the waveguide and absorbing structures.

The diameter of the circular waveguide was chosen to place the cutoff frequency of the first propagating mode (TE11) below the optical band of the \Imager.
The cutoff frequency of this mode is given by \cite[Chapter~5]{harrington_time-harmonic_2001}:
\begin{equation} \label{eqn:ch4-te11-cutoff}
  f_c = \frac{1.841 c}{2 \pi a },
\end{equation}
where $c$ is the speed of light in free space and $a$ is the radius of the waveguide.
For our waveguide I choose a diameter of \SI{0.6}{\mm}, which gives a cutoff frequency of \SI{292}{\GHz}, well below the lowest frequency in the band.
This choice was made so that the waveguide impedance was more uniform across the waveguide, which should make designing an efficiency absorbing structure easier.

The ideal absorbing structure for unpolarized light would be to cover the entire area of the waveguide with a sheet with a surface impedance equal to the waveguide impedance.
The characteristic impedance at frequency $f$ for a TE mode in waveguide is \cite[Chapter~2]{harrington_time-harmonic_2001}
\begin{equation} \label{eqn:ch4-wg-imp}
  Z (f) = \frac{\eta_f}{\sqrt{1 - (f_c/f)^2}},
\end{equation}
where $eta_f \approx \SI{377}{\ohm}$ is the impedance of free space and $f_c$ the cutoff frequency for the mode.
For our waveguide, the impedance at the band-center frequency of \SI{347}{\GHz} is \SI{700}{\ohm}.
The highest-resistance material available in the fabrication process for the \Imager's detectors is a \SI{20}{\nm} thick layer of PdAu, which for our fabrication process has a surface impedance of \SI{12}{\ohm}/sq%
\footnote{H.M.~Cho, Galen O'Neil, personal communication}%
, far too low to serve as an effective full-width waveguide absorber.

However, the \SI{12}{\ohm}/sq PdAu can still be used to create an absorbing structure with an effective sheet impedance of \abt{\SI{700}{\ohm}} by reducing the filling factor of the material, by, for example, making an absorber that consists of several thin strips.
A design rule-of-thumb for these absorbing structures is that the effective sheet impedance of the absorber is given by 
\begin{equation} \label{eqn:ch4-imp-fill-factor}
  Z_{eff} = R_s \frac{A_{tot}}{A_{abs}},
\end{equation}
where $R_s$ is the impedance of the absorbing material, $A_{abs}$ is the area covered by the absorbing material, and $A_{tot}$ is the total area of the waveguide.
This rule-of-thumb has been justified via semi-empirical means \cite{ulrich_far-infrared_1967,whitbourn_equivalent-circuit_1985} for free-space absorbing grids.
In waveguide, theoretical and empirical support for it's accuracy for a single strip in waveguide has also been given, provided that an additional factor of 2 is inserted in front of $A_{tot}$ \cite{datesman_analytical_2011}.

For the \Imager, I used this rule-of-thumb as a starting point, but to design the final absorbing structure I ran simulations using \HFSS\footnote{ANSYS, Inc., Canonsburg, PA}.
The model has the following features:
\begin{itemize}
\item \SI{0.6}{\mm} diameter circular waveguide leading to detector.
\item \SI{135}{\um} ``air gap'' between back of feedhorn array and detector wafer.
\item Proper back-short diameter of \SI{844}{\um} and back-short height of \SI{275}{\um}.
\item To keep the design simple, the back-short walls are not metalized.
      Rather, the detectors were fabricated on degenerate (Boron P-type) Si wafers with resistivity \SI{4}{\mOhm\cm}.
      The fields inside the Si are not solved for in this model.
      Instead, the Si is treated as (poor) conductor with a surface impedance appropriate for its conductivity.
      The skin depth in this Si are \SI{350}{\GHz} is \SI{5.5}{\um}, justifying this approximation.
      xxx what happens to the conductivity at low temp? Should go up, right? But does it still conduct, or does it start to act like a insulator?
\item The exact PdAu mesh used in the fabrication was included, using \SI{12}{\ohm}/sq as the surface impedance.
\item The Au heat-capacity ring (see \sectionref{sec:ch5-det-design}) was included as well.
\end{itemize}
\figref{fig:ch4-hfss-model} shows a schematic of the \HFSS\ model, including the fields traveling down the waveguide.
It also shows a close-up view of the current distribution on a portion of the PdAu mesh.

\begin{figure*}
\centering
\includegraphics[width=3in]{images/ch4-hfss-model.png}
\includegraphics[width=3in]{images/ch4-hfss-grid.png}
\caption{
  Screen-shots taken from the \HFSS\ model used to simulate the absorbing grid.
  \textbf{Left}
  View of the full model, with the magnitude of the electric field plotted on a plane bisecting the model.
  \textbf{Right}
  Close-up view of the absorbing grid, with the surface current density plotted.
  The waveguide mode being excited is polarized in the up/down direction.
  Because the wire are thin compared to the width of the waveguide (\SI{2}{\um} vs \SI{600}{\um}), the current distribution is difficult to see in this plot.
  But close examination shows that the wire perpendicular to the excited polarization have very low current, while those parallel have higher current.
  The current decays towards the edges of the waveguide.
}
\label{fig:ch4-hfss-model}
\end{figure*}

The results of these simulations are shown in \figref{fig:ch4-coupling}.
The band-averaged coupling efficiency of the mesh for unpolarized light is \SI{87}{\percent}.
Although the two polarizations have different absorption curves, with peak absorption at different frequencies, after averaging over the band their difference in efficiency is only \abt{\SI{1}{\percent}}.

Also shown in \figref{fig:ch4-coupling} is the frequency at which the TM01 waveguide mode turns on: \SI{383}{\GHz}.
Only \SI{0.9}{\percent} of the bandpass is located above this frequency, so I have ignored any coupling to TM01 and all higher order modes.

\begin{figure*}
\centering
\includegraphics{./drawings/ch4-coupling.pdf}
\caption{
  Plot showing coupling efficiency of the detector absorbing structures.
  The blue line is the transmission of the bandpass filter.
  The red and brown lines show the fractional power absorbed in the grid absorber.
  Although theses curves have different shapes, their integrated absorption over the band is withing \SI{1}{\percent} of each other.
  The band-averaged absorption for unpolarized light is \SI{87}{\percent}.
  The blue dashed vertical line at \SI{283}{\GHz} indicates the cutoff frequency of the next-highest-order mode, TM01; \SI{0.9}{\percent} of the bandwidth is above this frequency.
}
\label{fig:ch4-coupling}
\end{figure*}

\figref{fig:ch4-coupling-mis} contains plots showing the effect of misalignment between the feedhorns and the detectors.
For small misalignment (less than $\SI{100}{\um} \approx \SI{4}{\mils}$) the loss in coupling efficiency is small.
But for large misalignment the loss is efficiency will be large, and one mode will couple much worse than the other.
Because the beam for each individual mode is elliptical, this differential mode-coupling will lead to elliptical beams.
This could be part of the explanation for the combination of poor optical efficiency and elliptical beams described in \chapterref{c:imaging}.

\begin{figure*}
\centering
\includegraphics{./drawings/ch4-coupling-mis.pdf}
\caption{
  Plots showing impact of misalignment between feedhorns and detectors.
  The left plot shows band-averaged coupling efficiency vs misalignment, for misalignment in both $x$ and $y$ directions.
  The right plot shows the ratio of band-averaged coupling efficiency of the $x$-polarized mode to the $y$-polarized mode, again for misalignment in both $x$ and $y$ directions.
  For misalignment up to \SI{100}{\um} loss in coupling efficiency is small.
  But for large misalignment the efficiency drops substantially, and the modes couples very differently, which will lead to a elliptical beams.
}
\label{fig:ch4-coupling-mis}
\end{figure*}

\section{Predicted Optical Efficiency and Optical Loading on Detectors} \label{sec:ch4-opt-eff}

The waveguide behind the feedhorns in the \Imager\ causes the detectors to be sensitive to only the two degenerate TE11 waveguide modes over \SI{99}{\percent} of their bandwidth.
Because some of the light detected by the \Imager\ is reflected, we expect the light to be polarized to some extent.
But to simplify the analysis, I assume here that the light is unpolarized, so that the detectors are sensitive to two uncorrelated waveguide modes.

The optical power from a source of temperature $T$ in a single mode detected by a detector with efficiency $\eta(\nu)$ is given by \cite{zmuidzinas_thermal_2003,richards_bolometers_1994}
\begin{equation} \label{eqn:ch4-power-per-mode}
  \int \eta(\nu) h \nu n(\nu,T) d \nu ,
\end{equation}
where $n$ is the photon occupation number give by the Bose-Einstein factor
\begin{equation} \label{eqn:ch4-photon-n}
  n(\nu,T) = \frac{1}{e^{\frac{h \nu}{k_B T}} -1}.
\end{equation}
For \SI{350}{\GHz} light emitted by a \SI{300}{\K}, $n \approx 17$, so the Raleigh-Jeans limit $h \nu \ll k_B T$ holds, and \eqnref{eqn:ch4-power-per-mode} simplifies to
\begin{equation} \label{eqn:ch4-power-per-mode-rj}
  2 k_B T \int \eta(\nu) d \nu = 2 k_B T \eta_{tot} \Delta \nu,
\end{equation}
where $\eta_{tot}$ is the total optical efficiency of the system and $\Delta \nu$ is the optical bandwidth.
My calculations below assume the exact from of \eqnref{eqn:ch4-power-per-mode}, but \eqnref{eqn:ch4-power-per-mode-rj} is very useful for quick calculations and checks.
Photon \NEP\ is given by \cite[Equation~51]{zmuidzinas_thermal_2003}
\begin{equation} \label{eqn:ch4-photon-nep}
  \NEPph = 4 \int (h \nu)^2 \eta(\nu) n(\nu,T) (1 + \eta(\nu) n(\nu,T)) d \nu.
\end{equation}
Here the second term represents ``photon bunching'', which only becomes important when many photons are occupying a spatial mode.
For millimeter and sub-millimeter astronomy observing cold sources such as the Cosmic Microwave Background Radiation, this second term is often negligible, but it is important for the \Imager\ because we are observing \abt{\SI{300}{\K}} targets.

There are two points worth noting about \eqnref{eqn:ch4-photon-nep}.
First, it is tempting to simplify \eqnref{eqn:ch4-photon-nep} as:
\begin{equation}
  \NEPph = 4 (h \nu_0)^2 \eta_{tot} n(\nu_0,T) (1 + \eta_{tot} n(\nu_0,T)) \Delta \nu,
\end{equation}
where $\nu_0$ is the central frequency of the band.
However, this will not give the correct answer unless $\eta_{tot}$ and $\Delta \nu$ have been defined so that $\eta_{tot} = \int \eta^2(\nu) d\nu / \int \eta(\nu) d\nu$ and $\Delta \nu = \int \eta(\nu) d \nu / \eta_{tot}$.
I have not defined the bandwidth of the \Imager\ in this way, so I always evaluate the integral in \eqnref{eqn:ch4-photon-nep}.

Second, the \Imager\ is not viewing a single temperature $T$ that reaches the detectors with efficiency $\eta$.
Rather, the \Imager\ is viewing an \abt{\SI{300}{\K}} source that is attenuated by a series of attenuators and other lossy elements, each of which is at non-zero temperature and so emits power itself.
Because of the photon bunching term, it is not correct to calculate \NEPph\ for each optical component separately and then sum all values; doing so will underestimate the noise due to photon bunching.

The correct generalizations of \eqnref{eqn:ch4-power-per-mode} and \eqnref{eqn:ch4-photon-nep} is to treat the term $\eta(\nu) n(\nu,T)$ as an total photon occupation number that includes contributions from all sources.
I make the simplifying approximation that the frequency-dependence of the efficiency is the same for all source of optical power; i.e., it is set by the bandpass filter.
The total occupation number absorbed in the detectors is then given by
\begin{equation} \label{eqn:ch4-tot-n0}
  \eta(\nu) n(\nu,T) \equiv \tau_{bp}(\nu) \sum_k \eta_k \epsilon_k n_k(\nu,T_k).
\end{equation}
Here $\tau_{bp}(\nu)$ is the transmission of the bandpass, $\eta_k$ is the cumulative efficiency for light from source $k$ to be absorbed in the detector, $\epsilon_k$ is the emissivity of source $k$ and $n(\nu,T_k)$ is the Bose-Einstein factor \eqnref{eqn:ch4-photon-n} for source $k$, which is at temperature $T_k$.

Under these assumptions the expression for optical power becomes
\begin{equation} \label{eqn:ch4-opt-pow-all}
  2 \int h \nu \tau_{bp}(\nu) \left( \sum_k \eta_k \epsilon_k n(\nu,T_k) \right) d \nu.
\end{equation}
Note that in this case the optical powers from each source can be calculated separately and them summed.
The \NEPph\ is given by
\begin{equation} \label{eqn:ch4-nep-all}
  4 \int (h \nu)^2 \tau_{bp}(\nu) \left( \sum_k \eta_k \epsilon_k n(\nu,T_k) \right) 
       \left( 1 + \tau_{bp}(\nu) \left( \sum_k \eta_k \epsilon_k n(\nu,T_k) \right)  \right) d \nu.
\end{equation}


% \tableref{tab:ch4-opt-eff} lists all components of the \Imager\ and their contribution to optical efficiency.
\tableref{tab:ch4-opt-load} lists all components of the \Imager\ that contribute to optical loading.
For each component \eqnref{eqn:ch4-power-per-mode} is integrated over the \Imager's optical band, the emissivity factor from the table is applied uniformly over the band, and the transmission of all components after the given component is accounted for.
The temperatures listed for the filters inside the cryostat are estimates based on measurements described in \tableref{tab:ch4-filter-stack}.
Also listed are the photon occupation quantities $\epsilon_k \eta_k n(\nu_0,T_k)$ at the band-center frequency.
The predicted total optical loading on each detector is \SI{180}{\pW}, and the predicted optical $NEP_{ph} = \Pnoisef{0.85}$.

\begin{table*}
\centering
\caption{
  Optical load and photon noise from all components in the \Imager\ that contribute to optical load.
  $P_{opt}$ for each component is calculated according to \eqnref{eqn:ch4-opt-pow-all}.
  $\epsilon \eta n$ is calculated at the the center frequency of the band (\SI{347}{\GHz}) and also includes the transmission of the bandpass filter at that frequency.
  xxx what about bandpass filter?
  All powers are the power absorbed in the bolometer, and the \NEPph\ value is referred to power absorbed in the bolometer.
  The efficiency of the lens accounts for losses due to both reflection from the lens and absorption in the lens.
  ``Beam'' refers to power from the far-field that reaches detectors, while ``Spillover'' refers to power that reaches the detectors due to the portions of the beam that misses the primary mirror.
  % xxx include this? Listing optical power from each of the lens, beam, and spillover is not strictly neccessary, because I assume that all three sources are at the same temperature; I could equally well calculate optical power from a single \SI{295}{\K} source with emissivity of 1.0 looking into an optical system with efficiency \num{0.65}, the cumulative efficiency at the W1428 filter.
}
\label{tab:ch4-opt-load}
\begin{tabular}{ccccccc}
% The content of this table is producted by thesis/calc_optical_loading()
\toprule 
  Component  & 
  \specialcell{Temperature \\ (\si{\K})} & 
  Efficiency & 
  Emissivity & 
  \specialcell{$P_{opt}$ \\ (\si{\pW})} & 
  $\epsilon \eta n$ & 
  \specialcell{Cumulative \\ Efficiency} \\  
\midrule 
  Bolometer  &   1 & 0.87 &      &       &      & 0.87 \\ 
  Feedhorn   &   1 & 0.91 &      &       &      & 0.79 \\ 
  W1275      &   5 & 0.89 &      &       &      & 0.70 \\ 
  W1266      &  16 & 0.98 & 0.02 &   0.1 &  0.0 & 0.69 \\ 
  W1269      &  85 & 0.94 & 0.06 &   2.9 &  0.2 & 0.65 \\ 
  W1428      & 200 & 0.96 & 0.04 &   4.7 &  0.3 & 0.62 \\ 
  Lens       & 295 & 0.84 & 0.16 &  27.7 &  1.5 & 0.52 \\ 
  Beam       & 295 & 0.52 & 0.48 &  69.5 &  3.8 & 0.27 \\ 
  Spillover  & 295 & 0.00 & 1.00 &  75.3 &  4.1 & 0.00 \\ 
\midrule 
  Total      &     &      &      & 180.2 &  9.8 & 0.27 \\ 
\midrule 
  Total $\sqrt{\NEPph}$ & \Pnoise{0.85e-15} &   &  & & & \\ 
\bottomrule
\end{tabular}
\end{table*}

\section{Detector Readout} \label{sec:det-readout}

As described in \chapterref{c:tes}, the \Imager's detectors are voltage-biased, so that the detector output signal is a changing current.
To read out the detectors a low-noise current amplifier is required.
The \Imager uses a multiplexed \SQUID\ readout system to accomplish this.

A \SQUID\ is a Superconducting Quantum Interference device \cite{clarke_squid_2002}.
For the purposes of understanding this thesis, and the operation of the \Imager, it suffices to know that a \SQUID\ is a very sensitive magnetometer, and by coupling the magnetic field produced by a current into the \SQUID, the \SQUID\ can also be used to read out a current signal.

An important aspect of the response of a \SQUID\ to a input current is that the response is not linear; it is periodic.
\figref{fig:tes-bias-ramp-sc} and \figref{fig:tes-bias-ramp-trans} show examples of this behavior for the \SQUID's used in the \Imager.
This periodic response means that the readout system is not measuring the absolute current passing through the detectors, but rather it measures changes in the current from some offset value, an offset value which is not known and can be different for each detector.
When processing a set of detector outputs into a video, these detector offsets must be accounted for.
The algorithm used to do this is described in \sectionref{sec:ch8-algo}.

To reduce the number of wires running into the cryostat, the \Imager\ uses a time-division multiplex (\TDM) readout system \cite{chervenak_superconducting_1999,korte_time-division_2003,reintsema_prototype_2003}.
\figref{fig:ch4-tdm-schematic} shows a schematic of this readout system.
The detectors are divided into a number of columns, each column containing 32 rows.
Each detector output is coupled into the input coil of it's own ``1st-stage'' \SQUID.
Row address lines bias the 32 rows sequentially, so that at any given time only one detector per column is being read out by the system.
This sequential addressing means that the current noise power spectral density of the \SQUIDs is increased by a factor of 32 due to aliasing, but the current noise in the \Imager's \SQUIDs is low enough that \SQUID\ noise is not a significant contribution to the total noise of the detectors.
See, e.g., \figref{fig:rsh-l-plots} and \figref{fig:ch7-trans-noise} for a demonstration that \SQUID\ (\abt{\Inoise{1e-10}}) is noise is below the typical noise level of a detector biased into it's standard operation conditions (\abt{\Inoise{1e-9}}).
To linearize the \SQUID\ amplifier chain, feedback in the form of magnetic flux is applied to the 1st-stage \SQUIDs.

\begin{figure*}
\centering
\includegraphics[width=3in]{images/ch4-tdm-schematic.png}
\caption{
  Schematic of the Time-Division Multiplexing (\TDM) readout system used by the \Imager.
  The detectors are divided into rows and columns (2 each shown here).
  Each detector is coupled to it's own 1st-stage \SQUID (\SQ1).
  At any given time, only one row of \SQUIDs is biased, indicated via the ``Row address currents'' on the left.
}
\label{fig:ch4-tdm-schematic}
\end{figure*}

A set of electronics is required to control the multiplexed readout system and provide data to a computer for further analysis.
The \Imager uses the Multi-Channel Electronics (\MCE) as the electronics control system \cite{battistelli_functional_2008,battistelli_automated_2008,_mcewiki_2014}.
The \MCE\ was developed in order to achieve simple, remote operation of \TDM\ readout systems for sub-millimeter and millimeter astronomy.
It comes equipped with an software suite for controlling the electronics itself.
The entire electronics system for reading out 1024 detectors is contained within a single $\SI{15}{\in} \times \SI{14}{\in} \times \SI{14}{\in}$ crate.
All communication with the controlling data acquisition computer is via fiber optic cable.
See \figref{fig:ch4-mce-photo}) for a photo of the \MCE.

\begin{figure*}
\centering
\includegraphics[width=3in]{images/ch4-mce.jpg}
\caption{
  Photograph of the \MCE\ unit used to readout the \Imager's detectors.
  The \MCE\ is \SI{15}{\in} wide.
}
\label{fig:ch4-mce-photo}
\end{figure*}

The \MCE has many parameters that can be set to control its behavior, almost none of which I will describe here.
The exceptions are the parameters that control the readout rate of the system, listed in \tableref{tab:ch4-mce-parms}.
The table also gives the value of these parameters used by the \Imager when taking videos.
The rate at which data is reported to the data acquisition computer is given by
\begin{equation} \label{eqn:ch4-mce-readout-rate}
  f_{ro} = \frac{ \SI{50}{\MHz} }{\texttt{row\_len} \times \texttt{num\_rows} \times \texttt{data\_rate} }
\end{equation}
Using the values in \tableref{tab:ch4-mce-parms} the readout rate is \SI{3125}{\Hz}.

Using any value for \texttt{data\_rate} other than 1 will lead to aliasing of noise.
As discussed in \sectionref{sec:ch7-aliasing}, this raises the total amount of noise in the \Imager's detectors by \SI{14}{\percent} on average.
This noise aliasing penalty can be reduced by configuring the \MCE\ to internally apply a digital 4-pole low-pass filter prior to reporting data to the data acquisition computer \cite{mce_team_digital_????}.

\begin{table*}
\centering
\caption{
  A few important configuration parameters for the \MCE.
}
\label{tab:ch4-mce-parms}
\begin{tabular}{lcp{4in}}
% The content of this table is producted by thesis/calc_optical_loading()
\toprule 
  Parameter  & 
  Typical Value & 
  Explanation \\  
\midrule 
  \texttt{row\_len}  & 100 &
           Number of \SI{50}{\MHz} clock cycles spent on each row.  \\
  \texttt{num\_rows} & 32 &
           Number of rows to cycle over. \\
  \texttt{data\_rate} & 5 & The \MCE\ only reports every \texttt{data\_rate}-th sample to the data acquisition computer. \\
\bottomrule
\end{tabular}
\end{table*}
\section{Acknowledgments}

Bob Schwall and William Duncan of \NIST\ designed the cryostat/mirror mount and the \He4 sorption refrigerator.
The refrigerator was filled with \He4 by Simon Dicker at the University of Pennsylvania.
Bob Schwall of \NIST\ designed the cryostat, and provided useful advice and help in the lab during commissioning of the cryostat.
William Duncan designed, procured, and assembled the optical system.
Mandana Amiri and Matthew Hasselfield provided extensive, timely and invaluable support for the MCE hardware and software.



\chapter{Detector Design}\label{c:det-design}

This chapter describes the design and choice of parameters for the \TES\ bolometers that are used in the \Imager.

\section{Parameter Choice For Our Bolometers} \label{sec:det-parm-choice}

The primary parameters to be chosen when designing a \TES\ bolometer are the superconducting critical temperature $T_c$, the thermal conductance $G$, and the \TES\ island heat capactiy $C$.
These parameters are interrelated, and so cannot be chosen entirely independently of each other.
Some of the factors to consider are:
\begin{itemize}
  \item Detector noise scales with $\sqrt{T_c^2 G}$, so that lower values of $G$ and $T_c$ are better
  \item The saturation power of the \TES\ detector scales roughly like $G T_c$, so that if $G$ and $T_c$ are too small, the optical power falling on the detector will raise the temperature of the membrane above $T_c$, causing the device to not work.
    % do i talk about saturation power in ch 3? if not, I should!
  \item $T_c$ must be chosen to be higher than the acheivable bath temperature, and the bath temperature also affects the saturation power.
  \item The targetted detector time constant $\tau_{eff}$ depends not only on the detector natural time constant $\tau = G / C$ but also on the values of \Loop and $\beta_I$ at the chosen bias point.
        The loop gain depends primarily on the detector $\alpha$, but also on the saturation power.
\end{itemize}

% xxx say something about no attempt to shape transition - we get whatever alpha we get. no need to slow down trans, which is what normal metal bars do.

The following subsections outline the choice of $T_b$, $T_c$, $G$ and $C$ for the detectors in the first 251-detector sub-array.

\subsection{Choice of $T_b$ and $T_c$}

The relationship between detector noise and saturation power can be examined in more detail.
From \eqnref{eqn:ch3-g-fit} we have
\begin{equation} \label{eqn:ch5-psat}
P_{sat} = \frac{G T_c}{n}\left(1 - \left(\frac{T_b}{T_c}\right)^n\right).
\end{equation}
This can be solved for $G$ and substituted into the expression for \TES\ thermal fluctuation noise in \tableref{tab:tes-noise}, leading to
\begin{equation} \label{eqn:ch5-tes-noise}
S^2_{TFN} = \frac{n F(T_0, T_b) T_0 / T_b}{1-(T_b/T_c)^n} 4 k_B T_b P_{sat}.
\end{equation}
The \TES\ temperature $T_0$ only appears in the pre-factor, which depends only on the power-flow index $n$ and the ratio $T_b/T_0$.
This means that for fixed $P_{sat}$ and $T_b$, the ratio $T_c/T_b$ that gives the lowest detector noise depends only on $n$ and the form of $F$.
For values of $n$ in the range 3--4, this optimal ratio is $T_c \approx 1.8 T_b$, while the pre-factor itself is \abt{3.7}.

As discussed in \sectionref{sec:ch4-opt-eff}, the predicted loading on the \Imager's detectors is \SI{180}{\pW} and the photon noise from this load is \Pnoisef{0.85}.
Choosing a safety factor of 3 so that $P_{sat} = 3 \times \SI{200}{\pW}$, and targeting detector noise equal to \SI{50}{\percent} of the photon (so that total noise is a factor of $\sqrt{1.5} = 1.22$ higher than photon noise) we find that in order for the detector noise to be below the predicted photon noise we require
\begin{equation}
  T_b < \frac{1}{3.7} \frac{\NEPph^2}{4 k_B P_{sat}} =
        \frac{1}{3.7} \frac{0.5 \times (\num{0.85e-15})^2}{4 \times \SI{1.38e-23}{\J\per\K} \times 3 \times \SI{180}{\pW}} = 
        \SI{3.6}{\K}
\end{equation}
It would seem that we should be able to run the system off of a Pulse Tube Cooler.

However, this leaves very little margin for error in the design and implementation of the system, so for the \Imager\ we chose to use a \He4-sorption fridge to set the bath temperature.

The \Imager's base bath temperature under optical load is \SI{920}{\pW} (see \sectionref{xxx}).
The initial hopes for performance of the \He4-sorption fridge were that it's base temperature would be ~\SI{650}{\mK}, implying an ideal $T_c$ of \abt{\SI{1.2}{\K}}.
This is a convenient $T_c$ because it is the critical temperature of elemental Al \cite{xxx}, so Al was chosen as the \TES\ material.

In practice, it was discovered during testing of Al prototype detectors that the base temperature of the system was \SI{950}{\mK} under optical load, so that a higher $T_c$ could lead to better noise performance.
However, in order to change as little as possible between the prototype detectors and the first sub-array, I decided to continue using Al.

\subsection{Choice of $G$}

\tableref{tab:ch5-proto-parms} lists the measured properties and parameters of the prototype detectors.
These detectors had $P_{sat}$ at $T_b = \SI{970}{\mK}$, 5.1 times higher than the predicted optical load and 6.1 times the measured optical load.
A safety factor of 5--6 is overly conservative, so for the sub-array I decided to target a $G$ value of \SI{3.8}{\nW\per\K}, for a safety factor of 3.9 -- 4.7.

\begin{table*}
\centering
\caption{
  Measured Properties of Prototype Detectors.
  The methods and procedures used to measure these properties were the same as described for the sub-array in \chapterref{c:det-array}.
} 
\label{tab:ch5-proto-parms}
\begin{tabular}{l c}
\toprule
  Detector Property &  {Value} \\
\midrule
  $T_c$                 & \SI{1.2}{\K} \\
  $R_n$                 & \SI{3.4}{\mOhm} \\
  $n$                   & 3.9 \\
  $G$                   & \SI{5}{\nW\per\K} \\
  $\tau$                & \SI{12}{\ms} \\
  $\tau_{eff}$ (typical) & \SI{4}{\ms} \\
  $C = G \tau $         & \SI{60}{\pJ\per\K} \\
  $P_{opt}$              & \SI{150}{\pW\per\K} \\
  $\eta_{tot}$           & 0.25 \\
  $P_{sat}|_{\SI{970}{\mK}}$          & \SI{920}{\pW} \\
\bottomrule
\end{tabular}
\end{table*}

\subsection{Choice of $C$}

The detector's heat capacity $C$ is chosen to target a specific natural time constant $\tau$ once $G$ is chosen.
$\tau_{eff}$ in the prototype detectors was higher than the desired value of \SI{1}{\ms}.
To reduce risk of problems such as instability with the sub-array detectors, I decided to reduce $C$ to \SI{30}{\pJ\per\K}, which would give $\tau = \SI{8}{\ms}$.
As long as \Loop and $\beta_I$ at the operating bias point did not change, this would lead to $\tau_{eff} = \SI{2.6}{\ms}$.

\section{Detailed Detector Design} \label{sec:ch5-det-design}

The \Imager's detectors are fabricated using standard lithographic clean-room techniques on Si wafers.
In order to achieve the targetted $G$ and $C$ values the detectors are located on a suspended SiN membrane which is connected to the rest of the Si wafer by a set of thin ``legs''.
\figref{fig:ch5-det-layout} shows a cross-sectional schematic of the detectors, showing that they are suspended with no Si beneath them, as well as a labeled photograph of a prototype detector (the sub-array detectors are identical except for the length of the legs and the thickness of some of the layers; see \tableref{tab:ch5-det-dims}).

\begin{figure*}
\centering
\includegraphics[width=3.9in]{images/ch5-det-schematic.png}
\includegraphics[width=2.3in]{images/ch5-proto-pixel-labeled.png}
\caption{
  \textbf{Left} Cross-sectional schematic of a \Imager detector.
  The schematic is not to scale.
  \textbf{Right} Photograph of a prototype detector.
  The detectors fabricated for the sub-array are identical except for the length of the legs and the thickness of some of the layers; see \tableref{tab:ch5-det-dims}.
  The labeled parts of the detector are \textbf{A} Al \TES\ \textbf{B} Au heat capactiy ring \textbf{C} SiN leg connecting detector to substrate \textbf{D} PdAu absorbing mesh \textbf{E} PdAu heater resistor.
}
\label{fig:ch5-det-layout}
\end{figure*}

The detectors are fabricated on \SI{275}{\um} thick double-side-polished degenerate (Boron P-type) Si wafers.
A layer of SiO2 is grown on top of the silion, followed by a layer of SiN, prior to the main fabrication steps.
During fabrication the \TES\, Nb wiring leads, PdAu absorber, and Au heat-capacity ring are laid down, as well as an additional layer of insulator to allow wiring layers to cross over each other.
The SiN and SiO2 is removed from the areas between the legs, and then a Deep Reactive Ion Etch process is used to remove all silicon from behind the detector membrane.
This etch process does not remove SiN or SiO2, so that the legs, which still have SiN, are left in place.
The result is a susepended membrane connected to the rest of the wafer by a set of ``legs'' which provide the thermal conductance $G$.

\tableref{tab:ch5-det-dims} lists dimensions for both the prototype and sub-array detectors.

The leg geometry of the prototype detectors was based on a set of measurements taken at \NIST\ on SiN membranes at temperature around \SI{1}{\K}.
the leg geometry for the sub-array detectors was based on simple scaling of the prototype detecctors to the target $G$ of \SI{3.8}{\nW\per\K}.
This scaling was slightly complicated by the change in thickness of the SiO2 layers, which was made in order to add additional protection to the wiring laayers and better balance stress on the relieved membranes.
Assuming that $G$ scales linearly with the $A/L$ of the legs, the sub-array $G$ should be
\begin{align}
  G_{sub} & = G_{proto} \frac{A_{sub}}{A_{proto}} \frac{l_{proto}}{l_{sub}} \\
         & = G_{proto} \frac{(500 + 250 + 200)(11)}{(500 + 120 + 120)(11)} \frac{40}{67} \\
         & = \SI{5.0}{\nW\per\K} \times 1.28 \times 0.597 \\
         & = \SI{3.8}{\nW\per\K} 
\end{align}

\begin{table*}
\centering
\caption{
  Dimensions of prototype and sub-array detectors.
  The values shown for the sub-array detectors are those that are different from the prototypes; all other sub-array dimensions are the same as for the prototypes.
} 
\label{tab:ch5-det-dims}
\begin{tabular}{l c c}
\toprule
  Detector Dimension &  {Prototype} & Sub-Array \\
\midrule
  \TES\ Size           & $\SI{64}{\um} \times \SI{70}{\um}$ & \\
  \TES\ Thickness      & \SI{250}{\nm}       & \SI{180}{\nm} \\
  SiN Thickness        & \SI{500}{\nm}       & \\
  SiO2 Base Thickness  & \SI{120}{\nm}       & \SI{250}{\nm} \\
  SiO2 Cover Thickness & \SI{120}{\nm}       & \SI{200}{\nm} \\
  Number of Legs       & 8                   & \\
  Leg Length           & \SI{40}{\um}        & \SI{67}{\um} \\
  Leg Width            & \SI{11}{\um}        & \\
  Gold Ring Area       & $(\SI{393}{\um})^2$ & \\
  Gold Ring Thickness  & \SI{2000}{\nm}      & \SI{1000}{\nm} \\
  Nb Lead Width        & \SI{6}{\um}         & \\
  Nb Lead Thickness    & \SI{200}{\nm}       & \\
  PdAu Thickness       & \SI{20}{\nm}        & \\
\bottomrule
\end{tabular}
\end{table*}

\tableref{tab:ch5-det-heat-capacity} shows contributions to the heat capactiy from all components of the bolometer.
The membrane and Al \TES\ alone have insufficient heat capacity, so an Au ring was added to provide the targetted heat capacity.
The total heat capacity is dominated by the Au ring.

\begin{table*}
\centering
\caption{
  Contributions to total heat capacity of sub-array detectors.
  Note that the Debye contribution for Au at \SI{1.2}{\K} is still significant, so must not be ignored.
} 
\label{tab:ch5-det-heat-capacity}
\begin{tabular}{l S S S l}
\toprule
  Component & {Volume ($10^{-9}$\,\si{\cm^3})} & {$C_V$ (\si{\uJ\per\K\per\cm^3})} & {$C_{tot}$ (\si{\uJ\per\K})} & Source \\
\midrule 
    SiN & 203.6 &   1.0 &   0.2 & \cite{holmes_measurements_1998} \\ 
   SiO2 & 183.2 &   3.1 &   0.6 & \cite{zeller_thermal_1971,zink_specific_2004} \\ 
     Al &   0.8 & 196.8 &   0.2 & \cite{irwin_transition-edge_2005} \\ 
     Au & 154.4 & 161.3 &  24.9 & \cite{corak_atomic_1955} \\ 
\midrule 
  Total &       &       &  25.8 \\ 
\bottomrule
\end{tabular}
\end{table*}

\section{Detector Wafer Layout} \label{sec:ch5-layout}

\figref{fig:ch5-full-wafer} shows a photograph of the entire sub-array; the figure caption contains a detailed description of the features present on the sub-array.
In addition to the detectors, the wafer includes Au pads for attaching Au heat-sinking wire bonds and holes used for glueing the detector wafer to a Au-covered backshort wafer (see \sectionref{sec:ch5-focal-plane}).
Note that all detectors have heater resistors on their membranes, but only a subset have the resistors connected to wires that lead to bondpads.

The detectors are laid out on a square grid, chosen over a hexagonal close-packed layout because of the simpler wiring layout.
This choice does worsen \NETD.
A hexagonal close-packed array would allow the same number of detetors to be placed on the array, but with feedhorns \SI{7.5}{\percent} larger.
The larger feedhorn size could have improved the spill over effieciency from \num{0.52} to \num{0.544}, a \SI{4.6}{\percent} increase, leading to a \SI{4.6}{\percent} improvement in \NETD.

\begin{figure*}
\centering
\includegraphics{drawings/ch5-full-wafer.pdf}
\caption{
  Photograph of the 251-detector sub-array.
  Bondpads for connecting to the detectors run along the left and bottom sides.
  In the upper left, lower right, and upper right corners are Au pads for connecting Au wirebonds to allow the wafer to be heat-sunk to the rest of the \SI{1}{\K} stage.
  The upper left and lower right corners also contain small bondpads for connecting to detector heaters.
  While all detectors have heater resistors on their membrane, only the detectors along the left and lower edge have these resistors wired to bondpads.
  The 26 small holes spaced throughout the wafer are used to glue the detector wafer to a backshort wafer (see text).
  The three larger holes in the middle of the wafer are detector sites where the membrane was broken during fabrication.
  The large hole in the lower left, as well as the ``bulls-eye'' feature to its immediate upper right, were intended to be used as alignment features, although the alignment procedure actual used did not use these features.
  Photograph credit Dan Schmidt; full-resolution version available at \protect\url{http://www.flickr.com/photos/quantumsensors/8592792487}.
}
\label{fig:ch5-full-wafer}
\end{figure*}

\section{Predicted Noise} \label{sec:ch5-predicted-noise}

Using the targetted value of $G$ we can predict the total noise on the detectors as well as the expected \NETD\ in video images.
As discussed in \sectionref{sec:ch3-tes-noise}, intrinsic detector noise should be dominated by thermal fluctuation noise, given by
\begin{equation}
  S^2_{TFN} = 4 k_B T_0^2 G F(T_0, T_b).
\end{equation}
Using $G = \SI{3.8}{\nW\per\K}$, $T_0 = \SI{1.2}{\K}$, and $F = 0.83$ leads to $S_{TFN} = \Pnoisef{0.5}$.
This is \SI{60}{\percent} of \sectionref{sec:ch4-opt-eff}'s predicted photon noise of \Pnoisef{0.85}.
Summing the two noise sources in quadrature gives for the total noise (referred to power absorbed in the detector) $S_{tot} = \Pnoisef{1.0}$.

xxx need discussion of NETD, but want to write that section in ch2 first

\section{Focal Plane Assembly} \label{sec:ch5-focal-plane}

The \Imager's detector arrays are mounted on an Al platter that is thermally sunk to the cold head of the \He4-sorption fridge via a set of Cu ropes.
Mechanically the focal plane is attached to an Al frame via four Ti-6Al-4V ``spiders''.
The frame itself is bolted to the \SI{6}{\K} cold plate.
See \figref{fig:ch5-focal-plane-back}.

\begin{figure*}
\centering
\includegraphics{drawings/ch5-focal-plane-back.pdf}
\includegraphics{drawings/ch5-focal-plane-cryostat.pdf}
\caption{
  Photographs showing how the focal plane is attached to the \SI{6}{\K} Cold Plate.
  \textbf{Left}
  Back view of the focal plane.
  At each corner the focal plane is bolted to a Ti-6Al-4V ``spider'', which are clamped into Al blocks (e.g. rectangles \textbf{A}).
  The blocks are then bolted to an Al frame, on which the blocks and focal plane are resting in the photograph.
  The Al frame has Bock Black \cite{xxx} applied to in in an attempt to provide a surface on which stray infrared light will be absorbed other than the focal plane.
  Also visible are the 100-pin MDM connectors that the wires from the \MCE\ plug into (\textbf{B}).
  \textbf{Right}
  Photograph of the Al frame (\textbf{A}) and focal plane bolted onto the \SI{6}{\K} Cold Plate.
  The copper-colored area is the W1275 bandpass filter.
  This photograph was taken prior to the application of the Bock Black.
  Also visiable in the photograph are the end-points of the Cu ropes that connect the PTC 2nd-stage cold head to the \SI{90}{\K} plate (\textbf{B}).
}
\label{fig:ch5-focal-plane-back}
\end{figure*}

Each sub-array is glued to a \SI{275}{\um} thick Si wafer that has been micromachined to have the same outline as the sub-array and then covered with Au.
The glue was applied to the set of 26 holes in the detector wafer shown in \figref{fig:ch5-full-wafer}.
This wafer stack is then attached to an \SI{0.125}{\in} thick invar plate.
Invar was chosen because its thermal contraction upon cooling to \SI{4}{\K} and below is well-matched to Si; choosing a material that is poorly matched to Si could result in breaking the Si stack upon cooling.
The wafer stack is attached to the invar plate with a thin layer of Apiezon-N thermal grease.
Upon cooling to cryogenic temperatures Apiezon N grease solidifies, so that the detector wafer will not slip along the invar, and even at room temperature the grease is very thick so that if the back of the wafer stack is entirely covered with a layer of grease, the wafer will not slip.

The invar plates must be attached to the Al platter in a way that account for the differential thermal contraction between Al and Invar, while ensuring that the detectors are correctly aligned to the feedhorns.
This is done by aligning the invar, feedhorns, and Al platter to each other using stainless steel dowel pins.
Two pins align the feedhorns to the Al platter.
The dowel pins are inserted into the Al platter, with one matching hole and one matching slot in the feehorn array; a slot is used for one of the holes to avoid over-contraining the mechanical connection.
Two additional pins align each invar plate feedhorn array to the Al platter.
Again a matching hole and slot are present in the invar, but in this case the slot is neccesary not only avoid over-contraining the system but also to account for the differential thermal contraction between invar and Al.
The detector wafer is placed in the proper location on the invar platter using an alignment jig, and while the invar and grease have been warmed to \abt{\SI{30}{\celsius}}, a temperature at which the thermal grease becomes more viscous, making it easier to adjust the position of the wafer.
See \figref{fig:ch5-aligment-jig}.

\begin{figure*}
\centering
\includegraphics{drawings/ch5-alignment-jig.pdf}
\caption{
  Photograph of the alignment jig used to align the \Imager's sub-array to the invar plate.
  The invar plate and the metal bar are aligned to each other with dowel pins.
  The metal bar has conical cut-outs places so that when detectors are aligned with them, the wafer is aligned properly to the invar.
  In order to make it easy to move the detector wafer stack to the proper location, the entire assembly is warmed to \SI{30}{\celsius}; at this temperature the thermal grease thins.
}
\label{fig:ch5-alignment-jig}
\end{figure*}

Four 100-wire woven PhBr wire harness run from \SI{300}{\K} to the focal plane, heat sunk along the way at the \SI{90}{\K} and \SI{6}{\K} stages.
These wires carry the readout signals to and from the \MCE.
Once they reach the focal plane a circuit board routes the wires to each sub-array.
One 100-wire harness caried the row-address wires and the circuit board routes the wires to all mutliplexing chips in series (32 chips for the full array).
The other four 100-wire harnesses each carry \SQUID bias and feedback wires for a single sub-array.

To aid in routing of the wites, each sub-array has two ``wiring chips'' associated with it, visible in \figref{fig:ch5-focal-plane-platter}, and shown in schematic form in \figref{fig:ch5-wiring-chip-schematic}.
These chips aid in routing row-address and detector bias wires.
On top of each wiring chip are four multiplexing chips as well as four ``interface'' chips which contain Nyquist inductors and shunt resistors for the detectors.

\begin{figure*}
\centering
\includegraphics{drawings/ch5-wiring-chip-schematic.pdf}
\caption{
  Schematic showing close-up view of the wiring chips with multiplexing and interface chips on top.
}
\label{fig:ch5-wiring-chip-schematic}
\end{figure*}

\begin{figure*}
\centering
\includegraphics[width=6.00in]{images/ch5-focal-plane-platter.jpg}
\caption{
  Photograph of the focal plane platter while being assembled.
  The platter was machined out of Al-6061 and then Au-plated.
  The green circuit board routes the 500 PhBr wires running from room temperature to the multiplexing and interface chips.
}
\label{fig:ch5-focal-plane-platter}
\end{figure*}


\chapter{Sub-array Characterization}\label{c:det-array}

The detectors in the first 251-detector sub-array have not all been characterized to the same extent.
Partly this is due to manpower and time constraints.
Additionally, as described in \sectionref{sec:g-psat}, the saturation power of the detectors is \abt{2} times higher than expected, which means that only some detectors can be driven into the normal state using bias current alone --- even under optical load --- and then only at bath temperatures very close to $T_c$.
In particular, none of the detectors can be driven normal at the operating bath temperature of \SI{1100}{\mK}.
This means that we do not have current-vs-voltage (\IV) curves for any detectors at standard operating conditions, which limits the characterization than can be performed.

Seven detectors used in taking images have working heaters and no other problems.
Once calibrated, these heaters allow us to take full \IV\ curves at all bath temperatures, allowing measurements of the detector thermal conductance $G$, critical temperature $T_c$ and power-flow index $n$.
The heaters also allow a direct measurement of detector responsivity and time constants, and thus allow a measurement of detector noise referred to optical power absorbed on the detector.
These latter measurements have been carried out on four of the seven detectors.
Additionally, \Loop, $\beta_I$ and $R$ at standard operating conditions have been measured for all working detectors.
This chapter describes all these measurements, as well as measurements of shunt resistors and inductors for all detectors.
\tableref{tab:measurements} summarizes all of the measurements described in this chapter.
Sections discussing common mode noise, microphonic pickup and noise aliasing are also included.

Several measurements are taken at ``standard operating conditions'' (\SOC), which are the conditions under which the array is operated while acquiring video images.
These conditions are bath temperature $T_b = \SI{1100}{\mK}$ and detector bias $\mbox{\DAC} = \num{27000}$\footnote{The detector bias lines are controlled by digital-to-analog converter (\DAC) units as described in \sectionref{sec:det-readout}. Bias values in this chapter are in terms of \DAC\ units, where $\mbox{\DAC} = 2^{15} = 32768$ is the maximum bias that can be applied, and corresponds to $I_b \approx \SI{10}{\mA}$}.

When referring to individual detectors, I use the notation R$n$C$m$.
This notation refers to the detector read out on row $n$ of column $m$, where the indices are zero-based.

\begin{table*}[t]
\centering
\caption{Summary of measurements made on first 251-detector sub-array}
\label{tab:measurements}
\begin{tabular}{p{2.5in} l p{2.5in}}
\toprule
Measurement &  Section & Detectors Subset  \\
  \midrule
$R_{sh}$ and $L_{ny}$ & \ref{sec:shunt-nyquist} & All working detectors \\
Heater Resistors & \ref{sec:heater-r} & Seven detectors with working heaters on column 0 and 1 \\
$\tau$ & \ref{sec:tau-nat} & Four detectors with working heaters on column 0 \\
$G$, $T_c$, $n$, $P_{sat}$ & \ref{sec:g-psat} & Seven detectors with working heaters on column 0 and 1 \\
$\tau_{eff}$ and \DC\ responsivity & \ref{sec:teff-resp} & Seven detectors with working heaters on column 0 and 1 \\
\Loop, $\beta_I$, $R$ at a range of bias points & \ref{sec:bias-step} & Seven detectors with working heaters on column 0 and 1 \\
\Loop, $\beta_I$, $R$ at \SOC\ & \ref{sec:bias-step} & All working detectors\\
Detector noise at a range of bias points & \ref{sec:det-noise} & Four detectors with working heaters on column 0 and 1 \\
\bottomrule
\end{tabular}
\end{table*}

\section{Shunt Resistance Measurements}\label{sec:shunt-nyquist}

%
% All data from thesis/ana_rsh()
%

Our shunt resistors are located on interface chips that contain both shunt resistors and Nyquist inductors.
The design resistance of the shunts was \SI{180}{\uOhm}, and the design inductance was \SI{609}{\nH}.
Each chips contains 32 shunt resistors and 32 inductors.

To measure $R_{sh}$ and $L$ for these chips I took noise measurements using zero detector bias current at two different bath temperatures: 980~mK and 1160~mK.
At these bath temperatures and at zero detector bias the detectors are superconducting, so that measured noise is due to the shunt resistor, any parasitic resistance, and SQUID noise in the multiplexed readout system itself.
Data was collected at 3030.3~Hz, and 20 data acquisitions lasting 33 seconds were taken at each bath temperature.

A power spectrum was estimated for each detector for each data acquisition using \MATLAB's \texttt{pwelch} function, using a \FFT\ size of $2^{12}$.
Each resulting power spectrum was fit to a function of the form
\begin{eqnarray}\label{eqn:scnoise-fit}
	\frac{4 k_B T_b}{R_{sh}} \frac{1}{1 + (2 \pi f (L/R_{sh}))^2} + SQ,
\end{eqnarray}
where $k_B$ is Boltzmann's constant, $T_b$ is the bath temperature for the measurement, and $f$ is the frequency.
The shunt resistance $R_{sh}$, inductance $L$, and readout chain white noise level $SQ$ are the fit parameters.

\figref{fig:rsh-l-plots} shows histogram plots of the resulting \Rsh\ and $L$ values.
\Rsh\ has a mean of \SI{149}{\uOhm} with standard deviation \SI{6}{\uOhm}.
The values for \Rsh\ include any parasitic resistance in the circuit, but no evidence for significant parasitic resistance has ever been seen, so this parasitic resistance is assumed to be zero throughout this dissertation.

The value for $L$ includes the Nyquist inductance on the interface chip, the input inductance of the first-stage \SQUID\ of the multiplexed readout system, as well as any parasitic inductance in the circuit.
Using this approach it is not possible to extract the inductance of the Nyquist inductor itself, but this is not a problem because the total inductance is the relevant  quantity for understanding the behavior of the detector and its circuit.

$L$ has a mean of 568~nH with a standard deviation of 86~nH.
However, this mean includes two sets of clear outliers: all values for multiplexing row 4 are clustered around 200~nH, and all values for multiplexing row 25 are clustered around 440~nH.
The reason for these low inductances is not understood.
Excluding rows 4 and 25,  $L$ has a mean of 587~nH with a standard deviation of 44~nH.

The outlier $L$ values are more clearly visible in the lower left plot in \figref{fig:rsh-l-plots}, which also shows a small correlation between \Rsh\ and $L$.
It is not known whether this correlation is a real physical effect or an artifact of the measurement and curve fitting process.

\begin{figure*}
\includegraphics{drawings/ch6-rsh-l-plots.pdf}
\caption[Measurements of shunt resistors and Nyquist inductors]{
Plots summarizing results of measurements of shunts and Nyquist inductors.
\textbf{Upper Left} Histogram of shunt resistance \Rsh.
\textbf{Upper Right} Histogram of total inductance in circuit, which includes the interface chip Nyquist inductor, the inductance of the SQ1 input coil, and any parasitic inductance.
\textbf{Lower Left} Scatter plot showing all \Rsh\ and $L$ values. A correlation is apparent, the reason for which is not understood.
\textbf{Lower Right} Plot showing current noise power spectrum for a single data acquisition for \RCm{20}{6}, along with predicted power spectrum based on best fit to \eqnref{eqn:scnoise-fit} across all data acquisitions. The best fit values are $\Rsh=\SI{155}{\uOhm}$, $L = \SI{616}{nH}$, and \SQUID\ white noise level of \Inoise{1.2e-10}.}
\label{fig:rsh-l-plots}
\end{figure*}

\section{Measurement of Heater Resistors} \label{sec:heater-r}

Thirty-one detectors have heater resistors.
Twenty-three of these have the heater wire bonded to the heater bias line.
Of these 23, there are nine detectors which show no response to applied heater power.
Five of these nine are on the cut list (see \sectionref{sec:ch7-det-cuts}), so this is not surprising.
But four detectors can be biased into the transition and work well, but show no response to applied heater power (\RC{4}{7}, \RC{5}{7}, \RC{6}{6}, \RC{7}{6}).
The reasons for these detectors not showing a response to heater power are not fully understood.
This leaves 14 working detectors that show a response to applied heater power.

However, a short between one of the \TES\ bias lines and the heater bias line means that ramping the current bias for columns 6 and 7 also ramps the heater bias for all detectors.
This means that for the seven working heaters on columns 6 and 7, interpreting \IV\ curves is difficult because a different amount of heater power is applied at each detector bias value.
This means that for these seven detectors it is not possible to measure $G$ or to calibrate the power being applied by the resistors.

This leaves seven detectors on columns 0 and 1 with heaters for which good \IV\ curves can be taken, and $G$ measured.
The heaters on these seven detectors can be used to directly measure the detector responsivity, noise referred to input optical power, time constants and thermal conductance $G$.
But these measurements require knowing the heater resistance.

Following the procedure outlined in \sectionref{sec:ch3-iv-curve}, I took \IV\ curves at $T_b = \SI{1100}{\mK}$ using a range of heater biases.
\figref{fig:ch6-heater-r-plots} shows the results for \RCm{29}{1}.
The upper left plot shows the \TES\ \IV\ curves.
The upper right plot shows the same data, but transformed into \TES\ Joule power and \TES\ resistance.
As applied heater current decreases, the Joule power at the start of the transition decreases.
In the lower left, the Joule power at $0.99R_{n}$ is plotted vs applied heater current.
A fit to \eqnref{eqn:ch3-ptot-99Rn} is also plotted.
Finally, the lower right plot shows the $R$ vs $P_J$ plots after the heater power has been added to each curve.
This plot shows that the powers are equalized very high in the transition, where Joule power depends only on \TES\ resistance.
It also shows that this assumption breaks down deeper in the transition.

\begin{figure*}
\includegraphics{drawings/ch6-heater-r-plots.pdf}
\caption[Plots related to heater measurements]{
Plots related to heater measurements, for the case of \RCm{29}{1}.
\textbf{Upper Left} \IV\ curves. The \IV\ curves should turn vertical when the detector becomes fully superconducting at zero voltage, but these curves shown a non-infinite slope. The reason for this is that the readout system as configured for these \IV\ curves was unable keep up with the rapid change of current in the superconducting branch.
\textbf{Upper Right} Same data as in upper left plot, but represented in terms of \TES\ Joule power and resistance. As the bias current for the heaters is increased, the curves shift to the left.
\textbf{Lower Left} Measured $P_{J}$ vs heater current at $0.99R_n$, as well as the fit to \eqnref{eqn:ch3-ptot-99Rn}.
\textbf{Lower Right} Same plot as upper right, but the heater power based on $R_{htr} = \SI{23.6}{\ohm}$ has been added to each curve.
}
\label{fig:ch6-heater-r-plots}
\end{figure*}

\tableref{tab:basic-det-props} lists all measured heater resistors.
The seven heaters for columns 0 and 1 have a mean of \SI{23.1}{\ohm} with a standard deviation of \SI{1.2}{\ohm}.

\begin{table*}[t]
\centering
\caption[Measured detector properties]{
Measured detector properties.
$P_{opt} = \SI{150}{\pW}$ is assumed everywhere.
Uncertainties are 95 \% confidence intervals after marginalizing over other fit parameters, and do not include systematic uncertainties due to the unknown value of $P_{opt}$, uncertainty in the value of the shunt resistors, or possible errors in the calibration of the focal plane thermometer.
The value of $C$ is calculated using $C = \tau G$.
}
\label{tab:basic-det-props}
\begin{tabular}{c c c c c c c c}
\toprule
  Detector 
& \specialcell{$R_{htr}$ \\ (\si{\ohm})}
& \specialcell{$G$ \\ (\si{\nano\W\per\K})}
& $n$
& \specialcell{$T_c$ \\ (\si{\mK})} 
& \specialcell{$\tau$ \\ (\si{\ms})}
& \specialcell{$R_n$ \\ (\si{\mOhm})}
& \specialcell{$C$ \\ (\si{\pJ\per\K})} \\
\midrule
\RCm{29}{1} & 23.9 & 7.78 $\pm$ 0.08 & 3.55 $\pm$ 0.10 & 1209.0 $\pm$ 0.6 & 8.94 $\pm$ 0.1 & 4.39 & 70 $\pm$ 1.1 \\
\RCm{30}{1} & 23.5 & 7.73 $\pm$ 0.07 & 3.57 $\pm$ 0.09 & 1213.8 $\pm$ 0.6 & 8.82 $\pm$ 0.2 & 4.35 & 68 $\pm$ 1.4 \\
\RCm{31}{1} & 23.4 & 7.56 $\pm$ 0.10 & 3.67 $\pm$ 0.13 & 1215.4 $\pm$ 0.8 & 9.45 $\pm$ 0.1 & 4.32 & 71 $\pm$ 1.1 \\
\RCm{32}{1} & 23.0 & 6.89 $\pm$ 0.36 & 3.35 $\pm$ 0.50 & 1212.4 $\pm$ 3.2 & 10.22 $\pm$ 0.1 & 4.28 & 70 $\pm$ 3.7 \\
\RCm{29}{2} & 23.8 & 7.71 $\pm$ 0.08 & 3.58 $\pm$ 0.10 & 1213.6 $\pm$ 0.7 & 9.01 $\pm$ 0.1 & 4.36 & 69 $\pm$ 1.1 \\
\RCm{31}{2} & 20.4 & 6.35 $\pm$ 0.16 & 3.41 $\pm$ 0.24 & 1214.5 $\pm$ 1.6 & 9.51 $\pm$ 0.1 & 3.78 & 60 $\pm$ 1.6 \\
\RCm{32}{2} & 23.5 & 7.41 $\pm$ 0.25 & 3.70 $\pm$ 0.32 & 1215.8 $\pm$ 2.1 & 10.98 $\pm$ 0.1 & 4.38 & 81 $\pm$ 2.7 \\
\midrule
Mean & 23.1 & 7.35 & 3.55 & 1213.5 & 9.56 & 4.27 & 70 \\
\bottomrule
\end{tabular}
\end{table*}

\section{Measurement of Natural Time Constant $\tau$} \label{sec:tau-nat}

%% run thesis/ana_tau_nat.m for results

The approach described in \sectionref{sec:tau-nat-theory} was used to measure the natural time constant $\tau$.
The response to a step-function change in applied heater power was measured at bias points near the top of the transition at $T_{b} = \SI{1100}{\mK}$.
The response to multiple steps was averaged together prior to making a fit.
For each bias point the time constant $\tau_{meas}$ and change in \TES\ current $\delta I$ was obtained from curve fits to \eqnref{eqn:htr-step-resp-high-time}.
A fit was then performed to \eqnref{eqn:teff-from-tau}:
\begin{equation} \label{eqn:tau-nat-fit}
  \tau_{meas} = \tau - \tau \mathcal{K} (I_{bias} \delta I).
\end{equation}
In this fit, $\tau_{meas}$ is considered the dependent variable, the product $(I_{bias} \delta I)$ is the independent variable, and the variables that are fit for are $\tau$ and $\mathcal{K}$.

\figref{fig:tau-nat-plots} shows an example of fitting to \eqnref{eqn:tau-nat-fit}, and the measured values of $\tau$ are listed in \tableref{tab:basic-det-props}.

\begin{figure*}
  \centering
\includegraphics{drawings/ch6-tau-nat-plots.pdf}
\caption[Plot showing measurement of natural time constant $\tau$]{
  Plot showing measurement of natural time constant $\tau$ for \RCm{31}{1}. The fit is to \eqnref{eqn:tau-nat-fit}.
  The y-intercept at $I_{bias} \delta I = 0$ gives $\tau = \SI{9.45}{\ms}$.
} 
\label{fig:tau-nat-plots}
\end{figure*}

\section{Measurement of \textsc{TES} $G$} \label{sec:g-psat}

%
% thesis/ana_g() followed by thesis/genplot_ch6_g
%

With knowledge of the heater resistances, \IV\ curves can be taken over a wide range of bath temperatures, which enables a measurement of the \TES\ thermal conductance $G$, critical temperature $T_C$ and power-flow index $n$.
I took \IV\ curves at bath temperatures ranging from \SIrange{995}{1160}{\mK}, while adjusting the applied heater power so that each \IV\ curve had a clear normal branch.
Fits were performed against \eqnref{eqn:ch3-g-fit}:
\begin{equation}\label{eqn:g-fit}
P_{htr} + P_J + P_{opt}= \frac{G T_c}{n}\left(1 - \left(\frac{T_b}{T_c}\right)^n\right).
\end{equation}
The parameters to be fit to are $G$, $T_c$, and $n$.

A problem arises because the data described in this section were taken when the cryostat was open, so that $P_{opt}$ was non-zero, with an unknown value.
Because $P_{opt}$ is a simple additive constant, it is not possible to fit for this value unless another constraint is known, such as the value of $T_c$.
Unfortunately, the \TES\ circuit bias loop contains Al wire bonds, which have a $T_c$ close to that of the detectors themselves, making a direct measurement of the detector's $T_c$ difficult.

However, the value of $P_{opt}$ can be estimated in two different ways. 
First, the predicted optical load of \SI{180}{\pW} from \sectionref{sec:ch4-optical-design} can be used.
Second, optical load on the prototype detectors was estimated to be in the range \SIrange{135}{165}{\pW} using \IV curve measurements.
In this analysis I assume $P_{opt} = \SI{150}{\pW}$, while also showing how different assumptions change the values of  $G$, $T_c$, and $n$.

\tableref{tab:basic-det-props} lists the resulting values for $G$, $T_c$ and $n$.
\figref{fig:heater-g-plots} contains a plot of the data and fit for \RCm{31}{2}, as well as scatter plots showing measurements of $G$, $T_c$ and $n$.
\figref{fig:heater-g-popt-plots} shows the effect of different assumptions for $P_{opt}$ on measurements of $G$ and $T_c$.
The effects on $G$, $T_c$ and $n$ introduced by the unknown value of $P_{opt}$ are different.
$G$'s uncertainty from $P_{opt}$ is about the same size as the statistical uncertainty due to the fit, while $T_c$'s uncertainty from $P_{opt}$ is much larger than the statistical uncertainty.
$n$ shows no apparent trend with $P_{opt}$.
Specifically, a change in $P_{opt}$ from \SI{100}{\pW} to \SI{300}{\pW} increases $G$ by \SI{5.7}{\percent}, increases $T_c$ by \SI{2.2}{\percent}, and leaves $n$ unchanged.

\begin{figure*}
\includegraphics{drawings/ch6-g-plots.pdf}
\caption[$G$, $T_c$, and $n$ measurements]{
Plots summarizing results of $G$, $T_c$ and $n$ measurements for seven detectors with working heaters.
All error bars and ellipses are 95 \% confidence intervals for statistical error; any systematic error is not included.
\textbf{Left} Plot showing $P_{sat}$ vs $T_b$ for \RCm{31}{2}, assuming $P_{opt} = \SI{150}{\pW}$.
The red line shows the best fit to \eqnref{eqn:g-fit}.
The data covers 25 temperatures from \SIrange{995}{1160}{\mK}, and 11 different heater biases.
\textbf{Center} Scatter plot showing correlation between $G$ and $n$, as well as error ellipses showing covariance between the estimated $G$ and $n$ vales.
\textbf{Right} Scatter plot showing correlation between $G$ and $T_c$, as well as error ellipses showing covariance between the estimated $G$ and $T_c$ vales.
} 
\label{fig:heater-g-plots}
\end{figure*}

\begin{figure*}
\includegraphics{drawings/ch6-g-popt-plots.pdf}
\caption[Impact of different $P_{opt}$ assumptions]{
Plots showing effect of $P_{opt}$ assumptions on $G$ and $T_c$ measurements.
\textbf{Left} Plot showing variation of $G$ for \RCm{31}{2} vs assumed value of $P_{opt}$.
The statistical uncertainty in $G$ for this detector is approximately the same as the systematic uncertainty that results from the estimation of $P_{opt}$.
\textbf{Right} Plot showing variation of $T_c$ for \RCm{31}{2} vs assumed value of $P_{opt}$.
In this case the systematic uncertainty is larger than the statistical uncertainty, although the change is only \SI{2.2}{\percent} as $P_{opt}$ increases from \SI{100}{\pW} to \SI{300}{\pW}.
The value of $n$ shows no trend with $P_{opt}$.
} 
\label{fig:heater-g-popt-plots}
\end{figure*}

As discussed in \sectionref{sec:det-parm-choice}, the target $G$ value for these detectors was 3.7~nW/K.
The mean value for the seven measured detectors is \SI{7.35}{\nano\watt\per\kelvin}, \abt{2} times larger than the target.
The reason for this discrepancy is not known.
%thz5 high g - 67 um 
%thz5 high g - 80 um 
%thz4 proto - 40 um

\section{Direct Measurement of Detector Responsivity and $\tau_{eff}$} \label{sec:teff-resp}

%
% thesis/ana_heater_step() ... note that the file is created inside fit_heater_step()
%

Knowledge of $R_{htr}$ allows a direct measurement of the DC responsivity and $\tau_{eff}$ for the seven detectors with heaters.
Steps in heater bias current were applied to these detectors under \SOC.
The step size was made small so as to keep the detector response linear, and the response to many steps was averaged together to reduce noise.
The result was fit to \eqnref{eqn:htr-step-resp-high-time}:
\begin{equation} \label{eqn:ch6-heater-step-trans}
  \delta I(t) = - \delta P_{htr} s_I(0) (1 - e^{-t/\tau_{eff}}).
\end{equation}
\figref{fig:ch6-heater-step-trans} shows a sample fit to \eqnref{eqn:ch6-heater-step-trans}.
\tableref{tab:trans-det-props} lists the best-fit values of $s_I(0)$ and $\tau_{eff}$.

\begin{figure*}
\centering
\includegraphics{drawings/ch6-heater-step-trans.pdf}
\caption[Detector response to heater steps]{
Plot showing response of detector \RCm{30}{1} to step in applied heater power of \SI{1.41}{\pico\watt}.
Plots are for \RCm{30}{1} biased into \SOC.
The data averaged over 32 steps (16 up and 16 down), along with best fit to \eqnref{eqn:ch6-heater-step-trans}, are plotted.
The step in applied power begins at $t \approx \SI{0.6}{\ms}$, not $t = \SI{0}{\ms}$.
} 
\label{fig:ch6-heater-step-trans}
\end{figure*}

\begin{table*}[t]
\centering
\caption[Detector properties while biased into transition]{
Detector properties while biased into transition.
$P_{opt} = \SI{150}{\pW}$ is assumed everywhere.
%Uncertainties are 95 \% confidence intervals after marginalizing over other fit parameters, and do not include systematic uncertainties due to the unknown value of $P_{opt}$, uncertainty in the value of the shunt resistors, or possible errors in the calibration of the focal plane thermometer.
Values are for detectors under \SOC.
``N/A'' indicates a property that has not been measured for that detector.
}
\label{tab:trans-det-props}
\begin{tabular}{l l l l l l l l}
\toprule
Detector &  $s_I(0)$ (\si{\per\uV}) & $\tau_{eff}$ (\si{\ms}) & $R$ (\si{\mOhm}) & $R/R_n$ & \Loop & $\alpha$ & $\beta_I$ \\
\midrule
\RCm{29}{1} & 0.612 & 3.17 & 3.51 & 0.80 &  2.6 &  59 & 0.32 \\
\RCm{30}{1} & 0.691 & 2.44 & 3.34 & 0.76 &  4.0 &  90 & 0.45 \\
\RCm{31}{1} & 0.605 & 3.25 & 3.25 & 0.71 &  2.9 &  66 & 0.44 \\
\RCm{32}{1} & 0.687 & 2.81 & 2.72 & 0.62 &  8.9 & 155 & 1.98 \\
\RCm{29}{2} & 0.663 & 2.83 & N/A & N/A & N/A & N/A & N/A \\
\RCm{31}{2} & 0.731 & 2.44 & N/A & N/A & N/A & N/A & N/A \\
\RCm{32}{2} & 0.681 & 3.24 & N/A & N/A & N/A & N/A & N/A \\
\bottomrule
\end{tabular}
\end{table*}

\section{Bias Step Analysis: \Loop\ and $\beta_I$} \label{sec:bias-step}

As described in \sectionref{sec:lin-tes-eqn}, steps in the applied bias current can be used to measure \Loop, $\beta_I$ and $R$ for a detector.
In order to make this measurement, the data acquisition rate must be fast enough to track the fast electrical response of the \TES.
In addition, the servo roll-off must be either fast enough not to affect the electrical response, or the effect of the servo roll-off must be included in the fit.

I took measurements at \SI{15625}{\hertz}, which is fast enough to track the electrical response.
I found that the servo roll-off was too slow to be ignored for many detectors, so the function to be fit to is \eqnref{eqn:bias-step-resp} after being passed through a single pole lowpass filter with time constant $\tau_{servo}$; $\tau_{servo}$ is thus one of the parameters to be fit.
The data was taken at bias \DAC\ values of 25000, 26000, 27000, 28000, 29000, 30000, 31000 and 32000.

\figref{fig:ch6-bias-step-plots} and \figref{fig:ch6-bias-step-results} show the results of these measurements for the four detectors on column 0 with working heaters.
The response to many steps is averaged together.
The fits are generally good, but at some bias points a damped oscillatory response is present on top of the expected \eqnref{eqn:bias-step-resp} response.
The source of this is not understood; two possible explanations are the presence of an additional ``dangling'' heat capacity in the electrothermal circuit of \figref{fig:elec-thermal-circuit} \cite{hoevers_thermal_2000,zink_array-compatible_2006,maasilta_complex_2012}, or non-smooth structure in the detector's $R(T,I)$ curve.

\begin{figure*}
  \centering
\includegraphics{drawings/ch6-bias-step-plots.pdf}
\caption[Plots showing response of a detector to bias steps]{
  Plots showing response of a detector to bias steps.
\textbf{Left}
Response of \RCm{31}{1} to step in applied bias current, at a range of bias points.
In all cases there is a fast increase in the \TES\ current followed by a slow decay to the final current, which for these bias points is always less than the initial current.
This drop in current is a result of electrothermal feedback.
As the detector is biased deeper into the transition the decrease in current becomes larger, as a consequence of increasing loop gain and decreasing bias voltage; see \eqnref{eqn:si-full}.
\textbf{Upper Right}
Close-up view of initial stage of detector response.
Both the data and the best-fit curve to \eqnref{eqn:bias-step-resp} are shown, and the responses are offset vertically for clarity.
At some bias points a damped oscillatory response is present on top of the \eqnref{eqn:bias-step-resp} response; the source of this is not understood.
}
\label{fig:ch6-bias-step-plots}
\end{figure*}

\begin{figure*}
  \centering
\includegraphics{drawings/ch6-bias-step-results.pdf}
\caption[Results of bias step fits]{
  Plots showing results of fits for the four detectors tested at varying bias points in this section.
  The circled points are for \SOC.
}
\label{fig:ch6-bias-step-results}
\end{figure*}

Once values for \Loop, $\beta_I$ and $R$ are known, the \DC\ responsivity $s_I(0)$ and $\tau_{eff}$ can be calculated from \eqnref{eqn:si-full} and \eqnref{eqn:teff} respectively.
To check the accuracy of these calculations, I also measured the response of the detectors to steps in applied heater power at the same bias points for four of the seven detectors.
The results of these measurements, as well as the ratio of the calculated to measured values, are shown in \figref{fig:ch6-heater-step-pred-plots}.
The agreement between the calculated and measured values is good, indicating that response to detector bias steps can be used to predict $s_I(0)$ and $\tau_{eff}$ with good accuracy.
This is important because only a few detectors have working heaters, making direct measurements of $s_I(0)$ and $\tau_{eff}$ impossible for most detectors.

\begin{figure*}
  \centering
\includegraphics{drawings/ch6-heater-step-pred-plots.pdf}
\caption[$\tau_{eff}$ and $s_I(0)$ measurements]{
  Plots showing measurements of detector response times $\tau_{eff}$ and responsivity $s_I(0)$ for the four detectors of column 0 with working heaters.
  The circled points are for the \SOC.
\textbf{Upper Left}
Measurements of $\tau_{eff}$ for a range of bias points.
\textbf{Upper Right}
Measurements of $s_I(0)$ for a range of bias points.
\textbf{Lower Left}
Comparison of predicted and measured $\tau_{eff}$ for the same detectors.
\textbf{Lower Right}
Comparison of predicted and measured $s_I(0)$ for the same detectors.
}
\label{fig:ch6-heater-step-pred-plots}
\end{figure*}

Bias steps were also taken for all working detectors at \SOC.
\figref{fig:ch6-all-loop-plots} summarizes these measurements of \Loop, $\beta_I$, and $R$, both in terms of histograms for each parameter as well as scatter plots showing covariance between them.
\figref{fig:ch6-teff-si-pred} shows histograms of the resulting predictions for $\tau_{eff}$ and $s_I(0)$.
\SI{78}{\percent} of the working detectors have $\tau_{eff} < \SI{2}{ms}$.
%xxx comment on this speed compared to required.

\begin{figure*}
  \centering
\includegraphics{drawings/ch6-all-loop-plots.pdf}
\caption[Plots summarizing results of bias step measurements for all working detectors]{
  Plots summarizing results of bias step measurements for all working detectors.
All data taken at \SOC.
\textbf{Left Plots}
Histograms showing measured values of \Loop, $\beta_I$ and $R$.
\textbf{Right Plots}
Scatter plots showing how the three parameters \Loop, $\beta_I$ and $R$ correlate with each other.
Note that $R$ is plotted, not $R/R_n$. This is because $R_n$ is known only for those detectors on columns 0 and 1 with working heaters (see \sectionref{sec:heater-r}).
}
\label{fig:ch6-all-loop-plots}
\end{figure*}

\begin{figure*}
  \centering
\includegraphics{drawings/ch6-teff-si-pred.pdf}
\caption[Predictions for $\tau_{eff}$ and $s_I(0)$]{
  Plots showing distribution of predicted $\tau_{eff}$ and $s_I(0)$. The predictions use \eqnref{eqn:teff} and \eqnref{eqn:si-full}, with the values for $R$, \Loop and $\beta_I$ shown in \figref{fig:ch6-all-loop-plots}, and $R_{sh}$ values from \sectionref{sec:shunt-nyquist}. $R_p$ is assumed to be zero in all cases.
}
\label{fig:ch6-teff-si-pred}
\end{figure*}

\section{Common Mode Signal and $1/f$ noise} \label{sec:ch6-common}

% thesis/ana_common_mode() for all numbers

Our detectors have significant $1/f$ noise, with a \SI{3}{\decibel} knee of $\abt{\SI{0.7}{\hertz}}$.
Most of this noise is due to bath temperature fluctuations which are uncontrolled by the Cryocon \PID\ loop.
\figref{fig:ch6-cm-plots} contains plots summarizing this common-mode signal and $1/f$ noise.
The upper left plot shows raw 10 minute detector timestreams for 15 detectors.
The common mode signal is evident in these plots, and is much stronger than the white noise at frequencies of \SI{1}{\Hz} and slower.
The upper right plot shows the same detector timestreams after removal of the mean of all ``good'' timestreams for columns 0 and 1 (the only columns which were biased for this test).
The large reduction of $1/f$ noise is evident in this plot.
The lower left plot shows direct evidence for this via the current noise power spectral density both before and after subtracting the common mode.
Also plotted is the power spectral density after subtracting first the common mode and then the best-fit 4th order polynomial from the raw detector timestream.
Subtracting the 4th order polynomial does reduce noise at very low frequencies, but the effect is small.

The power spectral density plot has two important features.
First, a strong noise peak is located at \SI{1.411}{\Hz}.
This is caused by the \abt{\SI{1.4}{\Hz}} cycle of the \PTC; the physical mechanism could either be microphonic pickup of the vibrations caused by the \PTC\ cycle, or variation in bath temperature induced by the cycle.
This signal can be removed either through a common-mode subtraction scheme or a notch filter.
Second, the detector noise signal is unaffected by the common mode signal at frequencies faster than \SI{2}{\Hz}.
Because the frame rate of the video system is 6 FPS or faster, this indicates that the only impact of the strong common-mode noise signal on videos is the need to account for a time-varying detector offset.
Our approach to dealing with this offset is covered in \sectionref{sec:ch7-algo}.

The lower right plot in \figref{fig:ch6-cm-plots} shows the detector timestream for \RCm{29}{1}, translated into variation in bath temperature.
We can define a differential thermal conductance relative to changes in bath temperature $G_b$ via
\begin{equation}
  G_b \equiv \frac{dP_b}{d T_b} = G \left( \frac{T_b}{T} \right)^{n-1}.
\end{equation}
Then the equivalent bath temperature change for a given \TES\ current change will be given by
\begin{equation}
  \Delta T_b = \frac{\delta I}{s_I(0) G_b}.
\end{equation}
For this test the bath temperature was set to \SI{1100}{\mK}, so the implied temperature variations over several-minute timescales are a few parts in $10^{4}$.

\begin{figure*}
\includegraphics{drawings/ch6-cm-plots.pdf}
\caption[Plots summarizing common mode signal and $1/f$ noise]{
Plots summarizing common mode signal and $1/f$ noise.
\textbf{Upper Left}
Plot showing raw detector output for 15 detectors over a 10-minute data acquisition.
The data was acquired at \SI{312.5}{\Hz}, but only every 100th data point is plotted.
\textbf{Upper Right} 
The same data after removal of the common mode signal (as defined in the text).
\textbf{Lower Left} 
Current noise power spectral density for the raw data, the raw data minus the common mode (``No CM''), the raw data minus the common mode and the best-fit 4th-order polynomial (``No CM, Poly''), and the common mode itself (``CM'').
The strong noise peak at \SI{1.411}{\Hz} is due to the \PTC, as explained in the text.
\textbf{Lower Right} 
Raw timestream for \RCm{29}{1}, after conversion to an equivalent bath temperature variation, as described in the text.
}
\label{fig:ch6-cm-plots}
\end{figure*}

\section{Microphonic Pickup} \label{sec:ch6-microphonic-pickup}

% Run thesis/ana_ptc for results

\TES\ detectors can be prone to microphonic pickup.
The earliest version of the cryostat used for this project used a Gifford-McMahon (\GM) cryocooler, which vibrates the cryostat significantly more than a \PTC\ does.
The prototype detectors had significantly higher noise levels with the \GM\ cooler running than when it was off; in addition, the detector noise could be directly increased by striking the side of the cryostat with a soft mallet while the \GM\ cooler was turned off.
I interpreted this behavior as evidence for microphonic pickup, and as a result replaced the \GM\ cooler with a \PTC.

To check whether microphonic pickup was present for the production detectors and the \PTC, I took noise data both with the \PTC\ running and turned off.
In both cases the bath temperature was held steady at \SI{1100}{\mK} using the Cryocon temperature controller.
Common mode noise was removed and power spectra for each detector were calculated.
Then the excess noise for each detector was calculated as
\begin{equation}
  \sqrt{  \frac{ {\displaystyle \sum_{f \ge 6}} S^2_{I,\textsc{ptc}}(f) }
               { {\displaystyle \sum_{f \ge 6}} S^2_{I,\mbox{\tiny{No}}\textsc{ ptc}}(f) }}
\end{equation}
where $S^2_{I,\textsc{ptc}}$ and $S^2_{I,\mbox{\tiny{No}}\textsc{ ptc}}$ are the measured current noise power spectral densities with the \PTC\ turned on and off, expressed in units of \si{\A^2 \per \Hz}.

\figref{fig:ch6-ptc-plots} shows a histogram of this excess noise.
Most detectors have higher noise with the \PTC\ on, but the mean excess noise is only \SI{1}{\percent}.
The figure also contains a plot of the power spectral density for \RCm{29}{1} with the \PTC\ on and off.

\begin{figure*}
  \centering
\includegraphics{drawings/ch6-ptc-plots.pdf}
\caption[Plots showing impact of \PTC\ on noise]{
Plots showing impact of \PTC\ on noise.
\textbf{Left}
Histogram showing excess noise due to the \PTC, defined as ratio of total noise above \SI{6}{\Hz} (see text for precise definition).
More detectors have higher noise with the \PTC\ on than off, but the mean excess noise is only \SI{1}{\percent}.
\textbf{Right}
Current noise for \RCm{29}{1} with \PTC\ on and off, after subtracting common mode noise.
The noise below \SI{30}{\Hz} is 1.5--2.5 times higher with the \PTC\ on, but the total noise at the relevant frequencies of $f >= \SI{6}{\Hz}$ is only \SI{2.9}{\percent}.
}
\label{fig:ch6-ptc-plots}
\end{figure*}

\section{Noise Aliasing} \label{sec:ch6-aliasing}

As explained in \sectionref{sec:det-readout}, the \MCE\ takes data at \SI{15625}{\hertz}, but is only capable of sending data for the 251-detector sub-array to the readout computer at \SI{3125}{\hertz}.
This raises the question of how much noise is aliased from the \SIrange{3125}{15625}{\hertz} band to below \SI{3125}{\hertz}.
To check this I acquired data under \SOC\ for column 0 at the normal rate of \SI{3125}{\hertz} as well as at \SI{15625}{\hertz}.
Five data files were acquired for each case.
For each row in column 0 and for each data file a power spectrum was taken after subtracting the common mode signal, and the excess noise was calculated as
\begin{equation}
  \sqrt{  \frac{ {\displaystyle \sum_{6 \le f \le 1562.5}} S^2_{I,3125}(f) \Delta f }
               { {\displaystyle \sum_{6 \le f \le 1562.5}} S^2_{I,15625}(f) \Delta f }}
\end{equation}
where $S^2_{I,3125}$ and $S^2_{I,15625}$ are the measured current noise power spectral densities at the two different multiplexing rates, averaged over all 5 data files, expressed in units of \si{\A^2 \per \Hz}.
The sums are performed only over those frequency components in the range \SIrange{6}{1562.5}{\Hz}, and note that $\Delta f$ is different for the two sampling rates.

\figref{fig:ch6-noise-aliasing} summarizes the results.
The average excess noise due to aliasing is \SI{16}{\percent}, but some detectors have statistically significant higher levels of aliased noise; for example, \RCm{20}{1} has \SI{29}{\percent} excess noise.

The servo feedback parameter for these acquisitions was $I_{FB} = 50$, but when taking video images $I_{FB}$ is typically set to 10 or 20.
Because the slower servo feedback will roll off noise above \SI{3125}, the level of aliased noise will be lower under these configurations.

The \MCE\ can be configured to apply a 4-pole digital lowpass filter to the \SI{15625}{\Hz} data stream prior to sampling at \texttt{data\_rate} \cite{mce_team_digital_????}.
An appropriate choice of the cutoff frequency for this filter can greatly reduce aliased noise.
I have not yet implemented this filter, but plan to do so for future operation of the system.

\begin{figure*}
  \centering
\includegraphics{drawings/ch6-noise-aliasing.pdf}
\caption[Plots showing impact of noise aliasing]{
Plots showing impact of noise aliasing.
\textbf{Left}
Plot showing fractional excess noise (see text for definition) due to noise aliasing for all rows of column 0.
The error bars are for \SI{95}{\percent} confidence intervals, and the median excess noise is \SI{12}{\percent}.
\textbf{Right}
Sample power spectra at \SI{3125}{\hertz} and \SI{15625}{\hertz} for \RCm{5}{1}.
For this detector the excess noise is \SI{17}{\percent}.
}
\label{fig:ch6-noise-aliasing}
\end{figure*}

\section{Detector Noise} \label{sec:det-noise}

While taking the \Loop and $\beta_I$ measurements described in \sectionref{sec:bias-step}, I also took noise data for the four detectors with working heaters on column 0.
For each data acquisition the common mode signal was subtracted and the power spectra calculated using \MATLAB's \texttt{pwelch} function.
\figref{fig:ch6-trans-noise} shows the resulting power spectra, both in terms of the directly measured current noise $S^2_I$, as well as in terms of noise referred to power absorbed in the bolometer, $S^2_{P}$.
$S^2_{P}$ is calculated via
\begin{equation}
 S^2_{P} = S^2_I / s_I(0),
\end{equation}
where here the \DC\ detector responsivity $s_I(0)$ is calculated using the measurements described in \sectionref{sec:bias-step}.

Also plotted is the predicted detector noise level for each detector at each bias point, using a ``basic'' noise model.
This basic noise model uses the values of $\tau_{eff}$ and $s_I$ calculated in \sectionref{sec:bias-step}, and the values of $G$ and $T_c$ measured in \sectionref{sec:g-psat}, assuming \SI{150}{\pW} of optical power.
All three sources of detector noise from \sectionref{sec:ch3-tes-noise} are included.
The predicted photon noise level of \Pnoisef{0.85} from \sectionref{sec:ch5-predicted-noise} is also included.

Several features are visible in these plots.
First, \RCm{30}{1} shows several noise lines, the origin of which is not understood.
Second, for all four detectors the spread of noise levels in the \SIrange{1}{20}{\hertz} is much smaller when referred to power absorbed on the bolometer than in terms of current noise.
This is the expected behavior for a noise source that adds power directly to the bolometer.
When expressed in terms of power absorbed in the bolometer, such noise will be the same at all bias points.
But because detector responsivity changes with bias point, the noise level will be different when expressed in terms of current.
Third, the shape of the noise curves are roughly as expected, with noise roll-offs happening at approximately the correct frequencies.
Fourth, the measured noise levels are much higher than the predicted noise levels.
\RCm{29}{1} is 1.9 times higher, \RCm{30}{1} is 5.8 times higher (dominated by the noise spikes), \RCm{31}{1} is 1.7 times higher and \RCm{32}{1} is 1.6 times higher.

The reason for this high level of noise is not known.
Two possible culprits are a higher than expected level of photon noise or a much larger value of $G$ than was measured.
Both of these possibilities would affect the noise spectrum in the same way, so they can not be distinguished.
\figref{fig:ch6-trans-noise-model-plots} plots the measured noise spectrum for \RCm{31}{1} at \SOC, along with the basic noise model and a noise model that includes enough additional power noise to match the measured white noise level at low frequencies.
This could be due to a value of $G$ that in reality is 4.8 times higher than what was measured, or a photon noise level that is 1.9 times higher than predicted.

One problem with explaining the noise spectra this way is that the measured spectrum shows a shelf at \SIrange{100}{1000}{\hertz} which is not present in any of the models.
A second problem is that it is difficult to understand how the measured value of $G$ could be so far off.
Excess photon noise seems a more likely explanation, and could be due to \IR\ power leaking onto the detectors.
Even a small amount of \IR\ power can increase noise levels through the $\nu$ factor in \eqnref{eqn:photon-noise}.
It is also possible that a much higher level in-band optical power is falling on the detector than predicted.
This would both increase the photon noise directly, and increase the predicted thermal fluctuation noise because of the link between assumed optical power and my measurements of $G$ and $T_c$ as discussed in \sectionref{sec:g-psat}.

Another possible explanation is a poor calibration of the readout system, so that the ``real'' current noise is \abt{2} times lower than measured.
However, this would reduce the measured values of $R_{sh}$ and $L_{ny}$ by a factor of $2^2=4$.
This is unlikely, because the measured values are close to the design values, and these design values have been confirmed by measurements of other interface chips of the same design\footnote{John Appel, personal communication}.

I therefore conclude that the high level of detector noise is real, but I do not yet have a full explanation for the high values.

\begin{figure*}
  \centering
\includegraphics{drawings/ch6-trans-noise.pdf}
\caption[Detector noise measurements]{
Plots showing detector noise for the four detectors with working heaters on column 0.
The left column plots the directly measured current noise, after removing a common mode signal, at eight bias points from 25k to 32k.
The right column shows the same noise spectra, but referred to power absorbed in the bolometer.
For all four detectors, there is less spread in the low-frequency power noise than in the current noise, suggesting that the dominant source of noise at these frequencies deposits power on the bolometers.
This behavior is expected of either thermal fluctuation noise or photon noise.
Also plotted in the right column is the predicted noise spectrum, using parameters taken from \sectionref{sec:bias-step}, including \Pnoisef{0.85} of photon noise.
For all four detectors, the measured noise is higher than predicted by the noise model.
}
\label{fig:ch6-trans-noise}
\end{figure*}

\begin{figure*}
  \centering
\includegraphics{drawings/ch6-trans-noise-model-plots.pdf}
\caption[Measured and modeled noise]{
  Plot showing measured noise for \RCm{31}{1}, referred to power absorbed in bolometer, along with two noise models.
The red line is the basic detector noise model using measured values for all detector parameters as described in this chapter, including predicted \Pnoisef{0.85} of photon noise.
The black line is for a noise model that include enough excess ``power'' noise to match the measured white noise level at low frequencies.
This excess ``power'' noise could be due to a high level of photon noise (1.9 times higher than predicted) or a larger-than-measured value of $G$ (4.8 times higher than measured), or some combination of the two.
}
\label{fig:ch6-trans-noise-model-plots}
\end{figure*}

% I'm removing this stuff because after thinking about it some more,
% I'm not sure why point I'm really trying to make here. So drop it.
%
% Finally, \figref{fig:ch6-all-noise} contains a histogram of the total noise in each detector using three different values for the servo gain $I_{FB}$: 50, 20, and 10.
% The total noise is defined as
% \begin{equation} \label{eqn:ch6-tot-noise-defn}
%   \sqrt{ \sum_{f \ge \SI{6}{\hertz}} S^2_P(f_j) \Delta f}.
% \end{equation}
% Lower servo gains reduce the \SQUID\ and other out-of-band noise, so that the total noise decreases with decreasing $I_{FB}$.
% The detectors also become more tightly bunched in their noise performance at lower servo gain.

% \begin{figure*}
%   \centering
% \includegraphics{drawings/ch6-all-noise.pdf}
% \caption[Total in-band noise]{
% Histograms showing total in-band noise for all working detectors at three different servo gain values.
% The total noise deceases as servo gain is decreased, as a result of the servo loop reducing \SQUID\ and detector noise that is outside the bandwidth of the detector.
% Fewer outlier detectors also appear at low servo gain.
% The rightmost bins include all values great than \SI{100}{\fW}.}
% \label{fig:ch6-all-noise}
% \end{figure*}


\chapter{Imaging}\label{c:imaging}

% xxx consistently refer to flat vs normal videos

% xxx should update all plots with latest fudge factor of 1.098

This chapter describes the process of producing videos from detector timestreams, as well as measurements required to support that process.
I start with a brief description of which detectors are not used due to problems with their performance (\sectionref{sec:ch7-det-cuts}).
I then describe how we read out the position of the secondary mirror (\sectionref{sec:ch7-mirror-readout}), which controls where detectors are pointed at a given time, followed by a description of how we determine the focal distance, determine where each beam is pointing in the far-field, and the distance scale (\sectionref{sec:ch7-focus-distance}--\sectionref{sec:ch7-dist-scale}).
I also present measurements of the optical efficiency of the system (\sectionref{sec:ch7-opt-eff}) and a temperature scale calibration (\sectionref{sec:ch7-temp-scale}).
The algorithm used to produce video (and still) images is described in \sectionref{sec:ch7-algo}, and a discussion of the \NETD\ in the images is in \sectionref{sec:ch7-noise-model}.

\section{Detector Cuts} \label{sec:ch7-det-cuts}

Approximately \SI{16}{\percent} of the detectors in the first sub-array can not be used to generate images.
For some detectors, the membranes are broken.
Others appear intact upon visual inspection, but show no response to applied current even in the superconducting state.
Others work as expected while superconducting, but can not be biased so as to show a response to changes in optical power. 
And some are extremely noisy or consistently show other problems in the data stream.
\figref{fig:detector-cuts-wafer} and \figref{fig:detector-cuts-rc} contain plots summarizing this information graphically, organized by detector position on the wafer and by readout row/column respectively.

To determine which detectors show no response in the superconducting state, the temperature of the focal plane was set to \SI{975}{\mK}, well below the $T_c$ of the detectors.
The \TES\ bias current was ramped, and data was acquired while running the readout system open-loop.
As an example, \figref{fig:tes-bias-ramp-sc} shows the resulting data for rows 0 -- 4 of all columns.
Most row/column combinations show a response that maps out the $V$-$\Phi$ curve for the SQUID amplifier chain.
The row/column combinations that show no response indicate either a broken detector line, a broken SQUID on a multiplexing chip, broken wire bonds, or some other problem in the readout system.

Another group of detectors remain superconducting at the chosen bias point and operating temperature of \SI{1100}{\mK}.
This could be caused by an abnormally high $G$ and/or $T_c$ value, or by a short between the \TES\ leads after the shunt resistor.
\figref{fig:tes-bias-ramp-trans} shows the result of ramping the \TES\ bias current over a small range while running the readout system open-loop, and while the detectors are biased at \SOC.

\begin{figure*}
\centering
\includegraphics[width=\textwidth]{./images/1377812876_RCs_sq1ramptes_00.png}
\caption[\SQUID\ response to \TES\ bias ramp (superconducting)]{
  Plot showing response of SQUID amplifier chain to a ramp in the \TES\ bias current, while \TES\ is superconducting.
  Data is shown for rows 0--4 for all eight columns.
  \RC{0}{2}, \RC{0}{3}, \RC{1}{3}, \RC{1}{7} all show no response, only noise (note the change in vertical scale for these rows/columns).
  The vertical axis is in Analog-to-Digital-Converter units for the output of the \SQUID\ amplifier chain.
  The horizontal axis is the applied \TES\ bias current in \DAC\ units.
}
\label{fig:tes-bias-ramp-sc}
\end{figure*}

\begin{figure*}
\centering
\includegraphics[width=\textwidth]{./images/1378756510_RCs_sq1ramptes_00.png}
\caption[\SQUID\ response to \TES\ bias ramp (transition)]{Plot showing response of SQUID amplifier chain to ramp in \TES\ bias current, while \TES\ is biased into transition.
  The total change in applied bias current is the same as in \figref{fig:tes-bias-ramp-sc}.
  Data is shown for rows 0--4 for all eight columns.
  \RC{0}{2}, \RC{0}{3}, \RC{1}{3}, \RC{1}{7} all show no response, only noise (note the change in vertical scale for these rows/columns).
  \RC{0}{1}, \RC{2}{5} and \RC{3}{4} all respond as if they were still superconducting (see \figref{fig:tes-bias-ramp-sc}).
  For the other detectors, the much slower mapping of the $V$-$\phi$ curve indicates a much higher resistance in the \TES\ circuit loop, due to the \TES\ sitting in the transition to the normal state.
  The axis units are the same as in \figref{fig:tes-bias-ramp-sc}.
}
\label{fig:tes-bias-ramp-trans}
\end{figure*}

\begin{figure*}
\centering
\includegraphics{./drawings/ch7-detector-cuts-wafer.pdf}
\caption[Detector layout on wafer]{
Figure showing detector layout on the wafer, highlighting which detectors have problems and which are working.
Each detector is labeled (below) with its row/column.
The x and y position indices of the detectors on the wafer are also shown.
}
\label{fig:detector-cuts-wafer}
\end{figure*}

\begin{figure*}
\centering
\includegraphics{./drawings/ch7-detector-cuts-rc.pdf}
\caption[Detector to multiplexing row/column mapping]{
  Figure showing same information as \figref{fig:detector-cuts-wafer}, but organized in term of readout rows/columns. Each detector is labeled (below) with its position on the detector wafer.
  The row/column numbers are labeled on the left and top. Unused rows/columns as well as the rows/columns used to read out the position of the secondary mirror are also indicated. }
\label{fig:detector-cuts-rc}
\end{figure*}

\section{Readout of Mirror Position}\label{sec:ch7-mirror-readout}

The \Imager\ produces a time-ordered data stream (``timestream'') containing the output of each detector as a function of time.
In order to turn this timestream into a video, we must know where the optical system is pointing at all times.
This section describes how this pointing information is recorded in the timestream data by the \Imager.
It is also necessary to know where each detector is pointed relative to the optical boresight position; this relative detector pointing information is extracted from beam maps, as discussed in \sectionref{sec:ch7-beam-maps}.

The pointing of the optical system is determined by the positions of the two \BOSE\ actuators --- \DISP1 and \DISP2 --- that move the secondary mirror%
\footnote{See \sectionref{sec:ch4-optical-design}}.
The actuator control hardware provides two voltage signals which are proportional to the positions of the actuators.
This voltage signal is sent into the \Imager\ cryostat by a pair coaxial cables.
Inside the cryostat two \SI{1}{m} long Phosphor Bronze \AWG36 twisted pair wires carry the signal to the focal plane, where the signal is fed into the input coil of a 1st stage \SQUID.
Series resistors at room temperature (\SI{4.23}{\mega\ohm} for \DISP1, \SI{4.36}{\mega\ohm} for \DISP2) are used to reduce the maximum current flowing through the wires to \abt{\SI{2}{\uA}}, which is a value appropriate for the 1st stage \SQUID\ input.

This approach synchronizes the actuator position readout with the detector response, and allows both types of information to be read out by the same warm and cold electronics.

To convert the mirror output current to actuator displacement I configured the \BOSE\ actuators to move in sine-wave patterns with a frequency of \SI{0.1}{\Hz}.
I fit the amplitude of the mirror output current to a sine wave; the ratio of the command displacement amplitude to the mirror output current amplitude gives the desired conversion factor.
This procedure was carried out for eight different actuator displacement amplitudes ranging from \SIrange{0.25}{3.5}{\mm}.
The results are conversion factors of \SI{2.93}{\mm\per\uA} for \DISP1 and \SI{3.02}{\mm\per\uA} for \DISP2, with results for the different command amplitudes in excellent agreement.
% see cooldown39/ana_bose_factor.m

To convert actuator displacement to displacements in the far-field of the system, we use \eqnref{eqn:ch5-bose-to-x} and \eqnref{eqn:ch5-bose-to-y}, substituting $d = I \times \text{conversion factor}$:
\begin{equation}
\Delta x = F_d \frac{\sqrt{2}}{2} \left( I_{\text{DISP1}} \times \frac{\SI{2.93}{\mm}}{\SI{1}{\uA}} -
                              I_{\text{DISP2}} \times \frac{\SI{3.02}{\mm}}{\SI{1}{\uA}}  \right) \times
    \frac{\ang{0.276}} {\SI{1}{\mm}} \times
    \frac{\SI{19.33}{\cm}} {\ang{1}}
\end{equation}
\begin{equation}
\Delta y = F_d \frac{\sqrt{2}}{2} \left( I_{\text{DISP1}} \times \frac{\SI{2.93}{\mm}}{\SI{1}{\uA}} +
                              I_{\text{DISP2}} \times \frac{\SI{3.03}{\mm}}{\SI{1}{\uA}}  \right) \times
    \frac{\ang{0.276}} {\SI{1}{\mm}} \times
    \frac{\SI{19.33}{\cm}} {\ang{1}}
\end{equation}
Here I have also introduced a fit parameter $F_d$, which allows us to account for any error in manufacturing of the optics, errors in the \ZEMAX\ model, or errors in the actual vs command displacement of the \BOSE\ actuator.
The direct measurements of distances in the far-field described in \sectionref{sec:ch7-dist-scale} show that $F_d = 1.098$, and this value is used in all analysis for the remainder of this chapter.

% pivot 14.19 mm behind mirror vertex
% pivot 14mm in front on \BOSE\ attachment point

I also tested whether cross-talk appears between the actuator readout and the detectors.
To test this both actuators were moved in a \SI{6}{\Hz} sine-wave pattern over their maximum displacement range of $\pm\SI{3.5}{\mm}$, while the detectors were biased at \SOC.
Both actuators were moved at the same time, roughly \SI{135}{\degree} out of phase.
The level of cross-talk present can be quantified by performing a least-squares fit of each detector timestream $\vect{d}_{rc}$ to the model
\begin{equation}
	 \vect{d}_{rc} = A_1 \vect{d}_{\text{DISP1}} + A_2 \vect{d}_{\text{DISP2}}.
\end{equation}
Here $\vect{d}_{\text{DISP1}}$ and $\vect{d}_{\text{DISP2}}$ are the measured outputs for each actuator.

The fit values for $A_1$ and $A_2$ were clustered near zero, and were small compared to the noise in the detector timestream.
As a more stringent test, I calculated the cross-talk amplitudes $A_1$ and $A_2$ for each detector twice: once for the first half of the data acquisition and once for the second half.
If cross-talk is present to a statistically significant level, a scatter plot of the cross-talk amplitudes for the two halves of the data acquisition should show signs of correlation.
As can be seen in \figref{fig:ch7-bose-cross}, the points are clustered about the origin and no correlation is apparent.

\begin{figure*}
\centering
\includegraphics{drawings/ch7-bose-cross.pdf}
\caption[Plot showing cross-talk amplitudes]{
Plot showing cross-talk amplitudes.
The left plot is for \DISP1, the right for \DISP2.
Each actuator was moved +/- \SI{3.5}{\mm} at 6 Hz while the detectors were biased at \SOC.
The best-fit cross-talk amplitude for each actuator and detector were calculated for both the first half and the second half of the data acquisition, and these amplitudes are plotted against each other in these plots.
The lack of correlation in the scatter plots, as well as the clustering around the origin, indicate that any cross-talk present cannot be distinguished from noise in the detectors.
}
\label{fig:ch7-bose-cross}
\end{figure*}

\section{Focus Distance}\label{sec:ch7-focus-distance}

As described in \sectionref{sec:ch4-optical-design}, the \Imager\ is designed to focus at distances of \SIrange{16}{28}{\m}.
All results described in this chapter were with the \Imager\ configured to focus at 16~m.
To check the actual distance to the target focal plane, beam maps as described in \sectionref{sec:ch7-beam-maps} were performed with the black-body source located at different distances from the cryostat.
The focus distance was found to be \SI{17}{\m} in front of the vertex of the primary mirror.
\figref{fig:ch7-focus} shows beam maps for the same detector taken at \SI{17}{\m} and at \SI{15.8}{\m}.
At \SI{17}{\m} the black-body aperture is well defined and much warmer than its surroundings.
At \SI{15.8}{\m}, the black-body aperture is visible, but poorly defined with significant side-lobes.
The temperature of the aperture is much closer to the background than in the \SI{17}{\m} case.

The reasons for the difference between the measured focus position and the focus position predicted using \ZEMAX\ are not understood.
It is possible that the cryostat is located incorrectly relative to the mirrors, or that there were errors in the manufacturing or assembly of the optical components.

\begin{figure*}
\centering
\includegraphics{drawings/ch7-focus.pdf}
\caption[Focus distance plots]{
  Plots showing impact of observing objects not located at the far-field focal plane.
  In both cases the \SI{1030}{\celsius} black-body source with aperture diameter set to \SI{0.4}{\in} (\SI{1.0}{\cm}) was observed.
  The acquisitions were taken 2.5\,minutes apart, with the only change being the distance between the black-body and the primary mirror.
  \textbf{Left} Still image taken with the black-body located \SI{17}{\m} from the primary mirror.
  The aperture is clearly defined and \SI{80}{\K} warmer than its surroundings.
  % xxx need to comment on why 80 K is so much smaller than 1000 C
  \textbf{Right} Still image taken with the black-body located \SI{15.8}{\m} from the primary mirror.
  The black-body aperture is poorly defined with prominent side-lobe features, and a temperature no more than \SIrange{5}{10}{\K} warmer than the surroundings.
}
\label{fig:ch7-focus}
\end{figure*}

\section{Beam Maps} \label{sec:ch7-beam-maps}

As discussed in \sectionref{sec:ch4-feedhorn-design}, the \Imager\ feedhorns are predicted to have beams that are circularly symmetric and well-approximated by Gaussians with \FWHM\ of \SI{1.2}{\cm} at the target.
To verify these predictions beam maps were performed by raster scanning the beams over a stationary \SI{1030}{\celsius} black-body source.

The source used was an IR Labs IR-563/301\footnote{IRLabs, Inc. Tucson, AZ. USA} black-body.
This source reaches a maximum of \SI{1030}{\celsius} and has apertures ranging in size from \SIrange{0.0125}{0.6}{\in}.
Best results were achieved by covering an area around the black-body source with Al foil; this eliminated hot spots in the image due to the warmth of the housing of the black-body source itself.
% xxx find this picture and add it! \figref{xxx} shows a picture of the black-body source with Al foil mask.
The aperture diameter was set to \SI{0.2}{in} (\SI{0.51}{\cm}), which is smaller than the predicted beam and will have a minimal effect on the measured beam width.

The \Imager\ beams were raster scanned over the black-body source by moving one actuator at \SI{6}{\hertz} while the other actuator moved much more slowly at \SI{0.1}{\hertz}.
Scans were taken with the black-body in two different locations to ensure coverage of the entire sub-array.
At each black-body position two scans were taken with \DISP1 as the fast actuator and two with \DISP2 as the fast actuator, for a total of eight scans.

For each scan, the data stream for each detector was ``binned'' as described in \sectionref{sec:ch7-algo} to produce a beam map for each detector.
No common mode or polynomial was removed from the timestreams.
Actuator displacements were converted to distances in the far-field using the conversion factors discussed in \sectionref{sec:ch7-mirror-readout}.
The following 2-D elliptical Gaussian profile was then fit to each beam map:
\begin{multline}
  % Note the use of the null delimiters \left. and \right. to allow
  % the equation to be broken over lines.
  P(x,y) = O + A \, \text{exp} \left[  - \frac{1}{2} \left( \frac{ (x-x_0) \cos{\theta} + (y-y_0) \sin{\theta}}{\sigma_1} \right)^2 \right. \\ 
                              \left. - \frac{1}{2} \left( \frac{-(x-x_0) \sin{\theta} + (y-y_0) \cos{\theta}}{\sigma_2} \right)^2
                       \right] .
\end{multline}
Here $x$ and $y$ represent the position in the beam map, while $x_0$, $y_0$, $\sigma_1$, $\sigma_2$, $\theta$, $A$, and $O$ are the parameters to be fit.
$O$ represents an overall \DC\ offset in the map level.
Only the points within \SI{3}{\cm} of the map peak were included in the fit.
Beam maps at the edge of the scan were discarded, as were beam maps where the fitting routine performed poorly. % xxx quantify
The $\theta$, $\sigma_1$, and $\sigma_2$ parameters were all defined such that $\sigma_1 > \sigma_2$ and $0 < \theta < \ang{180}$.
The beam parameters across the eight scans were then averaged together to produce final beam maps.
\figref{fig:ch7-beam-summary} summarizes the final fit parameters for all the beams.
As can be seen from the width of the fit parameter histograms, the beam ellipticity and angle offset from the $x$-axis are statistically significant.

\figref{fig:ch7-all-beam-maps} shows the final beam maps.
The beams are elliptical, with a mean $\sigma_1 / \sigma_2 = 1.6$.
The mean beam angle is $\ang{70}$ counter-clockwise from the $x$-axis.
The beam size \FWHM\ is \SI{1.4}{\cm}, and is calculated from
\begin{equation}
  2 \sqrt{2 \ln{2}} \sqrt{\sigma_1 \sigma_2}
\end{equation}
where the prefactor $2 \sqrt{2 \ln{2}}$ converts from the Gaussian parameters $\sigma_{1,2}$ to the \FWHM\ of the Gaussian.

% xxx would be nice to add blurring as a possible explanation and make
% that quantitative.
In \sectionref{sec:ch4-feedhorn-design}, it was shown that the expected \FWHM\ beam width from this measurement is \SI{1.2}{\cm}, only about \SI{15}{\percent} smaller than observed.
The close match between the measured and predicted \FWHM\ beam widths is encouraging, but the beams should be circular rather than strongly elliptical as observed.
Another discrepancy between the measured and predicted beams is the distance between the beams.
The best-fit grid spacing between detector beams in the far field is \SI{2.0}{\cm}.
Using the plate scale extracted from \ZEMAX\ in \sectionref{sec:ch4-optical-design}, this is equivalent to a \SI{3.0}{\mm} detector spacing on the focal plane array, which is \SI{8.6}{\percent} larger than the design spacing of \SI{2.73}{\mm}.

The reasons for these discrepancies are not known.
Possible explanations for the error in plate scale are misalignment of the feedhorns with the primary and secondary mirrors, and errors in fabrication of the mirrors or feedhorns.
Either of these problems could also lead to elliptical beams, as could diffraction off the aperture stop located in the \SI{50}{\kelvin} radiation shield.
Another possible explanation for the elliptical beams could be misalignment between the detectors and the feedhorns.
As described in \sectionref{sec:ch4-feedhorn-design}, such misalignment leads to differential coupling between the two polarizations, and because the polarized beams from smooth-walled conical feedhorns are elliptical, misalignment could lead to elliptical measured beams.

To show the instantaneous field of view of the array compared to the area of the scanned image, \figref{fig:ch7-beams-on-image} shows the locations at which the beams are pointing when the \BOSE\ actuators are set to their zero positions.

Although the ellipticity of the beams is puzzling, the resolution is close to target, and the ellipticity is not a barrier to using the system to take video images.

\begin{figure*}
\centering
\includegraphics{drawings/ch7-beam-summary.pdf}
\caption[Beam fit parameters]{
  Plots summarizing final fit parameters for all beams.
  The histogram in the lower right shows that there is \abt{3} times more scatter in $\sigma_1$ than in $\sigma_2$. 
}
\label{fig:ch7-beam-summary}
\end{figure*}

\begin{figure*}
\centering
\includegraphics{drawings/ch7-all-beam-maps.pdf}
\caption[Beam maps]{
Plot showing final beam maps.
The ellipses represent the full-width-half-maximum (\FWHM) of the best-fit 2-D Gaussian for each beam. The beams are elliptical, with a mean $\sigma_1/\sigma_2$ of 1.6, mean beam angle to the $x$-axis of \ang{70}, and beam \FWHM\ of \SI{1.4}{\cm}.
The four grid locations in the extreme upper right corner, as well as the extreme lower left grid point, have no detectors and therefore no beams. All other missing beams are for cut detectors, as discussed in \sectionref{sec:ch7-det-cuts}.
The offset is relative to detector \RCm{18}{4}.
}
\label{fig:ch7-all-beam-maps}
\end{figure*}

\section{Direct Measurement of Distance Scale and Image Resolution} \label{sec:ch7-dist-scale}

To measure the factor $F_d$ described in \sectionref{sec:ch7-mirror-readout}, I placed a Styrofoam cooler filled with liquid Nitrogen (LN2) in the far-field of the system and scanned it with the system, using the data acquired to create still images.
A sheet of \ecco\footnote{Emerson \& Cuming Microwave Products, Inc.} was placed in the cooler to serve as a cold black surface for the system to observe.
Images were acquired of the cooler itself, as well as the cooler with Al foil strips taped to the outside.
Each strip was \abt{\SI{4}{\cm}} tall, and the strips were \SI{14.5}{\cm}, \SI{29}{\cm}, and \SI{47.2}{\cm} long.
The interior of the cooler is \SI{67.7}{\cm} wide.
For each image the \FWHM\ width of the strip --- or the cold space inside the cooler --- was calculated by taking a cut through the still image centered on the strip or cooler (see \figref{fig:ch7-check-dist}).

The result of these measurements is that $F_d = 1.098$.

\begin{figure*}
\centering
\includegraphics{drawings/ch7-check-dist.pdf}
\caption[Distance scale measurements]{
  Plots explaining measurement of distance scale.
  The left plot shows a close-up of the Styrofoam cooler filled with LN2, with a \SI{14.5}{\cm} by \SI{4}{\cm} strip taped to the outside.
  A cut of this image through the middle of strip (identified by the thin black line) is shown on the right.
  The \FWHM\ of this cut is shown as the red line.
  This measurement was repeated for two other Al foil strips, as well as the cooler without any Al foil strips, in order to establish the distance scale.
}
\label{fig:ch7-check-dist}
\end{figure*}

% Code: imaging/ana_reso_from_stills.m

Using the same cooler filled with LN2, a dime (diameter \SI{17.91}{\mm}) was taped to the outside of the cooler and a still image was taken.
\figref{fig:ch7-dime-map} shows the resulting still image, as well as a close-up of the are with the dime.
A 2-D Gaussian was fit to the resulting map.
The best-fit Gaussian has ellipticity 1.2 --- smaller than the individual beams --- and the \FWHM\ of the dime map is \SI{1.95}{\cm} --- larger than the beams.
Both differences are expected from convolution of the beam with the dime, which is not a point source.
The \FWHM\ of the dime map is \SI{2.0}{\cm}, which is \SI{15}{\percent} larger than the expected value of \SI{1.7}{\cm}. % blurring!
This serves as a rough confirmation of the beam size and locations as determined in \sectionref{sec:ch7-beam-maps}.

\begin{figure*}
\centering
\includegraphics{drawings/ch7-dime-map.pdf}
\caption[Image of LN2 cooler with dime]{
  Plots showing still image of Styrofoam cooler filled with LN2, with a dime (diameter \SI{1.791}{\cm}) taped to the outside.
  Red is warm, blue cold; the temperature scales are different in the two plots.
  \textbf{Left} The full map of the cooler. The white rectangle shows the area of the detail on the right.
  \textbf{Right} Detail of area within white rectangle: the dime.
  The black ellipse shows the \FWHM\ of the best-fit ellipse to the map. The ellipticity is \num{1.2}, the \FWHM\ of the principal axes are \SI{1.8}{\cm} and \SI{2.2}{\cm}, for an overall \FWHM\ of $\sqrt{1.8 \times 2.2} = \SI{2.0}{\cm}$.
}
\label{fig:ch7-dime-map}
\end{figure*}

\section{Optical Efficiency} \label{sec:ch7-opt-eff}

To measure the total optical efficiency of the system, \IV\ curves can be taken while a detector's beam is pointing at two known temperature loads.
As discussed in \sectionref{sec:ch3-iv-curve}, the difference in Joule power at $0.99 R_n$ gives the difference in total power dissipated in the bolometer.
In this case, the power difference will be caused by the different amount of power absorbed in the bolometer while observing the two temperatures.
If the two temperatures are $T_1$ and $T_2$, then the total optical efficiency $\eta_{tot}$ is given by
\begin{equation}
  \eta_{tot} = \frac{P_{iv,1} - P_{iv,2}}{P_{opt}(T_2) - P_{opt}(T_2)},
\end{equation}
where $P_{IV}$ is the Joule power in the detector at $0.99 R_n$ and $P_{opt}(T)$ is the optical power in both polarizations emitted by the source in a single spatial mode, given by \eqnref{eqn:ch4-power-per-mode}.

For a good measurement the load temperatures $T_1$ and $T_2$ should be as far apart as possible, and the detector's far-field beam should be filled by the load in each case.
I used the same ``\ecco\ submerged in LN2'' setup as described in \sectionref{sec:ch7-dist-scale} for the cold temperature load.
For the warm temperature load I used a sheet of \ecco\ backed by a thin sheet of Al.
``Room Temperature'' here is assumed to be \SI{295}{\kelvin} (\SI{71.3}{\fahrenheit}).

The temperature of LN2 is \SI{76}{\kelvin} in Boulder, CO where these measurements were made, but this does not mean that the temperature seen by the beams when pointed at the cooler is exactly \SI{76}{\kelvin}.
\begin{itemize}
  \item The beams are looking through the \SI{1.625}{\in} thick Styrofoam walls of the cooler, which may have some emission themselves.
  \item Although the \ecco\ sunk in LN2 is opaque at \SI{350}{\GHz}, it may reflect some amount of light from the surrounding room.
  \item It is possible that a small amount of water vapor could condense on the outside of the cooler, leading to further emission.
\end{itemize}
For the purposes of this measurement I assumed that the cold \ecco\ was black and that no water vapor was condensed on the cooler, which is consistent with a visual examination.
However, I do allow for a non-unity transmittance $\tau$ for the Styrofoam in the analysis below.

For each of eight detectors, four IV curves were taken under three conditions: pointing at the room-temperature \ecco\ (``ecco''), pointing at the cold \ecco\ in the cooler (``LN2''), and looking at the \ecco\ with the cooler's lid placed directly in front of the cooler (``lid'').
The lid is \SI{2}{\in} thick and the cooler's side is \SI{1.625}{\in} thick.
I assume that the transmittance takes the form $e^{-d \kappa}$, where $d$ is the thickness of the material and $\kappa$ is some constant which characterizes the attenuation length of \SI{350}{\GHz} light in Styrofoam.
Given this assumption, if the transmittance through the cooler's wall is $\tau$, then the transmittance through the lid is given by $\tau^{2/1.625} = \tau^{1.23}$.

The optical powers viewed under these three conditions is then given by (with $T_h = \SI{295}{\K}$ and $T_c = \SI{76}{\K}$)
\begin{align}
  P_{opt,ecco}  & = P_{opt}(T_h) \\
  \begin{split}
    P_{opt,LN2} & = \tau P_{opt}(T_c) + (1-\tau)P_{opt}(T_h) \\
              & = P_{opt}(T_h) - \tau (P_{opt}(T_h) - P_{opt}(T_c)),
  \end{split} \\
  \begin{split}
    P_{opt,lid} & = \tau^{1.23} P_{opt,LN2} + (1-\tau^{1.23})P_{opt}(T_h) \\
              & = P_{opt}(T_h) - \tau^{2.23}(P_{opt}(T_h) - P_{opt}(T_c)) .
  \end{split}
\end{align}
Here these optical powers include both polarizations and are referred to power emitted at the target.
The differences in optical power absorbed in the bolometer will then be
\begin{align}
  \Delta P_{b,ecco-LN2} & = \eta_{tot} \tau (P_{opt}(T_h) - P_{opt}(T_c)) \\
  \Delta P_{b,lid-LN2}  & = \eta_{tot} \tau (1-\tau^{1.23}) (P_{opt}(T_h) - P_{opt}(T_c)).
\end{align}
These last two equations can be solved for $\eta_{tot}$ and $\tau$. The results are
\begin{equation}
   \tau = \sqrt[1.23]{1 - \frac{\Delta P_{b,lid-LN2}}{\Delta P_{b,ecco-LN2}}},
\end{equation}
\begin{equation}
   \eta_{tot} = \frac{\Delta P_{b,ecco-LN2}}{ \tau ( P_{opt}(T_h) - P_{opt}(T_c) )} .
\end{equation}

The full set of three IV curves was repeated three times over the course of several hours.
The transmittance $\tau$ of the cooler wall was found to be \abt{\SI{90}{\percent}} and the optical efficiency $\eta_{tot}$ was \SI{13.6}{\percent}. 
\tableref{tab:opt-eff} gives the results of these measurements averaged over all detectors and all repetitions.

This optical efficiency is roughly half of the value predicted in \sectionref{sec:ch4-opt-eff}.
It is not clear where the optical power is being lost.
\chapterref{c:summary} describes future measurements that could help troubleshoot this problem.

\begin{table*}[t]
\centering
\caption[Results of optical efficiency measurements]{
Results of optical efficiency measurements.
The values are medians and the uncertainties give the \SIrange{25}{75}{\percent} range of measured values.
}
\label{tab:opt-eff}
\begin{tabular}{l l}
\toprule
Quantity &  Value \\
\midrule
$\eta_{tot}$ & $13.6 \pm  0.9$ \si{\percent} \\ 
$\tau$ & $90.6 \pm  4.8$ \si{\percent} \\ 
$P_{b,ecco-LN2}$   & $25.6 \pm  0.5$ \si{\pW} \\ 
$P_{b,lid-LN2}$ & $ 2.0 \pm  1.0$ \si{\pW} \\
\bottomrule
\end{tabular}
\end{table*}
 
\section{Temperature Scale Calibration} \label{sec:ch7-temp-scale}

The output of the readout system gives changes in current passing through the detectors.
In order to convert this current change to a temperature change we must first convert the current changes to changes in power absorbed in the bolometer by dividing by the power-to-current responsivity $s_I$.
To convert this to temperature changes in the far-field of the system, we must use the total optical efficiency $\eta_{tot}$, as well as the Rayleigh-Jeans limit of the optical power per spatial mode from a blackbody with a uniform temperature (see \sectionref{sec:ch4-opt-eff}).
The result is that the conversion from current to temperature is given by
\begin{equation} \label{eqn:ch7-I-to-T}
  \Delta T = \frac{\Delta I}{s_I(0) 2 k_B \eta_{tot} \Delta \nu}.
\end{equation}

Ideally this temperature scale would be measured directly by allowing the detectors to observe two known temperature loads and measuring the resulting change in current.
However, several difficulties arise in practice.
As the amount of optical power absorbed in a detector changes, the point occupied by the detector on its $R(T,I)$ surface changes, so that \Loop\, $\beta_I$, and the bias voltage $V_0$ all change.
This leads to a changing responsivity with optical load through \eqnref{eqn:si-full}, resulting in a non-linear detector response.
For small changes in load (such as we expect in images taken with the system) the changes will be small and can be ignored.
But in order to obtain an accurate measurement of the temperature scale, it is desirable to use a large change in temperature, such as the difference between room temperature and LN2.
For these larger temperature changes the change in responsivity may be significant. 

To check for detector non-linearity with varying optical load, I measured the detector responsivity directly using heaters both when the system was observing \ecco\ and when the cryostat window was covered with Al foil.
Covering the window with foil reduces the optical power reaching the detectors from outside the cryostat by \abt{\SI{38}{\pW}}.
For two of the detectors this led to an increase in responsivity of \abt{\SI{1}{\percent}}, while for the other two the increase was \abt{\SI{6.5}{\percent}}.
This makes it difficult to calibrate the temperature scale of the detector response more accurately than \SI{10}{\percent} when using a LN2 load.

Stable, uniform temperature distributions using stirred liquid water are available \cite{dietlein_aqueous_2008} and should be used if a more accurate temperature scale is desired in the future.

For this dissertation, however, I have not made these measurements.
Instead I have assumed that all detectors have the same optical efficiency as was measured for eight detectors in \sectionref{sec:ch7-opt-eff}, $\eta_{tot} = 0.136$.
For $s_I(0)$ I have used the predicted values at \SOC\ of \sectionref{sec:bias-step}, with the exception of three detectors (\RCm{25}{4}, \RCm{26}{2}, \RCm{26}{4}) for which the predicted $s_I(0)$ was clearly incorrect, as judged from inspecting detector timestreams and still images.
For these three detectors I have assumed a responsivity equal to the mean of all other predicted responsivities.

As a check on the accuracy of this temperature scale, the still image on the left of \figref{fig:ch7-dime-map} has \abt{\SI{180}{\K}} of contrast between the coldest section at the middle of the cooler and the warm area to the left of the cooler.
Using the estimate for transmittance $\tau$ of the cooler walls of \num{0.9}, the expected temperature differential is $0.9(295-76) = \SI{197}{\K}$.
Given the fact that responsivity will increase with optical loading, which leads to underestimates of temperature difference via \eqnref{eqn:ch7-I-to-T}, these numbers are in reasonable agreement, indicating that the temperature scale is accurate to within \abt{\SI{20}{\percent}}.

\section{Image Processing Algorithm} \label{sec:ch7-algo}

This section describes the algorithm used to turn raw detector timestreams into video images.
The algorithm currently used is simple, processing each video frame independently.

The algorithm's steps are as follows:
\begin{enumerate}
\item The  \MCE\ channels containing the actuator displacement (\RCm{32}{5} for \DISP1 and \RCm{32}{6} for \DISP2) are converted to displacements in the far-field as described in \sectionref{sec:ch7-mirror-readout}.
\item Determine full range of far-field displacements, accounting for all detectors. Define a \SI{1}{\cm} grid that covers this range in both $x$ and $y$ directions.
\item Each detector's output is multiplied by its ``current-to-temperature'' calibration factor (see \sectionref{sec:ch7-temp-scale}). At this point the units for all detector timestreams should be the same, but each detector timestream will have some unknown offset.
\item Divide detector timestreams into sections for each video frame based on \MCE\ readout rate and frequency at which the secondary mirror is rotating.
  If the mirror frequency is $f_m$ and the readout frequency is $f_{ro}$, then each video frame will cover $\lceil f_{ro} / f_{m} \rceil$ samples.
  \textbf{Example:} If the readout frequency is \SI{3125}{\Hz} and the mirror frequency is \SI{6}{\Hz}, then each video frame will cover $\lceil\frac{3125}{6}\rceil = \lceil 520.83 \rceil = 521$ samples.
\item For each video frame
  \begin{enumerate}
  \item Perform the following processing for each detector that is not on the cut list described in \sectionref{sec:ch7-det-cuts}:
    \begin{enumerate}
    \item If the detector's timestream shows evidence of glitches, do not include that detector's timestream for this video frame. The algorithm used to identify glitches is the following:
      \begin{enumerate}
      \item Calculate the differences $\Delta f_j = f_{j+1} - f_j$ between each consecutive value in the timestream.
      \item Calculate the standard deviation of the differences $\Delta f_j$.
      \item If the absolute value of any $\Delta f_j$ is greater than five times the standard deviation, identify this detector as having a glitch, so that it will be ignored for this video frame.
            On average this affects 3.2 detectors per frame.
            % xxx show example of glitch timestream? ... maybe show a big plot with all timestreams for a frame, highlighting those with glitches?
      \end{enumerate}
    \item Subtract the median value of this detector's timestream for this video frame.
          The advantages and disadvantages of this approach to removing detector offsets is discussed below.
    \item If the detector does not have a glitch, determine which image pixel the detector is pointing to at each point in time, using both the pointing position from the actuator readout described in \sectionref{sec:ch7-mirror-readout} and the beam pointing information from \sectionref{sec:ch7-beam-maps}.
    \item Add the detector's value to that pixel for the frame.
    \item Keep track of the total number of samples that have been added to each pixel.
    \end{enumerate}
  \item After each detector has been processed for the frame, divide the total value for each pixel by the number of samples across all detectors that have been used for that pixel.
  \end{enumerate}
  \item To improve contrast, clip the temperature scale to the range \SI{-3}{\K} to \SI{-3}{\K}.
        Any pixels in the image with no data were assigned a temperature of \SI{-3}{\K}.
  \item After all video frames have been processed, convert the resulting 3-dimensional array to a video using \MATLAB's \texttt{VideoWriter} object.
\end{enumerate}

\subsection{Discussion of Videos Taken With the \Imager}

\figref{fig:ch7-single-frames} shows four still images from a video that was processed according to this algorithm.
This video can be viewed online at \url{http://www.youtube.com/watch?v=ul_fd-KWH38}. % fix this use correct URL for this particular video!
They show the author with a ceramic knife hidden beneath a button-down cotton shirt.
The ceramic knife is visible as a dark area on the left of the shirt.
A darker line running down the center of the shirt is due to the extra layer of cotton backing the buttons; this additional cloth produces more attenuation of the warm light from the body, and so appears cooler.

The total temperature contrast in the images of \figref{fig:ch7-single-frames} is \SI{6}{\K}, but as described above higher and low values of temperature have been clipped to improve contrast.
In the raw images, the mean total contrast across all frames with the person present is \SI{8.2}{\K}.

To estimate the expected contrast we can take the following into account:
\begin{itemize}
  \item Human skin temperature is typically in the range \SIrange{33}{35}{\celsius} = \SIrange{306.2}{308.2}{\K} \cite{ramanathan_new_1964}.
  \item The emissivity of human skin has been reported as \num{0.65} at \SI{100}{\GHz} and \num{0.93} at \SI{500}{\GHz} \cite{appleby_standoff_2007}.
        Interpolating between these values leads to an emissivity of \num{0.825} at \SI{350}{\GHz}.
  \item The temperature the room in which theses images were taken was \abt{\SI{295}{\K}}.
\end{itemize}
If the coldest temperature in the images is given by the room temperature, then the total contrast in the image is predicted as
\begin{equation} \label{eqn:ch7-pred-contrast}
  0.825 \times \SI{307}{\K} + (1 - 0.825) \times \SI{295}{\K} - \SI{295}{\K} =  \SI{9.9}{\K}.
\end{equation}
This is \SI{1.7}{\K} larger than observed.
Given uncertainties in the actual skin temperature of the person in the image, the true emissivity of human skin at \SI{350}{\GHz}, the true background temperature and the true temperature of the illuminating light, the near agreement between the predicted and observed values gives confidence that the temperature scale is roughly correct.

\begin{figure*}
\centering
\includegraphics{drawings/ch7-single-frames.pdf}
\caption[Sample still images from a video taken with the \Imager]{
Sample still images from a video taken with the \Imager.
Time proceeds left-to-right and top-to-bottom.
The stills are 20 frames (\SI{3.33}{\s}) apart.
The person in the images is the author.
A ceramic knife hidden beneath a button-down cotton shirt is visible on the left of each image.
The darker line running down the center of the shirt is due to the extra layer of cotton backing the buttons; this additional cloth produces more attenuation of the warm light from the body.
As discussed in the text, the temperature range in these images was clipped to $\pm \SI{3}{\K}$ for better contrast.
The total contrast in the raw images is \SI{8.2}{\K}, about \SI{1.7}{\K} lower than expected.
See text for discussion.
}
\label{fig:ch7-single-frames}
\end{figure*}

\begin{figure*}
\centering
\includegraphics{drawings/ch7-beams-on-image.pdf}
\caption[Plot showing where beams are pointed in far-field]{
  Plot showing where beams are pointed in far-field.
  The background image is frame 25 from the video discussed in \sectionref{sec:ch7-algo}.
  The beam \FWHM\ ellipses are in blue.
}
\label{fig:ch7-beams-on-image}
\end{figure*}

\subsection{Discussion of Median-Subtraction}

As discussed in \sectionref{sec:det-readout}, the output of the readout system does not give the absolute current passing through each detector.
Instead it gives the change in current relative to some offset, which is not a-priori known.
My approach for dealing with this offset is, for each video frame, to subtract from each detector's timestream the median of that detector's timestream during that video frame.
Although crude, this approach does a good job of accounting for the offsets in the detector timestreams, as well as accounting for the common-mode drift described in \sectionref{sec:ch6-common}.
However, elliptical scanning artifacts are still visible in the images of \figref{fig:ch7-single-frames}.
These artifacts arise in part because the median-subtraction scheme will work ``perfectly'' only in the case where the median temperature viewed by each detector during the frame is the same.
If this is not true for a particular detector, the ellipse traced out by that detector in the image will appear warmer or cooler than its surroundings (depending on whether the distribution that it viewed was cooler or warmer than its surroundings), leading to elliptical scanning artifacts.

An additional consequence of this approach is that the median color in each video frame will be the same; i.e., a particular color in frame 1 may not represent the same temperature in frame 10.
This effect can be see in the early frames of the video available online.
In that video the early frames of the image are looking at the background of the lab, and only later does a person move into the frame.
As the person moves into the frame, some areas of the background become darker, not because they are changing to a lower temperature, but because they are now lower relative to the median temperature in the frame.
Although surprising when first seen, this effect does not prevent the system from being effective in detecting concealed weapons.

Other image processing algorithms could be used to deal better with the detector offsets.
One method would be to take a ``flat'' image of a uniform temperature distribution prior to taking videos, and use this image to normalize all detector offsets relative to each other.
However, an approach would still be needed to deal with common-mode drift, as well as other sources of low-frequency $1/f$ noise which could cause detector offsets to drift independently of each other.

Another method is to use the fact that the ellipse mapped out by each detector during a video frame overlaps with the ellipses of many other detectors.
It should be possible to take advantage of this to set the offsets for each detector correctly on a frame-by-frame basis.
It is possible that an approach like this could also be used to normalize the responsivity of all the detectors relative to each other.

An iterative approach is also possible, based on the fact that median-subtraction evidently comes close to properly removing detector offsets.
An algorithm could be developed which checks whether the points mapped out by each detector's ellipse are different than the neighboring points.
If the difference crosses some threshold, the offset could be adjusted and the frame processed with a new set of offsets.
This process could be repeated until a self-consistent image is created.

Other algorithmic approaches are certainly possible as well, and should be a focus of future work.

\section{Image Noise Model}\label{sec:ch7-noise-model}


The temperature scale established in \sectionref{sec:ch7-temp-scale} allows us to convert the measured detector white-noise level of \abt{\Pnoise{2.4e-15}} to a temperature noise via
\begin{equation}
  S_T = \frac{S_P(0)}{2 k_B \eta_{tot} \Delta \nu}.
\end{equation}
This results in a temperature noise level of \abt{\Tnoise{15e-3}}, referred to the temperature viewed in the far-field of the system.

To use this noise level to make a prediction for the \NETD\ in the image we can use the radiometer equation \cite{kraus_radio_1986}:
\begin{equation} \label{eqn:ch7-radiometer}
  \text{\NETD} = \frac{S_T}{\sqrt{2 t_{dwell}}},
\end{equation}
where $t_{dwell}$ is the total integration time including all detectors for each pixel in the image%
\footnote{%
In this equation the factor of 2 accounts for the fact that the noise power spectral density is defined so that the total variance in the signal is given by the integral of the power spectral density up to the one-half of the sampling frequency, i.e. up to the Nyquist frequency.
}.

All images in this chapter contain \abt{4030} pixels, each \SI{1}{\cm^2} in area%
\footnote{
  Although the images shown are all rectangular, due to the elliptical scanning pattern the actual area of the images that contains data is smaller than the rectangle.
}.
Each video frame lasts $\frac{1}{6}$\,\si{\s}, and there are \abt{210} good detectors contributing.
$t_{dwell}$ is thus given by
\begin{equation} \label{eqn:ch7-t-dwell}
  t_{dwell} = \frac{210 \times \frac{1}{6}\,\si{s}}{4030} = \SI{8.7}{\ms}.
\end{equation}
Plugging this into \eqnref{eqn:ch7-radiometer} gives an \NETD\ of \abt{\SI{115}{\mK}}.

Verifying this noise level in a video image is difficult because we do not know a-priori whether there are any regions in the image which have a flat temperature distribution.
However, if we place a sheet of \ecco\ directly in front of the cryostat window, where all detector beams are large and covering similar areas, then all detectors should be viewing close to the same temperature distribution.
If data is acquired in this state, while the secondary mirror is moving, and the resulting timestreams are run through the same software used to create ``real'' videos, then the standard deviation of the temperature in each frame should give a good estimate of the \NETD.

I carried out this procedure, creating a ``flat'' video with 19 frames.
The mean \NETD\ across the 19 frames is \SI{101}{\mK}.
\figref{fig:ch7-flat-netd} shows a histogram of the \NETD\ distribution from all 19 frames, a histogram showing the temperature offset distribution for the second frame (which has $\NETD\ = \SI{100}{\mK}$), the second frame itself, and one of the still images from \figref{fig:ch7-single-frames}.
Both still images use the same temperature-difference-to-gray-scale mapping.

Comparing the two frames visually, it is clear that \NETD\ in the true video frame is dominated not by detector noise, but by artifacts of the scan, visible as elliptical arcs in the image.
These artifacts are likely caused by the median-subtraction scheme of \sectionref{sec:ch7-algo} not properly accounting for the detector offsets.
Nevertheless, in areas of the video still where there appear to be few scan artifacts (such as the arm on the lower right), the level of noise in the video still appears comparable to the flat still, indicating that \SI{100}{\mK} is a reasonable estimate of the \NETD\ in the image caused by detector noise.

\begin{figure*}
\centering
\includegraphics{drawings/ch7-flat-netd.pdf}
\caption[Plots relating to measurement of \NETD\ in flat frame images]{
  Plots relating to measurement of \NETD\ in flat frame images.
  \textbf{Upper Left} Histogram of \NETD\ for each of 19 frames taken from a ``flat'' video image. This \NETD\ is defined as the standard deviation of the temperatures across all pixels which were visited by at least one detector. 
  \textbf{Upper Right} Histogram showing distribution of temperature offsets within the second video frame, which has an \NETD\ of \SI{100}{\mK}.
                       The far left and far right bins include outliers that extend all the way to \SI{-1}{\K} on the left and \SI{1.5}{\K} on the right.
                       Removing these outliers results in \NETD\ values that are roughly \SI{15}{\percent} smaller.
  \textbf{Lower Left} The second frame of the flat video.
  \textbf{Lower Right} Frame 25 of the video discussed is \sectionref{sec:ch7-algo}.
These two frames use the same temperature-offset-to-color mapping to aid in visual comparison.
}
\label{fig:ch7-flat-netd}
\end{figure*}

The agreement between the predicted \NETD\ of \SI{115}{\mK} and the observed flat image \NETD\ of \SI{100}{\mK} is an encouraging sign that we understand the behavior of the system.
However, there are several ways in which the noise modeling used in this section simplifies matters:
\begin{itemize}
\item The white noise level for each detector is different.
\item The total integration time per pixel is not uniform.
      Some pixels end up receiving more detector samples than others, as shown in \figref{fig:ch7-flat-frame-cnt}.
\item The detector noise is not white.
      \eqnref{eqn:ch7-radiometer} is strictly only true for a noise spectrum that is white, while the detector noise spectrums have roll-offs due to the detector time constant $\tau_{eff}$, \SQUID\ noise, a roll-off due to the \SQUID\ servo loop, and other features.
      Because the noise roll-offs reduce the variance of the detector timestream, this should tend to reduce noise in the map.
      However, the roll-offs also mean that consecutive samples in a detector timestream are correlated, so that averaging them will not reduce noise by the full ``square-root of the number of samples'' factor that is appropriate for uncorrelated noise.
\end{itemize}
A more careful analysis of these factors would be useful, but has not yet been performed.

\begin{figure*}
\centering
\includegraphics{drawings/ch7-flat-frame-cnt.pdf}
\caption[Samples per pixel in image]{
  Distribution of samples per pixel for flat field image shown in \figref{fig:ch7-flat-netd}.
  Different video frames have slightly different distributions, but the overall features are all similar to what is shown here.
  The bins in the histogram on the right are one sample wide, so that it can be seen that a handful of pixels around the outer edge receive only one sample.
}
\label{fig:ch7-flat-frame-cnt}
\end{figure*}

% xxx could a partial explanation for optical efficiency be the
% Piesiewicz alpha = 0.11 for HDPE? Would kill efficiency by factor of
% 0.88 relative to Lamb.



\include{ch8-summary}

% Now the bibliography. It should be single-spaced.
\SingleSpacing
\printbibliography

\end{document}

%%% Local Variables:
%%% mode: latex
%%% TeX-master: t
%%% End:
