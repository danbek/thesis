% !TEX TS-program = pdflatex
% !TEX encoding = UTF-8 Unicode

% Example of the Memoir class, an alternative to the default LaTeX classes such as article and book, with many added features built into the class itself.

%\documentclass[12pt]{article} % for a long document
%\usepackage{setspace}

\documentclass[12pt,article]{memoir} % for a short document
\usepackage[utf8]{inputenc} % set input encoding to utf8

\usepackage{enumitem}
\usepackage{graphicx}
\usepackage{booktabs}

\setlist{nolistsep} % or \setlist{noitemsep} to leave space around whole list


% My macros
\newcommand{\figref}[1]{Figure~\ref{#1}}
\newcommand{\tableref}[1]{Table~\ref{#1}}

\title{A 350 GHz Video Imaging System}
\author{Dan Becker}
%\date{} % Delete this line to display the current date

%%% BEGIN DOCUMENT
\begin{document}

\maketitle

%%%\begin{abstract}
%%%Passive millimeter-wavelength video imaging systems hold promise for detection of security threats at a distance, such as including suicide bomb belts and maritime threats in fog.
%%%Achieving optimal noise and optical performance for these system requires large numbers of cryogenic millimeter-wavelength radiation detectors. Large-format arrays of superconducting Transition Edge Sensor (TES) bolometers have been proven to meet requirement for both noise and number of detectors.
%%%We are developing a video- rate millimeter-wavelength imaging system using 1004 TES bolometers as detectors.
%%%This demonstration system detects is intended to have photon-noise-limited performance, and will be used to investigate phenomenology of passive millimeter-wavelength video images, with the goal of identifying what performance tradeoffs can be made when building a deployable system.
%%%It observes light in a 10\% band centered at 350 GHz, and is designed to take video images at distances ranging from 16 m to 28 m.
%%%When operating at 16 m, the resolution is 1 cm over a 1 m by 1 m field of view.
%%%The system is predicted to take video images with a noise equivalent temperature difference (NETD) of 100 mK at 20 frames per second.
%%%This thesis describes the design and implementation of this system, as well as imaging results from the first 251-detector subarray to be installed.
%%%\end{abstract}

%\tableofcontents* % the asterisk means that the contents itself isn't put into the ToC

%%%(211 words, 350 allowed)

\chapter{Introduction}\label{c:intro}

\chapter{System Specifications, Challenges and Solutions}\label{c:specs}

\chapter{TES Bolometer Theory}\label{c:tes}

\chapter{System Design Overview}\label{c:sys-design}

\section{Cryostat Design}\label{s-cryo-design}

The cryostat for the 350~GHz Imager was designed with the goals of simplicity, reliability and turn-key automated operation.
Highly reliable and easy-to-use cryogen-free mechanical cryocoolers are available from many vendors, but these cryocoolers are seldom capable of reaching temperatures below 2.5~K.
To reach sub-Kelvin temperatures a second refrigeration stage is required, which in our case is a He-4 sorption refrigerator.
The He-4 sorption refrigerator is based on a proven design and its use can easily be automated.
Three temperature stages within the cryostat are provided in order to provide intercepts for heatsinking wiring and other objects that are thermal connected to room temperature. 
The result is a cryogen-free cryogenic system that can be controlled remotely and has run for over 1000 (xxx) hours.

The cryostat itself was designed and built by Precision Cryogenics (xxx give addr and other info?) to specifications provided by the 350~GHz Imager team.
\figref{fig:cryo-cutaway} shows a cutaway view of the cryostat, and \tableref{tab:temp-optical-load} lists the temperatures typically reached by different parts of the cryostate during operation when the cryostat is open optically.
The cryostat has two main parts: a cylinder containing both the PTC and the He4 sorption refrigerator, and a box located at the bottom of the cylinder which contains temperature intercept plates and the focal plane.
There are three temperature stages, the ``90~K'' Cold Plate, the ``4~K'' Cold Plate, and the Focal Plane.
The PTC 1st stage is connected to the ``90~K'' Cold Plate by a tube of Al xxx and a set of xxx copper braids.
The combination of this long thermal path with the high heat load on the optical filters sunk to the ``90~K'' stage explains the 45~K temperature differential between the ``90~K`` cold plate and the PTC 1st stage.
The PTC 2nd stage is connected to the ``4 K'' Cold Plate by a large (xxx by xxx) cylinder of xxx Cu\footnote{This cryostat was originally designed to work with a different cryocooler. The PTC currently installed had a shorter distance between the 1st and 2nd stages, neccesitating the Cu cylinder to take up this extra space}, followed by tube of OFHC (xxx check) Cu followed by a set of xxx copper braids.
The Cu tube is broken into two halves, and the condensation plate (see below) of the sorption fridge is clamped between these two halves. The ``90 K'' Cold Plate is stood off from the cryostat vacuum jacket by four carbon-fiber standoffs. (xxx check with Bob S). The ``4~K'' Cold Plate stands off from the ``90 K'' Cold Plate by 12 (xxx check) supports made of G-10.

The first two temperature intercept stages are provided by a Cryomech PT407 Pulse Tube Cryorefrigerator (xxx reference company info).
The PT407 has two cooling stages.
The first stage has 25~W of cooling power at 55~K while the second stage has 0.7~W at 4.2 K.
xxx - sentence about how PTCs work.
Our PT407 uses a remote motor, so that the cold head attached to the cryostat has no moving parts, minimizing vibration of the cryostat.
Vibration of the cryostat can lead to microphonic pickup either directly in the detectors themselves or in the readout circuitry, leading to much higher detector noise.


\begin{figure}[h]
\centering
\includegraphics[width=0.6\textwidth]{images/cryostat-cutaway.png}
\caption{Cutaway view of the 350 GHz Imager. xxx should probably label things here.}
\label{fig:cryo-cutaway}
\end{figure}

\begin{table}
\centering
\caption{Temperatures Reached Under Optical Load xxx check all when temps have settled xxx should I add temps when closed optically?}
\label{tab:temp-optical-load}
\begin{tabular}{l r}
\toprule
Temperature Stage &  Temperature (K)\\
\midrule
PTC 1st Stage Cold Head 			& 52 \\
PTC 2st Stage Cold Head 			& 3.6 \\
Cryostat ``50 K'' Cold Plate 		& 96 \\
Cryostat ``4 K'' Cold Plate 			& 6.4 \\
Sorption Fridge Condensation Plate 	& 3.8 \\
Focal Plane 						& 0.970 \\
\bottomrule
\end{tabular}
\end{table}

\chapter{Detector Design}\label{c:det-design}

\chapter{Focal Plane Design}\label{c:fp-design}

\chapter{Detector Characterization}\label{c:det-char}

\chapter{Imaging}\label{c:imaging}

\chapter{Vocab}

Placeholder for terms

\begin{description}
\item[``90~K'' Cold Plate] 
\item[``4~K'' Cold Plate] 
\item[Sorption Fridge] 
\item[350~GHz Imager] 

\end{description}

\end{document}
