\chapter{Detector Design}\label{c:det-design}

This chapter describes the design and choice of parameters for the \TES\ bolometers that are used in the \Imager.

\section{Parameter Choice For Our Bolometers} \label{sec:det-parm-choice}

The primary parameters to be chosen when designing a \TES\ bolometer are the superconducting critical temperature $T_c$, the thermal conductance $G$, and the \TES\ island heat capactiy $C$.
These parameters are interrelated, and so cannot be chosen entirely independently of each other.
Some of the factors to consider are:
\begin{itemize}
  \item Detector noise scales with $\sqrt{T_c^2 G}$, so that lower values of $G$ and $T_c$ are better
  \item The saturation power of the \TES\ detector scales roughly like $G T_c$, so that if $G$ and $T_c$ are too small, the optical power falling on the detector will raise the temperature of the membrane above $T_c$, causing the device to not work.
    % do i talk about saturation power in ch 3? if not, I should!
  \item $T_c$ must be chosen to be higher than the acheivable bath temperature, and the bath temperature also affects the saturation power.
  \item The targetted detector time constant $\tau_{eff}$ depends not only on the detector natural time constant $\tau = G / C$ but also on the values of \Loop and $\beta_I$ at the chosen bias point.
        The loop gain depends primarily on the detector $\alpha$, but also on the saturation power.
\end{itemize}

The following subsections outline the choice of $T_b$, $T_c$, $G$ and $C$ for the detectors in the first 251-detector sub-array.

xxx say something about no attempt to shape transition - we get whatever alpha we get. no need to slow down trans, which is what normal metal bars do.

\subsection{Choice of $T_b$ and $T_c$}

The relationship between detector noise and saturation power can be examined in more detail.
From \eqnref{eqn:ch3-g-fit} we have
\begin{equation} \label{eqn:ch5-psat}
P_{sat} = \frac{G T_c}{n}\left(1 - \left(\frac{T_b}{T_c}\right)^n\right).
\end{equation}
This can be solved for $G$ and substituted into the expression for \TES\ thermal fluctuation noise in \tableref{tab:tes-noise}, leading to
\begin{equation} \label{eqn:ch5-tes-noise}
S^2_{TFN} = \frac{n F(T_0, T_b) T_0 / T_b}{1-(T_b/T_c)^n} 4 k_B T_b P_{sat}.
\end{equation}
The \TES\ temperature $T_0$ only appears in the pre-factor, which depends only on the power-flow index $n$ and the ratio $T_b/T_0$.
This means that for fixed $P_{sat}$ and $T_b$, the ratio $T_c/T_b$ that gives the lowest detector noise depends only on $n$ and the form of $F$.
For values of $n$ in the range 3--4, this optimal ratio is $T_c \approx 1.8 T_b$, while the pre-factor itself is \abt{3.7}.

As discussed in \sectionref{sec:ch4-opt-eff}, the predicted loading on the \Imager's detectors is \SI{180}{\pW} and the photon noise from this load is \Pnoisef{0.85}.
Choosing a safety factor of 3 so that $P_{sat} = 3 \times \SI{200}{\pW}$, and targeting detector noise equal to \SI{50}{\percent} of the photon (so that total noise is a factor of $\sqrt{1.5} = 1.22$ higher than photon noise) we find that in order for the detector noise to be below the predicted photon noise we require
\begin{equation}
  T_b < \frac{1}{3.7} \frac{\NEPph^2}{4 k_B P_{sat}} =
        \frac{1}{3.7} \frac{0.5 \times (\num{0.85e-15})^2}{4 \times \SI{1.38e-23}{\J\per\K} \times 3 \times \SI{180}{\pW}} = 
        \SI{3.6}{\K}
\end{equation}
It would seem that we should be able to run the system off of a Pulse Tube Cooler.

However, this leaves very little margin for error in the design and implementation of the system, so for the \Imager\ we chose to use a \He4-sorption fridge to set the bath temperature.

The \Imager's base bath temperature under optical load is \SI{970} (\see \sectionref{xxx}).
The initial hopes for performance of the \He4-sorption fridge were that it's base temperature would be ~\SI{650}{\mK}, implying an ideal $T_c$ of \abt{\SI{1.2}{\K}}.
This is a convenient $T_c$ because it is the critical temperature of elemental Al \cite{xxx}, so Al was chosen as the \TES\ material.

In practice, it was discovered during testing of Al prototype detectors that the base temperature of the system was \SI{950}{\mK} under optical load, so that a higher $T_c$ could lead to better noise performance.
However, in order to change as little as possible between the prototype detectors and the first sub-array, I decided to continue using Al.

\subsection{Choice of $G$}

\tableref{tab:ch5-proto-parms} lists the measured properties and parameters of the prototype detectors.
These detectors had $P_{sat}$ at $T_b = \SI{970}{\mK}$, 5.1 times higher than the predicted optical load and 6.1 times the measured optical load.
A safety factor of 5--6 is overly conservative, so for the sub-array I decided to target a $G$ value of \SI{3.7}{\nW\per\K}, for a safety factor of 3.8 -- 4.5.

\begin{table*}
\centering
\caption{
  Measured Properties of Prototype Detectors.
  The methods and procedures used to measure these properties were the same as described for the sub-array in \chapterref{c:det-array}.
} 
\label{tab:ch5-proto-parms}
\begin{tabular}{l c}
\toprule
  Detector Property &  {Value} \\
\midrule
  $T_c$                 & \SI{1.2}{\K} \\
  $R_n$                 & \SI{3.4}{\mOhm} \\
  $n$                   & 3.9 \\
  $G$                   & \SI{5}{\nW\per\K} \\
  $\tau$                & \SI{12}{\ms} \\
  $\tau_{eff} (typical)$ & \SI{4}{\ms} \\
  $C = G \tau $         & \SI{60}{\pJ\per\K} \\
  $P_{opt}$              & \SI{150}{\pW\per\K} \\
  $\eta_{tot}$           & 0.25 \\
  $P_{sat,970}$          & \SI{920}{\pW} \\
\bottomrule
\end{tabular}
\end{table*}

\subsection{Choice of $C$}

The detector's heat capacity $C$ is chosen to target a specific natural time constant $\tau$ once $G$ is chosen.
$\tau_{eff}$ in the prototype detectors was higher than the desired value of \SI{1}{\ms}.
To reduce risk of problems such as instability with the sub-array detectors, I decided to reduce $C$ to \SI{30}{\pJ\per\K}, which would give $\tau = \SI{8}{\ms}$.
As long as \Loop and $\beta_I$ at the operating bias point did not change, this would lead to $\tau_{eff} = \SI{2.7}{\ms}$.

\section{Detailed Detector Design} \label{sec:ch5-det-design}

\begin{table*}
\centering
\caption{
  Dimensions of prototype and sub-array detectors.
} 
\label{tab:ch5-proto-parms}
\begin{tabular}{l c}
\toprule
  Detector Dimension &  {Prototype Value} & Sub-Array Value \\
\midrule
  \TES\ Size           & $ \SI{64}{\um} \times \SI{70}{\um}$ & \\
  \TES\ Thickness      & \SI{3.4}{\mOhm} & \\
  SiN Thickness        & 3.9 & \\
  I1 Thickness         & \SI{5}{\nW\per\K} & \\
  I2 Thickness         & \SI{12}{\ms} & \\
  Number of Legs       & \SI{4}{\ms} & \\
  Leg Length           & \SI{4}{\ms} & \\
  Gold Ring Area       & \SI{150}{\pW\per\K} & \\
  Gold Ring Thickness  & 0.25 & \\
\bottomrule
\end{tabular}
\end{table*}

\section{Detector Wafer Layout} \label{sec:ch5-layout}

\section{Predicted Noise} \label{sec:ch5-predicted-noise}

