\chapter{\textsc{TES} Bolometer Theory}\label{c:tes}

This chapter summarizes the \TES\ theory used in this thesis.
I start by describing the \TES\ electrical and thermal circuits, defining relevant parameters, and stating the linearized \TES\ equations.
For reference, I then summarize the important consequences of these equations, including expressions for detector responsivity, detector response to step functions in applied power and bias current, and detector noise.
I do not derive most of these results, because excellent references are available \cite{irwin_application_1995,irwin_transition-edge_2005, mather_bolometer_1982}.

I discuss the derivation of two results in more detail.
First, I give an expression for the time-domain response to a step function in applied detector bias current.
Second, I describe a new approach for measuring the natural detector time constant $\tau$ by extrapolating several measurements of the effective detector time constant $\tau_{eff}$ high in the transition. 

\section{\textsc{TES} Electrical And Thermal Circuits}

\figref{fig:elec-thermal-circuit} shows the electrical and thermal circuits for a \TES\ bolometer.
The bolometer is voltage-biased by passing a bias current $I_{bias}$ through a shunt resistor \Rsh\ which has a much lower resistance than the normal-state resistance $R_n$ of the \TES.
Because $R_{sh} \ll R_n$, the \TES\ itself is under a voltage bias.
The current through the \TES\ is inductively coupled into a \SQUID\ for readout.
The inductance $L$ in the diagram represents the sum of the input inductance of the \SQUID, a Nyquist inductor used to limit the noise bandwidth of the detector circuit, and any parasitic inductance present in the circuit.

The \TES\ itself is represented by a variable resistance $R$, which depends on both the current through the \TES\ and the temperature of the \TES.
The \TES\ is thermally sunk to a heat capacity $C$ which is weakly linked to a temperature bath $T_b$ through a thermal conductance $G$.
Optical power falls onto the heat capacity, causing the temperature $T$ of the heat capacity and the \TES\ to rise above $T_b$.
Power dissipated in any heater resistor\footnote{As described in \sectionref{sec:heater-r}, 31~detectors have heater resistors} present on the \TES\ also contributes to this temperature rise.

Because the resistance of the \TES\ depends on the temperature of the \TES, and the temperature of the \TES\ depends on the resistance of the \TES\ through Joule heating, the electrical and thermal behavior of the \TES\ are coupled.
This coupling acts as feedback, termed ``negative electrothermal feedback'', first described in the context of \TES\ detectors by Irwin\cite{irwin_application_1995}.

This coupling acts as negative feedback.
As the optical power falling on the \TES\ increases, the temperature of the \TES\ increases, which causes the resistance of the \TES\ to increase as well.
Because the \TES\ is voltage-biased, the Joule heating is inversely proportional to the resistance, so the Joule heating decreases, which causes the temperature of the \TES\ to decrease, opposing the effect of the increased optical power.
The negative electrothermal feedback speeds up the response time of the detector and allows the detector to self-bias into the superconducting transition.

\begin{figure*}
\centering
\includegraphics{drawings/ch3-elec-thermal-circuit.pdf}
\caption{Electrical and thermal \TES\ circuits.
\textbf{Left} Electrical \TES\ circuit.
The \TES\ is biased by a stiff current $I_{bias}$ shunted across a resistor $R_{sh}$ that is much smaller than the normal-state resistance of the \TES.
The \TES\ is represented by a variable resistance $R$, and $R_{par}$ represents any parasitic resistance in the circuit.
The current through the \TES\ is inductively coupled into a \SQUID\ for readout.
The inductance $L$ represents the sum of the input inductance of the \SQUID, a Nyquist inductor, and any parasitic inductance present in the circuit.
\textbf{Middle} Thevenin-equivalent \TES\ circuit used in derivation of the linearized electrical and thermal equations for the \TES.
\textbf{Right} Thermal \TES\ circuit. The \TES\ is thermally sunk to a heat capacity $C$ which absorbs optical power. The heat capacity $C$ is connected to a heat bath $T_b$ by a weak thermal link $G$, so that its temperature is elevated above $T_b$ to $T$.
The \TES\ is warmed to temperature $T$ by applied optical power $P_{opt}$, power dissipated in a heater via $I_{htr}$ (if present), and Joule heating of the \TES\ itself.}
\label{fig:elec-thermal-circuit}
\end{figure*}

\section{Linearized Electrical and Thermal Circuits}\label{sec:lin-tes-eqn}

In the limit of small changes in \TES\ current and temperature, the resistance of the \TES\ can be expressed as
\begin{equation}
R(T_0+\delta T,I_0+\delta I) = R_0 + \alpha \frac{R_0}{T_0} \delta T + %
									 \beta_I \frac{R_0}{I_0} \delta I.
\end{equation}
The power flowing through the thermal link $G$ is assumed to follow a power law of the form
\begin{eqnarray}
P_{b} = K(T^n - T^n_{b}),
\end{eqnarray}
which can also be written in the form
\begin{eqnarray}
P_{b} = \frac{GT}{n}\left(1 - \left(\frac{T_{b}}{T}\right)^n\right),
\end{eqnarray}
where
\begin{equation}
G \equiv \frac{d P_{b}}{dT} = K n T^{n-1}.
\end{equation}

With these definitions it can be shown \cite{irwin_transition-edge_2005} that the behavior of the \TES\ is described by a pair of coupled first-order differential equations:
\begin{equation}
\frac{d}{\mathop{dt}} \begin{pmatrix} \delta I \\ \delta T \end{pmatrix}
	= - \mathcal{M} \begin{pmatrix}	\delta I \\	\delta T \end{pmatrix}
      + \begin{pmatrix} \delta V / L \\ \delta P /C \end{pmatrix},
\end{equation}
where the matrix $\mathcal{M}$ is
\begin{equation}
% For spacing see http://tex.stackexchange.com/questions/14071/how-can-i-increase-the-line-spacing-in-a-matrix
\mathcal{M} = \begin{pmatrix}
		\frac{1}{\tau_{el}} & \frac{\Loop G}{I_0 L} \\[0.75em] 
		-\frac{I_0 R_0(2 + \beta_I)}{C} & \frac{1}{\tau_I}
    \end{pmatrix}.
\end{equation}
\tableref{tab:tes-theory-summary} describes all symbols use in these equations and the rest of the chapter.

These coupled equations can be solved under different initial conditions and applied forces $\delta V$ and $\delta P$.
Discussion of three cases follows.

\textbf{\TES\ Power-to-Current Responsivity}
Driving the \TES\ with a sinusoidal $\delta P$ term, holding detector bias constant, leads to the following expression for the detector power-to-current responsivity:
\begin{equation}\label{eqn:si-full}
s_I(\omega) = 
- \frac{ \frac{1}{V_0} \frac{1}{\gamma} \frac{\Loop}{\Loop + 1} }
       { 1 + j \omega \left( \tau_{eff} - \frac{1}{\gamma}\frac{\Loop}{\Loop + 1} \frac{L}{R_0}\right) - \omega^2 \frac{L}{R_0}\frac{\tau_{eff}}{1 + \beta_I + R_L / R_0}},
\end{equation}
\begin{equation}
\gamma \equiv 1 + \frac{\beta_I}{1+\Loop} - \frac{\Loop - 1}{\Loop + 1}\frac{R_L}{R_0}.
\end{equation}
Here $\tau_{eff}$ is given by
\begin{equation} \label{eqn:teff}
  \tau_{eff} = \frac{\tau}{1 + \frac{1 - R_L / R_0}{1 + \beta_I + R_L / R_0} \Loop}.
\end{equation}

While imposing, these expressions are much simpler in the limit which generally hold for operating \TES\ detectors: strong voltage bias ($R_L \ll R_0$), and $\tau \ll L/R$. 
In this limit the power-to-current responsivity becomes
\begin{equation}
s_I(\omega) = -\frac{1}{V_0} \frac{\Loop}{1+\beta_I + \Loop} \frac{1}{1 + j \omega \frac{\tau}{1 +\Loop/(1+\beta_I)}}
\end{equation}
The detector response time is given by the natural detector time constant $\tau$, sped up by a factor of $1 + \Loop(1+\beta_I)$; for $\beta_I \ll 1$, this factor is typical of negative feedback, and justifies calling \Loop\ the ``loop gain'' of the detector.
In the further limit of strong electrothermal feedback ($\Loop \gg 1, \gg \beta_I$), the \DC\ responsivity is simply the inverse of the voltage bias.
This means that because of the strong electrothermal coupling, an increase in applied optical power is exactly canceled by a decrease in detector Joule heating, so that the \TES\ temperature remains unchanged.

\textbf{\TES\ Response to Step Function in Power}
As demonstrated in \sectionref{sec:bias-step}, our detectors are always operated in a regime where $\tau_{eff} \gg \tau_{el}$.
Under these conditions, \eqnref{eqn:si-full} simplifies to
\begin{equation} \label{eqn:htr-step-resp-high}
s_I(\omega) = - \frac{1}{V_0 \gamma}\frac{\Loop}{\Loop + 1}
                       \left(1 + j \omega \tau_{eff}\right)^{-1}.
\end{equation}
This implies that the time-domain response to step in applied power, for example from a heater, is
\begin{align} \label{eqn:htr-step-resp-high-time}
    \delta I(t) & = - \delta P s_I(0) (1 - e^{-t/\tau_{eff}}) \\
                & = - \frac{\delta P}{V_0 \gamma}\frac{\Loop}{\Loop + 1}
                      (1 - e^{-t/\tau_{eff}}).
\end{align}
This can be used to measure $\tau_{eff}$ directly as well as \DC\ responsivity once heater power has been calibrated (\sectionref{sec:teff-resp}).
As described in \sectionref{sec:tau-nat-theory}, it can also be used to measure the detector natural time constant $\tau$.
These measurements are described further in \sectionref{sec:tau-nat}.

\textbf{\TES\ Response to Step Function in Bias Current}
To derive the behavior of the \TES\ after a step function in applied bias, we solve the equations under the conditions
\begin{equation}
\begin{pmatrix} \delta I(0) \\ \delta T(0) \end{pmatrix} = \begin{pmatrix} 0 \\ 0 \end{pmatrix}
\end{equation}
with constant driving force starting at time zero of
\begin{equation}
\begin{pmatrix} \delta I_{bias} R_{sh} / L \\ 0 \end{pmatrix}
\end{equation}
Solving this system leads to the following expression for the \TES\ current as a function of time:\footnote{A Mathematica notebook which verifies this solution, as well as other solutions to the linearized \TES\ equations, can be found at https://gist.github.com/danbek/8591076}
\begin{equation}\label{eqn:bias-step-resp}
\delta I (t)
   = - \frac{\delta I_{bias} R_{sh}}{R_0} 
       \frac{(\Loop - 1)
             \left(1 - \frac{\tau_{eff} - \tau_I}{\tau_{eff} - \tau_{el}} e^{-t/\tau_{eff}}
                 	       + \frac{\tau_{el} - \tau_I}{\tau_{eff} - \tau_{el}} e^{-t/\tau_{el}} \right)}
            {1 + \beta_I + R_L/R_0 + \Loop(1 - R_L/R_0)}
       .
\end{equation}
This expression is complicated, but the behavior can be understood as follows.
Immediately after an increase in bias current the voltage across the \TES\ begins to increase, with a time constant of $\tau_{el}$.
As the voltage increases, the Joule power in the \TES\ increases, which warms the \TES.
This warming increases the resistance of the \TES.
Because the \TES is voltage-biased, this reduces Joule power in the \TES, which tends to cool the detector as well as reduce current through the detector.
This negative electrothermal feedback effect occurs with a time constant of $\tau_{eff}$.
Whether the final current through the \TES\ is higher or lower than the original currents depends on the loop gain.
For $\Loop < 1$ the current increases, for $\Loop > 1$ it decreases and for $\Loop = 0$ the current through the \TES\ remains unchanged.

\eqnref{eqn:bias-step-resp} depends on \Loop\ and $\beta_I$ in a complicated way through $\tau_{eff}$, $\tau_I$, $\tau_{el}$, and the prefactor.
Nevertheless, if the response of a \TES\ to a bias step can be measured with sufficient bandwidth to track the initial fast electrical response, bias steps can be used to measure \Loop\ and $\beta_I$ by performing non-linear parameter fitting to \eqnref{eqn:bias-step-resp}.
Measurements of \Loop\ and $\beta_I$ using this technique are described in \sectionref{sec:bias-step}.

When the \TES\ is superconducting, \eqnref{eqn:bias-step-resp} takes on a much simpler form.
Setting $R_0 = \Loop = \beta_I = 0$, the result is
\begin{equation}\label{eqn:bias-step-resp-sc}
\delta I(t)
   = - \frac{\delta I_{bias} R_{sh}}{R_{L}} 
       \left(1 - e^{-t/(L/R_L)} \right).
\end{equation}
Similarly, when the detector is fully normal, so that $\Loop = \beta_I = 0$, \eqnref{eqn:bias-step-resp} becomes
\begin{equation}\label{eqn:bias-step-resp-normal}
\delta I(t)
   = - \frac{\delta I_{bias} R_{sh}}{R_n + R_{L}} 
       \left(1 - e^{-t/(L/(R_n+R_L))} \right).
\end{equation}
The \TES\ response to bias steps in the superconducting and normal states can thus be used as measurements of $L$ and $R_n$.

\begin{table*}[t]
\centering
\caption{Symbols and parameters used in describing behavior of \TES\ detectors.}
\label{tab:tes-theory-summary}
{\renewcommand{\arraystretch}{1.5}%
\begin{tabular}{l l}
\toprule
Symbol &  Explanation \\
\midrule
\addlinespace
$I_{bias}$ & Current applied across shunt to bias \TES. \\
$R$ & \TES\ resistance (depends on temperature and current) \\
$R_n$ & \TES\ normal-state resistance \\
$R_{sh}$ & Shunt resistance \\
$R_{par}$ & Represents any parasitic resistance in \TES\ circuit \\
$R_L \equiv R_{sh} + R_{par}$ & Load resistance used in analysis of \TES\ circuit \\
$T$ & \TES\ temperature \\
$T_b$ & Thermal bath temperature \\
$I_0$, $R_0$, $V_0$, $T_0$ & \TES\ current, resistance, voltage, and temperature at bias point \\
$P_{bath} = K(T^n - T_{b}^n)$ & Total heat flow from \TES\ island to heat bath \\
$n$ & Power-flow index \\
$P_{opt}$ & Optical power falling onto \TES\ heat capacity \\
$P_{htr}$ & Power applied to \TES\ by heater resistor \\
$P_{J}$ & Joule power dissipated by \TES\ \\
$C$ & Heat capacity of \TES\ island \\
$G \equiv \frac{d P_{bath}}{d T} = K n T^{n-1}$ & Weak-link differential thermal conductance \\
$\tau \equiv \frac{C}{G}$ & \TES\ natural time constant \\
$\tau_{el} \equiv \frac{L}{R_L + R_0(1 + \beta_I)}$ & \TES\ electrical time constant \\
$\tau_I \equiv \frac{\tau}{1-\Loop}$ & \TES\ constant-current time constant \\
$\tau_{eff} \equiv \frac{\tau}{1 + \frac{1 - R_L / R_0}{1 + \beta_I + R_L / R_0}\Loop}$ & \TES\ effective time constant \\
$\alpha \equiv \frac{T_0}{R_0} \frac{\partial R}{\partial T}$ & \\
$\beta_I \equiv \frac{I_0}{R_0} \frac{\partial R}{\partial I}$ & \\
$\Loop \equiv \frac{I_0^2 R_0 \alpha}{G T_0}$ & Loop gain \\
$\delta V = \delta I_{bias} R_{sh}$ & Change in bias voltage applied to \TES\ \\
$\delta P$ & Change in power (optical or heater) falling on \TES\ \\
\bottomrule
\end{tabular}
}
\end{table*}

\section{Measurement of Natural Time Constant}\label{sec:tau-nat-theory}

Near the top of the superconducting transition, $\Loop < 1$, so that $\tau_{eff} > 0.5 \tau$.
Detectors circuits are always designed so that $\tau \gg L/R_n$, so that the response to a step in applied heater power is given by \eqnref{eqn:htr-step-resp-high-time}.

As the fully normal state is approached, $\tau_{eff}$ approaches $\tau$, so that measuring the $\tau_{eff}$ very high in the transition will give a measurement of $\tau$.
However, the power-to-current  responsivity becomes smaller and smaller high in the transition, so that signal-to-noise becomes worse and worse.
If we knew \Loop\ and $\beta_I$ high in the transition we could extract $\tau$ from $\tau_{eff}$, but we do not always know these values.

To avoid these problems, an expression can be obtained linking $\tau$ and $\tau_{eff}$ that holds independent of location in the transition, as long as the assumption $\tau_{eff} \gg L/R_0$ holds.
The \DC\ response to a step in applied power $\delta $ is given by
\begin{equation}
\delta I = \frac{\delta P}{I_0 R_0}\frac{\Loop}{1 + \beta_I + R_L/R_0 + (1 - R_L / R_0)\Loop}.
\end{equation}
This equation can solved for \Loop, and then substituted into the expression for $\tau_{eff}$.
This leads to
\begin{equation}\label{eqn:teff-from-tau}
\tau_{eff} = \tau - \tau \mathcal{K} I_{bias} \delta I,
\end{equation}
\begin{equation}
\mathcal{K} \equiv \frac{R_{sh}}{\delta P} \frac{R_0 - R_L}{R_0 + R_L}.
\end{equation}
Here the relation
\begin{equation}
I = I_{bias}\frac{R_{sh}}{R + R_L}
\end{equation}
has also been used.

\eqnref{eqn:teff-from-tau} holds independent of \Loop\ and $\beta_I$.
The factor $\mathcal{K}$ depends on the bias point, but high in the transition this dependence is weak, so that $\mathcal{K}$ can be treated as a constant.

To use \eqnref{eqn:teff-from-tau} to measure $\tau$, steps in heater power are applied to the \TES\ at a set of bias points close to the normal state.
At each bias point the \DC\ change in \TES\ current $\delta I$ and $\tau_{eff}$ are measured by fitting the \TES\ response to \eqnref{eqn:htr-step-resp-high}, and the bias current $I_{bias}$ is recorded.
A non-linear curve fit can then be applied to \eqnref{eqn:teff-from-tau} to solve for $\tau$ and $\mathcal{K}$.
Alternately, $\mathcal{K}$ can be calculated if all factors feeding into it are known, and then \eqnref{eqn:teff-from-tau} can be solved directly for $\tau$.

\sectionref{sec:tau-nat} presents measurements of $\tau$ for 4 detectors using this technique.

\section{\textsc{IV} Curve Analysis} \label{sec:ch3-iv-curve}

\TES\ detector current-vs-voltage (\IV) curves contain important information about the behavior of \TES\ detectors.
They directly give the resistance of the \TES\ in both the normal state and throughout the superconducting transition.
But they also allow other properties of the \TES\ to be measured by comparing \IV\ curves taken under different operating conditions, such as different bath temperatures and applied heater and/or optical power loads.

The total amount of power flowing through the \TES\ thermal conductance $G$ is given by
\begin{equation}\label{eqn:ch3-tes-ptot}
P_{tot} = K(T^n - T_b^n) = P_{opt} + P_{htr} + I^2 R(T,I).
\end{equation}
We can make the assumption that at the start of the superconducting transition, where $R \approx R_n$, $\beta_I = 0$, i.e.\ the resistance of the \TES\ depends only on the \TES\ temperature, and not on the current through the \TES.
This assumption has been observed to hold empirically for many different types of \TES\ detectors, and there are also theoretical reasons to expect it to be true\cite{bennett_resistance_2013}.
Under this assumption, each time the \TES\ reaches, e.g.\ $R = 0.99R_n$, the temperature of the \TES\ is the same, and therefore $P_{tot}$ will be the same.
A relationship of the following form must therefore hold:
\begin{equation}\label{eqn:ch3-tes-99Rn}
P_{J} \equiv I^2 R = P_{tot} - P_{opt} - P_{htr}.
\end{equation}

\eqnref{eqn:ch3-tes-99Rn} is used in two different ways in this thesis to extract information about the \TES\ detectors, as described in the following subsections.

\subsection{Calibration of Heater Resistors}

If a set of \IV\ curves are taken at the same bath temperature but different heater biases, \eqnref{eqn:ch3-tes-99Rn} takes on the form
\begin{equation}\label{eqn:ch3-rhtr-fit}
P_J = (P_{tot} - P_{opt}) - I_{htr}^2 R_{htr}.
\end{equation}
A fit can be made to this function, with the quantities $R_{htr}$ and $P_x = P_{tot} - P_{opt}$ to be solved for.
This allows measurement of the resistance of the \TES\ heaters.

\figref{fig:ch7-heater-r-plots} (reproduced in this chapter for convenience as \figref{fig:ch3-heater-r-plots}) explains this idea further by a series of plots, all of which were taken for \RCm{29}{1}.
The upper left plot shows a set of \TES\ \IV\ curves taken at $T_b = 1100$~mK, with only the applied heater bias changing.
The upper right plot shows the same data, but transformed into \TES\ Joule power and \TES\ resistance.
As applied heater current decreases, the Joule power at the start of the transition decreases.
In the lower left, the Joule power at $0.99R_{n}$ is plotted vs applied heater current.
A fit to \eqnref{eqn:ch3-rhtr-fit} is also plotted.
Finally, the lower right plot show the $R$ vs $P_J$ plots after the heater power has been added to each curve.
This plots shows that the powers are equalized very high in the transition, where the assumption of Joule power dependent only on \TES\ resistance holds.
It also shows that this assumption breaks down deeper in the transition.

It is worth noting that determining the values of $R_{htr}$ requires knowing the current through the heaters.
Any error in the assumed heater current will lead to a corresponding error in the derived $R_{htr}$ values.
But because $R_{htr}$ is determined through the power dissipated in the resistor, the product $I_{htr}^2 R_{htr}$ will be unchanged under different assumptions for $I_{htr}$.
This means that whenever the value $R_{htr}$ is used to calculated a power, the power value will be correct even in the face of an incorrect assumption about the current through the heater.

\sectionref{sec:heater-r} uses this approach to calculate $R_{htr}$ for the seven working heaters on columns 0 and 1.

\begin{figure*}
\includegraphics{drawings/ch7-heater-r-plots.pdf}
\caption{Plots describing heater measurements, for the case of \RCm{29}{1}.
\textbf{Upper Left} \IV\ curves. The \IV\ curves should become vertical when the detector becomes fully superconducting at zero voltage, but these curves shows a non-infinite slope. The reason for this is that the readout system as configured for these \IV\ curves was unable keep up with the rapid change of current in the superconducting branch.
\textbf{Upper Right} Same data as in upper left plot, but represented in terms of \TES\ Joule power and resistance. As the bias current for the heaters is increased, the curves shift to the left.
\textbf{Lower Left} Measured $P_{J}$ vs heater current at $0.99R_n$, as well as fit to \eqnref{eqn:ch3-rhtr-fit}.
\textbf{Lower Right} Same plot as upper right, but the heater power based on $R_{htr} = \SI{23.6}{\ohm}$ has been added to each curve.
This demonstrates that $\beta_I = 0$ does not hold below the very top of the transition.}
\label{fig:ch3-heater-r-plots}
\end{figure*}

\subsection{Measurement of \textsc{TES} Differential Thermal Conductance $G$}

With knowledge of the heater resistances, \IV\ curves can be taken over a wide range of bath temperatures, which enables a measurement of the \TES\ thermal conductance $G$ and transition temperature $T_c$.
In this case $P_{tot}$ will different for each \IV\ curve, so that \eqnref{eqn:ch3-tes-ptot} can be used in the form
\begin{equation}\label{eqn:ch3-g-fit}
P_{htr} + P_J + P_{opt}= \frac{G T_c}{n}\left(1 - \left(\frac{T_b}{T_c}\right)^n\right).
\end{equation}
A non-linear curve fit can then be used to find $G$, $T_c$, and $n$.
The upper plots in \figref{fig:heater-g-plots} (reproduced in this chapter for convenience as \figref{fig:ch3-g-plots}) show an example of this fit for \RCm{31}{2}.
The fit procedure leads to correlation between the fit values of $G$ and $n$ which indicated degeneracy in the fit between $G$ and $n$.

\sectionref{sec:g-psat} uses this approach to calculate $G$, $T_c$ and $n$ for the seven detectors with working heaters on columns 0 and 1.

\begin{figure*}
\includegraphics{drawings/ch3-g-plots.pdf}
\caption{Plots showing fit to \eqnref{eqn:ch3-g-fit} for \RCm{31}{2}.
\textbf{Left} Plot showing $P_{sat}$ vs $T_b$ assuming $P_{opt} = 150$~pW (see \sectionref{sec:g-psat}).
The red line shows the best fit to \eqnref{eqn:ch3-g-fit}.
The data cover 36 data points including 25 temperatures from \SIrange{995}{1160}{\mK} and 11 different heater biases.
\textbf{Right} Scatter plot showing covariance between the fitted values of $G$ and $n$, in terms of 95 \% confidence ellipses.} 
\label{fig:ch3-g-plots}
\end{figure*}

\section{\textsc{TES} Bolometer Noise} \label{sec:ch3-tes-noise}

There are three sources of detector noise in \TES\ bolometers: Johnson noise in the \TES\ resistance, Johnson noise in the load resistor $R_L$, and thermal fluctuation noise across the weak thermal link $G$.
Additionally, intrinsic fluctuations in the number of arriving photons leads to photon noise, which can be significant source of noise for low-temperature \TES\ bolometers.
Expressions for these sources of noise are shown in \tableref{tab:tes-noise}.

The function $F$ that enters into the thermal fluctuation noise accounts for the temperature gradient between the \TES\ and the bath.
The form of $F$ depends on whether the mean free path of phonons crossing the thermal conductance is long or short compared with the conductance.
In the case of a short mean free path, $F$ depends on $n$ and is given by \cite{mather_bolometer_1982}
\begin{equation}
  F(T_0,T_b) = \frac{n}{2n+1} \frac{1 - (T_b/T_0)^{2n+1}}{1 - (T_b/T_0)^{n}}.
\end{equation}
In the case of a long mean free path, $F$ is independent of $n$ and is given by \cite{boyle_performance_1959}
\begin{equation}
  F(T_0, T_b) = \frac{1}{2} (1 + (T_b/T_0)^5)
\end{equation}
For the detectors described in this thesis, $n \approx 3.5$, $T_0 \approx 1.2$~K, and $T_b \approx 1.1$~K.
Under these conditions, both expressions for $F$ have approximately the same value, 0.83.

For typical operating conditions of \TES\ bolometers, thermal fluctuation noise dominates Johnson noise at low frequencies.
This can be see by taking the ratio of $S^2_{TES}$ to $S^2_{TFN}$ (ignoring factors of order unity):
\begin{equation}
\frac{S^2_{TES}}{S^2_{TFN}} \approx \frac{1}{\alpha \Loop} (1 + (\omega \tau)^2)
\end{equation}
At low frequencies the \TES\ resistor current noise is suppressed below thermal fluctuation noise by a factor of $1/\alpha \Loop$.
\TES\ detectors are always biased so that $\Loop > 1$, and values for $\alpha$ in the transition for our detectors are 20--400 (see \figref{fig:ch7-bias-step-results} in \sectionref{sec:bias-step}).
Examination of \tableref{tab:tes-noise} shows that current noise from the load resistor is lower than that from the \TES\ resistor by a factor of $(\Loop-1)^2 (R_0 / R_L) (T_0/T_L)$.
All this points to the conclusion that detector noise in \TES\ bolometers is dominated by thermal fluctuation noise.

Photon noise arises due to quantum fluctuations in the number of photons arriving during a given time interval.
This noise is expressed as \cite{zmuidzinas_thermal_2003}
\begin{equation}\label{eqn:photon-noise}
  S^2_{ph} = 2 h \nu P_{opt} (1 + \eta \bar{n}),
\end{equation}
where $\bar{n}$ is the photon occupation number, given by
\begin{equation}
  \bar{n} = \frac{k_B T}{h \nu}
\end{equation}

\sectionref{sec:ch5-predicted-noise} discusses predicted noise levels for our detectors.
\sectionref{sec:det-noise} discusses measurements of detector noise.

\begin{table*}[t]
\centering
\caption{Noise in \TES\ bolometers, referred to power absorbed in bolometer. To obtain current noise passing through the bolometer, multiply each power spectral density by $|s_I(\omega)|^2$.}
\label{tab:tes-noise}
{\renewcommand{\arraystretch}{2.0}%
\begin{tabular}{l l}
\toprule
Noise Source &  Noise Power Spectral Density \\
\midrule
\TES\ Resistor & $S^2_{TES} = 4 k_B T_0 I_0^2 R_0 \xi(I_0) \frac{(1 + (\omega \tau)^2) } {\Loop^2}$ \\
Load Resistor & $S^2_{L} = 4 k_B T_L I_0^2 R_L \frac{(1 + (\omega \tau_I)^2) (\Loop -1)^2} {\Loop^2}$ \\
Thermal Fluctuation Noise & $S^2_{TFN} = 4 k_B T_0^2 G F(T_0, T_b)$ \\
\bottomrule
\end{tabular}
}
\end{table*}
