\chapter{System Specifications and Solutions}\label{c:specs}

% xxx abbrev element names like Al, Au?

The \Imager's specifications were chosen to allow the system to serve as a ``gold standard'' for investigating the phenomenology of passive video imaging.
The intended use of the system is to take images of a variety of concealed weapons and other objects, hidden beneath different types of clothing, in both indoor and outdoor scenarios.
The goal is to gain a thorough understanding of trade-offs between video frame rate, image resolution, and noise.
This understanding will then aid the design of future systems for use in specific operational scenarios.

To that end, the system has been designed to achieve high resolution images with \NETD\ that is limited by photon noise.
This chapter describes the specifications for the \Imager, and summarizes the technical approach taken to meet those specifications.

\section{Specifications} \label{sec:ch2-specifications}

The first goal of the \Imager\ is to achieve uncompromised noise performance through the use of a large number of photon-noise-limited detectors.
As discussed below, this requires the use of detectors at cryogenic temperatures.
We have chosen to use 1004 Transition Edge Sensor (\TES) bolometers as the detectors.
The detectors will be installed as four individual 251-detector sub-arrays.
This thesis discusses results using one of these four sub-arrays.

The atmospheric window at \abt{\SI{350}{\GHz}} (\abt{\SI{860}{\um}}) was chosen for the optical band.
As can be seen from \figref{fig:ch1-clothes-atmos-trans}, this band is a good trade-off between transmission through the atmosphere and clothing (which favors lower frequencies) and spatial resolution (which favors higher frequencies).
A \SI{35}{\GHz} full-width-half-maximum (\FWHM) band was chosen to fit within this atmospheric window.

The design standoff distances from the observed target are \SIrange{16}{28}{\m}, configurable by changing the distance between the cryostat containing the detectors and the rest of the optical system.
This range of standoff distances was chosen to provide separation from observation targets such as a suicide bomber, without requiring an unreasonably large optical aperture.
A desirable spatial resolution at \SI{16}{\m} standoff would be \SI{1}{\cm}.
However, at the center wavelength of the observation band of $\lambda_0 = \SI{863}{\um}$ (see \sectionref{sec:ch4-filters}), using the Raleigh criterion \eqnref{eqn:ch1-raleigh} leads to a required aperture size of
\begin{equation}
  D = 1.22 \times \SI{863}{\um} \times \frac{\SI{16}{\m}}{\SI{1}{\cm}} = \SI{1.7}{\m}.
\end{equation}
It was decided that a mirror of this size would be too expensive and large for this prototype system, so \SI{1.3}{\m} was chosen instead.
After accounting for the fact that the outer edge of the primary mirror is not illuminated (\sectionref{sec:ch4-optical-design}), the predicted resolution of the system is \SI{1.4}{\cm} (\sectionref{sec:ch4-feedhorn-design}).

Security applications can require observing a moving target, so that video frame-rates are needed.
Exactly how fast the frame-rate needs to be in order to support accurate tracking can vary based on the scenario.
The components of the \Imager\ have all been designed to allow frame rates of up to 20 frames per second.
The design field of view at \SI{16}{\m} is $\SI{1}{\m} \times \SI{1}{\m}$.

In order to achieve all of these requirements, other aspects of the system design were compromised.
In particular, the desire for high spatial resolution at longer standoff distances requires a large optical aperture, which increases the size of the system, reducing portability.
Although the system is mounted on rollers so that it can be moved within the lab or into other labs, the location at which it points can not be steered in real-time.
Additionally, the focus distance can only be changed by reconfiguring the mounting structure that connects the cryostat to the optics.

\section{Transition Edge Sensor Bolometer Basics} \label{sec:ch2-tes-basics}

A bolometer detects optical power by measuring the temperature of an isolated object which absorbs the optical power.
\figref{fig:ch2-bolo-cartoon} shows a schematic illustration of a bolometer.
Optical power $P_{opt}$ falls onto an absorber with heat capacity $C$.
The optical power is thermalized in the absorber, causing its temperature $T$ to rise.
The temperature $T$ to which it rises is determined by the thermal conductance $G$ which connects the absorber to a thermal bath at temperature $T_b$.
See \cite{richards_bolometers_1994} for an excellent overview of bolometric detectors.

\begin{figure*}
\centering
\includegraphics{drawings/ch2-bolo-cartoon.pdf}
\caption[Bolometer schematic]{
  Schematic illustration of a bolometer.
  The bolometer detects optical power $P_{opt}$ by absorbing it in an absorber with heat capacity $C$.
  The absorbed optical power causes the absorber temperature to rise to a temperature $T$ above a thermal bath held at $T_b$.
  The rise in temperature is determined by the thermal conductance $G$.
}
\label{fig:ch2-bolo-cartoon}
\end{figure*}

In a Transition Edge Sensor (\TES) bolometer a thin superconducting film is used as a thermometer to measure the temperature $T$.
A superconductor is a material which loses all electrical resistance when its temperature falls below a critical temperature $T_c$ \cite{tinkham_introduction_1996}.
This phase transition from the ``normal'' to ``superconducting'' state can be very narrow.
The steepness of the transition is characterized by the logarithmic temperature sensitivity $\alpha$ given by
\begin{equation}
  \alpha = \frac{T_0}{R_0} \frac{\partial R}{\partial T}.
\end{equation}
Typical values of $\alpha$ for \TES\ detectors are 100 -- 1000.

The first demonstration of a \TES\ bolometer was in 1941 \cite{andrews_attenuated_1942}, and the first demonstration using Al (the material used for the detectors in this thesis) was in 1977 \cite{clarke_superconductive_1977}.
But operation of a \TES\ detector requires maintaining the temperature of the \TES\ within the narrow transition, which proved challenging.
An increase in the temperature of the \TES\ --- caused by an increase in absorbed optical power --- causes a rise in the temperature of the \TES.
When current-biased this rise in temperature causes an increase in $I^2 R$ Joule heating, causing the temperature of the detector to rise further, leading to instability.

Thus, \TES\ bolometers did not come into widespread use until the development of the voltage-biased \TES\ sensors \cite{irwin_application_1995} coupled with Superconducting Quantum Interference Device (\SQUID) readout systems.
When voltage-biased, a rise in \TES\ temperature leads to a decrease in Joule heating, which lowers the temperature of the device.
With a sufficiently steep transition, this electrothermal feedback process allows any change in incident power to be exactly matched by an opposite change in Joule heating, so that the device self-stabilizes at a new current level in the normal-to-superconducting transition while leaving the device temperature unchanged.
Because the device is voltage-biased, absorbed optical power causes a current change, so that a current amplifier is required for readout.
This is accomplished through the use of \SQUIDs, which are very sensitive magnetometers and so can be used to detect small changes in current.

\chapterref{c:tes} covers the theory of \TES\ operation required for understanding of this thesis.
The most detailed and authoritative reference for \TES\ detectors is by Irwin and Hilton \cite{irwin_transition-edge_2005}.

\section{Transition Edge Sensors for Passive Imaging} \label{sec:ch2-tes-passive}

\TES\ detector have two important advantages for building passive imaging systems: the can be photon-noise-limited, and it is straigtforward to manufacture, operate, and read out arrays containing large numbers of them.

The fundamental noise limit in bolometers is set by random thermal fluctuations of the temperature of the absorbing element, termed ``thermal fluctuation noise''.
As described in \sectionref{sec:det-parm-choice}, the noise equivalent power (\NEP) of this noise source, expressed in units of \pnoise\ referred to power absorbed in the bolometer, is proportional to $\sqrt{k_B T_b P_{opt}}$.
In a properly designed \TES, this source of noise will dominate other sources of noise in the system (Johnson noise in the \TES, \SQUID\ noise).
A sufficiently low bath temperature $T_b$ can reduce thermal fluctuation noise to below the noise level caused by fluctuations in the arrival rate of photons.
As described in \sectionref{sec:ch5-predicted-noise}, the expected \NETD\ for the \Imager\ populated with 1004 detectors is \SI{38}{\mK}, well below the performance benchmark established in \sectionref{sec:ch1-netd-reqs}.

The ability to put large numbers of \TES\ detectors onto focal planes is due to two factors.
First, they can be fabricated using standard lithographic clean-room techniques, which allows development of array-scale focal planes.
Second, the self-stabilizing behavior of voltage-biased \TES\ detectors enables array-scale operation because it greatly relaxes requirements on uniformity of detector characteristics across a wafer in the fabrication process.

Reading out a large number of detectors individually requires an even larger number of wires, which complicates cryogenic designs.
An important part of the success of \TES\ detector arrays for millimeter and submillimeter astronomy has been the development of multiplexed \SQUID\ readout systems.
%xx\cite{tdm,berkeley fdm, europe}
The \Imager\ uses a time-division multiplexed readout system, the basics of which are described in \sectionref{sec:det-readout}, and the details of which can be found in the references cited there.
 

\section{Other Cooled Detector Imaging Systems}

Aside from the work described in this thesis, three other groups are also working on cooled detector passive imaging systems.
See \tableref{tab:ch2-sys-compare} for a summary of some characteristics of these and other security imaging systems.

One system has been developed by a group working at the Institute of Photonic Technology (IPHT) in Jena, Germany.
The first generation of this system \cite{heinz_toward_2011} operates at \SI{350}{\GHz} with a \SI{23}{\percent} optical bandwidth.
\NETD\ is measured at \SI{0.4}{\K} at 10 frames per second over a \SI{1}{\m} diameter field of view.
Currently under development is a second generation system \cite{heinz_development_2013,may_next_2013}.
This system increases the number of detectors from 20 to 64 and the frame rate from 10 to 25 frames per second.
\NETD\ is predicted to be \SI{160}{\mK}.

A second system is under development by a group at MilliLab in Finland, working in collaboration with researchers at \NIST.
The first generation of this system operated over a wide optical band, \SIrange{200}{1000}{\GHz}, achieving \SI{0.6}{\K} \NETD\ \cite{grossman_passive_2010}.
A second generation system is currently underway, with most changes intended to reduce the size and power requirements of the system, while also doubling the number of detectors from 64 to 128 \cite{luukanen_applications_2012}.
The system uses superconducting \TES\ bolometers with $T_c \approx \SI{9}{\K}$.
The readout system is entirely different than that used by the Jena system and the system described in this thesis.
In order to eliminate the use of \SQUIDs, an approach based on room-temperature amplification was developed that relies on negative feedback to keep the detectors voltage-biased \cite{penttila_low-noise_2006}.

Finally, a group based at Cardiff University in Wales is in the early stages of developing a passive imaging system \cite{day_broadband_2003} based on Microwave Kinetic Inductance Detectors \cite{wood_kidcam_2011}.
No images or \NETD\ estimates have yet been presented publicly for this system.

\begin{sidewaystable}
\small
\centering
\caption[Table summarizing capabilities of different video imaging systems]{
  Table summarizing capabilities of different security video imaging systems.
  The values for the MilliLab II and IPHT Jena II systems are predicted, not measured, as measurements of the performance of these systems have not yet been published.
  As discussed in \sectionref{sec:ch8-noise-model}, the \NETD\ for the \Imager\ is based on a ``flat field'' video.
  Values labeled ``N/A'' are unavailable.
}
\label{tab:ch2-sys-compare}
\begin{tabular}{lccccccc}
\toprule
System &
 \specialcell{Optical Band \\ (\si{\GHz})} &
 \specialcell{Detector \\ Count } &
 Frame Rate &
 \specialcell{Standoff \\ Distance (\si{\m})} &
 \specialcell{Field of View \\ ($\si{m} \times \si{\m}$)} &
 \specialcell{Resolution \\ (\si{\cm})} &
 \specialcell{\NETD\ \\ (\si{\mK})} \\
\midrule
ThruVision & $250 \pm 20$ & 8 & 6 & 5.0 -- 15.0 & $1 \times 1$ (at \SI{10}{\m}) & N/A & 1000 \\
Microsemi GEN 2 & $90 \pm$ N/A & N/A & 4 -- 12 & \abt{3} & N/A & 5 & N/A \\
MilliLab I & 200 -- 1000 & 64 & 6 & 5.0 & $4 \times 2$ & 4.0 & 600 \\
MilliLab II & 200 -- 1000 & 128 & 6 & 5.0 & $2 \times 1$ & 2.5 & 400 \\
IPHT Jena I & $350 \pm 40$ & 20 & 10 & 8.5 & \SI{1}{\m} diam. & 1.7 & 400 \\
IPHT Jena II & $350 \pm 40$ & 64 & 25 & 8.5 & $2 \times 1$ & 1.0 & 160 \\
\NIST\ \Imager\ & $347 \pm 17$ & 251 & 6 & 16 & $0.78 \times 0.55$ & 1.7 & 100 \\
\bottomrule
\end{tabular}
\end{sidewaystable}

