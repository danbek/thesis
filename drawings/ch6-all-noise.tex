\documentclass[12pt]{article}

\usepackage{../thesis}

\usepackage{tikz} 
\usepackage{pgfplots} % drawing plots right here in this file!
\pgfplotsset{compat=1.9} % latest stable release

\usepgfplotslibrary{groupplots}

\pgfplotsset{
  every axis/.append style={
    scale only axis,
    %width=2.5in,height=1.6in,
  },
  /tikz/every picture/.append style={
    baseline
  }
}

\begin{document}

%
% thesis/ana_all_noise()
% 

\pagestyle{empty}

\begin{tikzpicture}

\begin{groupplot}[
    group style={group size=1 by 3,
                 vertical sep=0.2in,
                 %horizontal sep=0.5in,
                 x descriptions at=edge bottom,
                 },
    width=4in,height=2in,
    %xmode=log,ymode=log,
    xmin=0,xmax=100,
    ymin=0,ymax=80,
    ylabel={Quantity},
    xlabel={Total Noise Standard Deviation (\si{\fW})}
    ]

  \nextgroupplot
    \addplot[hist={data min=0,data max=100,bins=50},fill=blue!75]
      table[y=totP]{../data/ch6-all-noise-50.dat};

  \nextgroupplot
    \addplot[hist={data min=0,data max=100,bins=50},fill=blue!75]
      table[y=totP]{../data/ch6-all-noise-20.dat};

  \nextgroupplot
    \addplot[hist={data min=0,data max=100,bins=50},fill=blue!75]
      table[y=totP]{../data/ch6-all-noise-10.dat};

\end{groupplot}

\node[anchor=west] at ($(group c1r1.center)$) {$I_{FB} = 50$};
\node[anchor=west] at ($(group c1r2.center)$) {$I_{FB} = 20$};
\node[anchor=west] at ($(group c1r3.center)$) {$I_{FB} = 10$};

\end{tikzpicture}
\end{document}
