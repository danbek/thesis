\chapter{Introduction} \label{c:intro}

% http://ieeexplore.ieee.org/stamp/stamp.jsp?tp=&arnumber=6005328
% Instead, the image is dominated by speckle that is characteristic for coherent imaging. Speckle comes from the large variations in the intensity of backscattered radiation coming from the diversity in angles of the beam-target incidence, and it dominates any difference in the intrinsic reflectivity of, say, PVC pipe material, clothing, and skin.

%% RFI HSHQDC-14-00015
%% commercially available standoff - Brijot Gen 2, SET CounterBomber
%% TSA 700 AIT (advanced imaging technology) systems at 160 airports

For over thirty years the millimeter and sub-millimeter-wavelength spectrum has been the subject of intense interest for military and security imaging applications.
The reason for this interest is that the spectral region from \SIrange{100}{1000}{\GHz} offers a good compromise between transmission through obscuring materials (favoring lower frequencies) and spatial resolution (favoring higher frequencies) \cite{kruse_why_1981}.
This interest has driven a long line of technological advancement in sources, detectors, and other technologies at these wavelengths \cite{popovic_thz_2011}.
This advancement has taken place in both ``active'' imaging, in which an observation target is illuminated by light and the reflections from that target are detected, and `passive'' imaging, in which the target's thermal emissions are detected.

Over a similar time span the millimeter and sub-millimeter astronomical community has also been very interested in these wavelengths.
In particular, the desire to make more and more detailed maps of the Cosmic Microwave Background (\CMB) radiation have driven this community's need for instruments capable of higher and higher sensitivities.
During the 1980's and early 1990's individual cooled bolometric detectors capable of achieving photon-noise-limited --- or near photon-noise-limited --- performance were developed.
By 1994 it was clear that for astronomical applications the only way to increase sensitivity was to develop arrays of bolometers, but at that time no technology for the production of monolithic arrays was yet available \cite{richards_bolometers_1994}.

The development of voltage-biased superconducting Transition-Edge-Sensor detectors enabled the development of large-scale detector array.
These detectors are based on the use of thin superconducting films as the bolometer's thermometer element, and can fabricated at large scale using standard lithographic techniques.
Their operation and advantages for array-scale operation are described in \chapterref{c:tes}.
To readout these detector arrays, several groups have developed \SQUID-based multiplexed readout systems \cite{tdm,berkeley fdm, europe}.
This technology is now mature and is routinely deployed on both ground and balloon-borne experiments in arrays containing up to 10,000 bolometric detectors \cite{holland_scuba-2:_2013}.

The development of this technology offers new opportunities for passive imaging for security and other applications.
Specifically, it is now possible to develop focal planes capable of video-rate imaging with temperature resolution of \SI{100}{\mK} or below.
This thesis describes the design and development of the \NIST\ \Imager, a system developed to perform detection of concealed weapons at distances of \SIrange{26}{28}{\m} by producing video-rate images at \SI{350}{\GHz}.
This chapter provides an overview of millimeter-wavelength imaging and the problems that our system is intended to solve.
xxx need to say something about how this chapter justifies the need for cryogenic detectors.

\section{Security Imaging}

The frequency range \SIrange{100}{1000}{\GHz} is attractive for detection of concealed weapons or contraband because common clothing materials have high transmission in this range \cite{bjarnason_millimeter-wave_2004}.
In general, as shown in \figref{fig:ch1-clothes-atmos-trans}, transmission through clothing steadily decreases as frequency increases.
This trend tends to push systems toward lower frequencies.
An example likely familiar to readers are the L3 Provision systems operating at airports within the USA.
These are ``holographic radar'' systems operating at ~\SI{30}{\GHz}, intended for close-range portal screening, and based on technology developed at the Pacific Northwest National Lab (PNNL) \cite{sheen_cylindrical_1998,mcmakin_dual-surface_2009}.
For applications in which it is acceptable to require individuals to pass through and pause at a particular location, these systems have excellent image quality.
Although the ProVision system only takes still images with \abt{\SI{2}{s}} image acquisition time, similar portal screen systems with video-rate capabilities are also under development \cite{lyons_reflect-array_2013}.

\begin{figure*}
\centering
\includegraphics{drawings/ch1-clothes-bjarnason.pdf}
\includegraphics{drawings/ch1-atmos-trans.pdf}
\caption[Clothing and Atmospheric Transmission vs Frequency]{
  \textbf{Left}
  Plot of clothing transmission vs frequency.
  Taken from \cite{bjarnason_millimeter-wave_2004}.
  As frequency increases, transmission through all kinds of clothing decreases.
  The \SI{-10}{\dB} observation band of the \Imager\ is highlighted (\SIrange{318}{376}{\GHz}).
  \textbf{right}
  Plot showing a model of zenith atmospheric transmission on the summit of Mauna Kea, assuming \SI{0.5}{\mm} of precipitable water vapor.
  The plot is illustrative only, demonstrating the presence of atmospheric transmission windows and the general trend of worse transmission at higher frequencies.
  The data was obtained using the Caltech Submillimeter Observatory (CSO) Atmospheric Transmission Interactive Plotter \cite{darek_lis_cso_????}, which is based on a published model \cite{pardo_atmospheric_2001}.
}
\label{fig:ch1-clothes-atmos-trans}
\end{figure*}

But there are other applications and operational scenarios in which portal screening systems are not feasible.
One example is detection of suicide bomb belts, a scenario in which is it desirable to make a detection while the person being observed is some distance away.
In these scenarios it is not reasonable to expect observation targets to be stationary, so video-rate imaging is also required.
These applications are generally referred to as ``standoff'' detection because the imaging system ``stands off'' some distance from the target being imaged.
For these applications the choice of optical frequency is less clear than for portal imaging.
As shown in \figref{fig:ch1-clothes-atmos-trans}, not only does transmission through clothing fall with increasing frequency, but transmission through the atmosphere does as well, although due to atmospheric windows and line features the trend is not monotonic.
Nevertheless, it is clear that lower frequencies are better for transmission through the atmosphere.

However, lower frequencies also imply worse spatial resolution for a fixed optical aperture size.
The angular resolution achievable by an aperture of diameter $D$ observing at wavelength $\lambda$ is, by the Raleigh criterion,
\begin{equation} \label{eqn:ch1-raleigh}
  \Theta \sim 1.2 \lambda / D.
\end{equation}
For portal screening this is not prohibitive; the L3 Provision system has an effective aperture of \SI{1.7}{\m} \cite{mcmakin_dual-surface_2009}.
But for standoff distances of \SI{10}{\m} or more, a system operating at \SI{30}{\GHz} would need an aperture of size \abt{\SI{8}{\m}} in order to achieve \SI{1}{\cm} resolution.
This strongly drives choice of frequency for standoff detection to higher frequencies.

Security imaging systems for standoff applications broadly fall into two categories: ``active'' and ``passive''.
Active systems actively illuminate the target to be images with light and use the reflected light to obtain an image.
Passive system detect thermal blackbody emission that are naturally emitted by all objects.
Because the temperatures of objects being viewed are so low (\abt{\SI{300}{\K}}), active systems would seem to have an inherent signal-to-noise advantage over passive systems.
But the phenomenology of active imaging suffers from two problems that to-date have allowed passive imaging system to generally exceed the image quality achieved with active systems.

\subsection{Active Standoff Imaging}

Both problems stem from the fact that active imaging systems use single-moded coherent sources of light, somewhat analogous to a laser point operating at visible frequencies \cite{petkie_active_2008}.
The first problem is generally referred to as ``specular reflections''.
This refers to the fact that the intensity of light that is reflected off of the target and subsequently detected by the active imaging system is strongly dependent on the angle of the target relative to the illuminating beam.
This leads to strong highlight areas in active images which can be \SI{40}{\dB} or more higher than neighboring areas, making images difficult to interpret.
For a concrete example, the reader can imagine the way that sunlight at just the right angle can reflect off of the fender of another car on the road, making it difficult for their eyes to interpret what they are looking at.

The second problem is known as ``speckle''.
When a coherent light source is diffusely reflected from a surface which varies on distance scales comparable to or larger than a wavelength, some areas of the surface will randomly be oriented more favorably than others for reflecting light back to the system.
This leads to a random distribution of bright spots in the image known as speckle \cite{goodman_fundamental_1976}.
This phenomena acts as a kind of noise in active images, and in practice the signal-to-noise ratio of active imaging systems is often limited by speckle rather than noise inherent in the detection system itself.

The active imaging community has long been aware of these issues and is working actively to address them.
One recent approach uses modulated multi-moded illumination to avoid these issues \cite{petkie_multimode_2012,patrick_elimination_2012}.
Another approach is to use the active system as a radar system rather than an intensity detector.
Because the radar system detects phase differences rather than intensity differences, it should in theory be less susceptible to specular reflections and speckle.

The active imaging system that at this time is closest to producing video-rate imaging largely free of specular-reflection and speckle artifacts is the \SI{340}{\GHz} radar system developed at the Jet Propulsion Laboratory \cite{cooper_thz_2011}.
This system operates at standoff distances of \SI{16}{\m} achieving a spatial resolution of \SI{1}{\cm} at 4 frames per second.

\section{Required Image Noise for Passive Imaging} \label{sec:ch1-netd-reqs}

Passive imaging does not suffer from the problems of specular reflection and speckle.
Rather, the primary challenge for passive imaging is implementing sufficiently sensitive detectors to achieve low-noise images at video frame rates.
One commonly used figure-of-merit for noise in a passive imaging system is the Noise Equivalent Temperature Difference (\NETD) of the image, defined as the difference in the temperature distribution at the observation target which can not be distinguished from noise in the image.
I was unable to find a detailed study of the required \NETD\ for detection of concealed weapons or other contraband, but the ``lore'' of the field is that for non-metallic concealed threats in an indoor environment, image signals are \SIrange{0.5}{1.0}{\K}, with \SI{200}{\mK}  better \NETD\ required for detection.
The remainder of this section justifies this lore quantitatively.

To estimate the required \NETD\ for a passive imaging system we must investigate both the required signal-to-noise for detection and the expected contrast (signal) in passive imaging scenarios.
The required signal-to-noise ratio for object detection depends on the size of the object \cite{steven_w._smith_scientist_1997}.
This is explored in a simple way in \figref{fig:ch1-sn}.
This figure shows a sequence of simulated images with a \SI{22.5}{\cm} (\SI{9}{in}) knife in the middle left if the image, and a $2\times2$ hot pixel block in the upper right.
At a signal to noise ratio of 1, the knife is barely visible if you know where to look for it.
At a signal-to-noise ratio of 2 the knife becomes visible but the much smaller block is not.
The block can barely be made out at \SN\ 4, and at \SN\ 6 both block and knife are clearly visible.
Based on this I assume a required \SN\  of 4 for detection; this means that in order to reliably detect an object with contrast \SI{1}{\K} in an image, the required \NETD\ is \SI{0.25}{\K}.

\begin{figure*}
\centering
\includegraphics{drawings/ch1-sn.pdf}
\caption[Signal-to-noise ratio for object detection]{
  Exploration of signal-to-noise ratio (\SN) required for object detection.
  Each image contains $100 \times 100$ pixels.
  In the middle left of each image is a simple \SI{22.5}{\cm} model of a knife.
  In the upper right a $2\times2$ block of pixels has been set to be ``hot''.
  Gaussian noise was added to each image with a level appropriate for the \SN\ listed in the title.
  At $\SN\ = 1$ the knife is perhaps visible if you know where to look but the block is not.
  At $\SN\ = 2$ the knife become visible but the much smaller block in not.
  The block begins to become visible at $\SN\ = 4$, and is clearly visible at $\SN\ = 6$.
}
\label{fig:ch1-sn}
\end{figure*}

The contrast in a passive image is set by the temperatures, emissivities and transmittivities of the objects being imaged, along with the temperature of the ambient light.
\figref{fig:ch1-t-tot} depicts the situation schematically.
We consider an object at temperature $T_{1}$ with emissivity $\epsilon_1$, covered by an object at temperature $T_{cov}$ and transmittivity $\tau_{cov}$.
The entire scene is illuminated by blackbody radiation at temperature $T_{amb}$.
We assume that the covering objects has no reflection, so that $\epsilon_{cov} = 1 - \tau_{cov}$.
The total temperature seen by an observer looking at the object through the covering will be
\begin{equation} \label{eqn:ch1-t-tot}
  T_{tot,1} = (1 - \epsilon_{1}) \tau_{cov}^2 T_{amb} + 
           (1 + \tau_{cov}(1 - \epsilon_{1}))(1-\tau_{cov}) T_{cov} + 
           \tau_{cov} \epsilon_{1} T_1
\end{equation}
A second object behind the cover with temperature $T_2$ and emissivity $\epsilon_2$ will appear to have a temperature given by \eqnref{eqn:ch1-t-tot} with the subscript 1 replaced by 2 everywhere.
The contrast seen between these two objects will then be
\begin{equation} \label{eqn:ch1-delta-t}
  \Delta T_{1,2} = \tau_{cov} \left[ (\epsilon_2 - \epsilon_1) \tau_{cov} T_{amb} + 
                                    (\epsilon_2 - \epsilon_1) (1-\tau_{cov}) T_{cov} + 
                                    (T_1 \epsilon_1 - T_2 \epsilon_2) \right]
\end{equation}
This equation shows that in order for contrast to appear in the image, we require difference in temperature or emissivity between the two objects, or both.

\begin{figure*}
\centering
\includegraphics[width=6in]{images/ch1-t-tot.png}
\caption[Apparent temperature of a covered illuminated object]{
  Schematic showing total temperature seen looking at a object at temperature $T_1$ with emissivity $\epsilon_1$ through a cover at temperature $T_{cov}$ with transmittivity $\tau_{cov}$, all illuminated by ambient temperature $T_{amb}$.
  The object is assumed to have no transmission and the cover to have no reflection.
  The black arrows indicate transmission of ambient light through the cover, reflecting off the object, and passing back through the cover to the detecting system.
  The red arrows show the emission of the cover, which reaches the detector both directly and after reflecting off the object.
  The blue arrows show the emission of the object itself.
}
\label{fig:ch1-t-tot}
\end{figure*}

We consider the case where the objects have the same temperature but different emissivities.
This would be the case for objects strapped next to skin, underneath clothing for a period of time so that objects come into thermal equilibrium with the body.
In this case \eqnref{eqn:ch1-delta-t} reduces to
\begin{equation}
  \Delta T_{1,2} = \tau_{cov} (\epsilon_1 - \epsilon_2) \left[ (1 - \tau_{cov}) (T - T_{cov}) + \tau_{cov} (T - T_{amb}) \right].
\end{equation}
In security screening scenarios we would typically expect $T > T_{cov} > T_{amb}$, so the rightmost factor will be positive.
If we further assume that $T_{cov}$ is midway between $T$ and $T_{amb}$, then this further simplifies to
\begin{equation} \label{eqn:ch1-delta-t-simple}
  \Delta T_{1,2} = \frac{1}{2}\tau_{cov}(1+\tau_{cov}) (\epsilon_1 - \epsilon_2) (T - T_{amb}).
\end{equation}
The factor of $\frac{1}{2}$ represents that fact that at low $\tau$, under these assumptions, the contrast is dominated by the difference in temperature between the cover and the object, which is half of the difference between the object and the ambient light.

Outdoors, $T_{amb}$ will have contributions both from the ground (or structures/vegetation at ground level) and from the sky.
The sky temperature depends strongly on the weather \cite{appleby_standoff_2007}, and at \SI{350}{\GHz} could be as low as \SI{100}{\K} on a clear winter day, or approach \SI{310}{\K} on a hot day with high humidity.
Depending on the temperature and emission properties of local ground cover, under the worst-case scenario $T_{amb}$ outdoors could be very close to human body temperature.
This means that requirements on \NETD\ for outdoor imaging is the worst-case scenarios could easily be \SI{200}{\mK} or better, and motivates the search for imaging systems with \NETD\ at or below \SI{100}{\mK} for outdoor applications. 

Indoors, $T_{amb}$ will be determined by the temperature of the room, typically $\SI{295}{\K} = \SI{72}{\fahrenheit}$.
We can take as a challenging scenario the detection of plastic explosives hidden beneath a woolen sweater.
From \cite{bjarnason_millimeter-wave_2004} we use $\tau_{cov} = 0.5$ and from \cite{appleby_standoff_2007} we can take $\epsilon_{explosive} - \epsilon_{skin} = 0.08$, a case where the emissivity of the explosive is higher than that of skin.
\eqnref{eqn:ch1-delta-t-simple} gives $\Delta T_{1,2} = \SI{0.45}{\K}$, or a required \NETD\ of \SI{112.5}{\mK} assuming a required \SN\ of 4.
If the explosive material is cooler than body temperature than the requirements on \NETD\ will be eased, but it is the worst-case scenarios that should design system requirements.

From this we conclude that \NETD\ values of \SI{100}{\mK} or lower are required for the most challenging passive imaging scenarios.
The question remains as to what technology is capable of reaching this level of sensitivity.

\section{Passive Imaging Technology}

One option for passive imaging is the use of coherent detectors.
Many such systems have been described in the literature, and several commercial solutions are available.
However, to-date these systems suffer from an inability to achieve video-rate \NETD\ values low enough to meet the requirements discussed in \sectionref{sec:ch1-netd-reqs}.
Due to lack of available components, these systems operate at \SI{90}{\GHz} or below, which makes high-resolution imaging at distances over \SI{5}{\m} impractical due to the large aperture sizes required.

xxx need to list systems here!

Several companies have built passive imaging systems at 90 GHz based on this technology.
A typical example is the Microsemi GEN 2\footnote{%
Microsemi Corporation, Aliso Viejo, CA. This technology was acquired from Brijot systems in 2011.}, which works at \SI{90}{\GHz}.
This system achieve \SI{5}{\cm} resolution at 4--12 frames per second.
Standoff distance is a few meters and \NETD\ is not quoted.
While radiometric systems such at this perform better than micro-bolometer based systems, the \NETD\ is still not low enough forxxx, and the optical frequency is too low for acceptable resolution for standoff imaging at \SI{10}{\m} or greater (xxx need to justify this better).
Radiometric systems at \SI{350}{\GHz} to-date have poorer performance than at \SI{90}{\GHz}, so moving to higher frequencies in order to improve spatial resolution will lead to worse noise performance.

A second option is the use of bolometric (thermal) detector.
A bolometer is an optical power detector; it is sensitive to the square of the incident electromagnetic field.
Bolometers are characterized by a Noise Equivalent Power (\NEP), defined as the detected signal power equal to standard deviation of the noise in a \SI{1}{\Hz} post-detection bandwidth.
To convert the \NEP\ of the detectors to an \NETD\ for an image we must first have a conversion between detected optical power and source temperature.
For a detector sensitive to $M$ spatial modes of the electromagnetic field and with total optical efficiency $\eta_{tot}$, the optical power detected per unit optical frequency is given by the Planck law in the form
\begin{equation} \label{eqn:ch1-planck}
  P_{\nu}(\nu,T) \Delta \nu = \eta_{tot} M h \nu \frac{1}{e^{\frac{h \nu}{k_B T}} - 1} \Delta \nu,
\end{equation}
where $h$ is Planck's constant and $k_B$ is Boltzmann's constant and $T$ is the temperature of the source.
For detection of light around \SI{350}{\GHz}, and source temperatures in the range \SIrange{50}{300}{\K}, the Raleigh-Jeans approximation
\begin{equation}
  \frac{1}{e^{\frac{h \nu}{k_B T}} - 1} \approx \frac{k_B T}{h \nu}
\end{equation}
holds to within \SI{20}{\percent}, so that the total optical power in an optical bandwidth reduces to 
\begin{equation}
  P_{\nu} = \eta_{tot} M k_B T \Delta \nu.
\end{equation}
This allows us to convert a detector \NEP\ to a Noise Equivalent Temperature (\NET) via
\begin{equation}
  NET = \frac{NEP}{\eta_{tot} M k_B \Delta \nu}.
\end{equation}

To convert this detector \NET\ to an \NETD\ for an image, we make the assumptions that noise for each detector can be modeled as an uncorrelated Gaussian noise source, so that the noise in for a given pixel in the image will be given by the \NET\ divided by the square root of twice the integration time for that pixel across all detectors\footnote{%
  The factor of two accounts for the fact that \NEP\ is defined to give the total variance of a signal when integrated to only the Nyquist frequency, whereas the full bandwidth up to the sampling frequency in available for reducing noise.
}. 
We consider an image covering an area $A$ with square pixels of side length $s$, produced by an imaging system with $N$ detectors and video frame rate of $FPS$.
The integration time per pixel will then be given by
\begin{equation}
  \frac{N / FPS}{A / s^2},
\end{equation}
and so the \NETD\ of each image will be
\begin{align}
  NETD & = \frac{NET}{\sqrt{ 2 \frac{N / FPS}{A / s^2}}} \\
       & = \frac{NEP}{\eta_{tot} M k_B \Delta \nu} \frac{1}{s} \sqrt{\frac{A\,FPS}{2 N}} .
       \label{eqn:ch1-netd-defn}
\end{align}

Typical \NEP\ values for uncooled micro-bolometers in the millimeter-wave region are \Pnoisep{10}--\Pnoisep{100} \cite{nemarich_microbolometer_2005}.
In order to turn this \NEP\ into an \NETD\ we must make some assumptions.
In order to achieve sufficient spatial resolution detectors at these wavelengths are typically sensitive to 2 modes, one per polarization, so we set $M = 2$.
We can generously assume one detector per image pixel, so that $\sqrt{A/N}/s = 1$.
A very good optical efficiency would be $\eta_{tot} = 0.5$, and a typical bandwidth at \SI{350}{\GHz} is \SI{35}{\GHz}.
The minimum frame rate for video imaging around 6 frames per second.
Plugging these numbers into \eqnref{eqn:ch1-netd-defn} leads to an \NETD\ of \SI{50}{\K} for the best-case scenario of \Pnoisep{10} \NEP.
This is more than two orders of magnitude higher than the requirements discussed in \sectionref{sec:ch1-netd-reqs}.
It is clear that uncooled bolometer arrays can not achieve the required performance.

The conclusion is that to achieve sufficient \NETD\ and spatial resolution for passive standoff imaging at distances of \SI{10}{\m} and greater, room temperature detectors to-date do not have sufficient sensitivity.
For this reason groups working on this problem have turned to cooled bolometer technology.
The next section provides an overview of the efforts in this direction to-date.

%% Keepig the following for now, because I don't really believe it,
%% and it certainly oversimplifies matters. Near-IR bolometers are viewing the peak of the
%% 300 K blackbody spectrum, so they are not in the RJ limit, so the
%% equations above don't really apply. In particular, it's not clear
%% to me whether NETD really scales with the inverser of the optical
%% bandwidth. Maybe at some point I could think about this more
%% carefully, but for now I'm removing it from the text.
% 
% Uncooled bolometers can provide excellent images in the near IR
% A typical example is the IR-TCM HD 1024 Thermography Camera Module\footnote{Jenoptik AG, Jena, Germany}, whicih produces $1024 \times 768$ pixel images with an \NETD\ of \SI{50}{\mK} at 30 frames per second using an uncooled microbolometer focal plane array.
% But uncoooled bolometer arrays can achieve this performance in the near IR because the available optical bandwidth is much larger than what is available in the \SIrange{100}{1000}{\GHz} range.
% The IR-TCM HD 1024 detects the \SIrange{7.5}{14}{\um} range, which is an optical bandwidth of \SI{19}{\THz}, a factor of \abt{530} larger than that available to the \Imager.
% Even after reducing the number of pixels in an image to a $100 \times 100$, and the frame rate to 6 frames per second, the \NETD\ from this uncooled microbolometer system would be \abt{27} times worse at \SI{350}{\GHz} than in the near-IR.

\section{Cooled Bolometer Imaging Systems}

Aside from the work described in this thesis, two groups are also working on cooled bolometer passive imaging systems.

xxx easier at work b/c of references.
